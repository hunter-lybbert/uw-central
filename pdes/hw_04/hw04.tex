\documentclass[10pt]{amsart}
\usepackage[margin=1.4in]{geometry}
\usepackage{amssymb,amsmath,enumitem,url, bm}
\usepackage{graphicx,subfig}
\graphicspath{ {./images/} }
\usepackage{cancel}

\newcommand{\D}{\mathrm{d}}
\newcommand{\I}{\mathrm{i}}
\DeclareMathOperator{\E}{e}
\DeclareMathOperator{\OO}{O}
\DeclareMathOperator{\oo}{o}
\DeclareMathOperator{\erfc}{erfc}
\DeclareMathOperator{\real}{Re}
\DeclareMathOperator{\imag}{Im}
\usepackage{tikz}
\usepackage[framemethod=tikz]{mdframed}
\theoremstyle{nonumberplain}

\mdtheorem[innertopmargin=-5pt]{sol}{Solution}
%\newmdtheoremenv[innertopmargin=-5pt]{sol}{Solution}

\begin{document}
\pagestyle{empty}

\newcommand{\mline}{\vspace{.2in}\hrule\vspace{.2in}}

\noindent
\text{Hunter Lybbert} \\
\text{Student ID: 2426454} \\
\text{05-01-25} \\
\text{AMATH 503} \\
% header containing your name, student number, due date, course, and the homework number as a title.

\title{\bf {Homework 4} }


\maketitle
\noindent
Exercises come from \textit{Introduction to Partial Differential Equations by Peter J. Olver} as well as supplemented by instructor provided exercises.
\mline
\begin{enumerate}[label={\bf {\arabic*}:}]
\item We consider the following IBVP:
$$
\begin{cases}
u_t = u_{xx}, &x \in (0, 1), t> 0, \\
u_x(0, t) + 2u(0, t) = 0, u_x(1, t) - 2u(1, t) = 0, &t > 0, \\
u(x, 0) = \phi(x), &x \in (0, 1).
\end{cases}
$$
Solve this IBVP by using separation of variables and analyze the long-term behavior of the solution as $t \rightarrow +\infty$
\\

\noindent
\textit{Solution:} \\
We begin by assuming the equation $u(x, t)$ takes on the form
$$u(x, t) = X(x)T(t)$$
hence, we assume
\begin{align*}
X(x)T^\prime(t) &= X^{\prime\prime}(x)T(t) \\
\frac{T^\prime(t)}{T(t)} &= \frac {X^{\prime\prime}(x)} {X(x)} = -\lambda.
\end{align*}
This gives use the equations
\begin{align}
T^\prime(t) + \lambda T(t) &= 0 \label{eq:eq_t}, \\
X^{\prime\prime}(x) + \lambda X(x) &= 0 \label{eq:eq_x}.
\end{align}
We first solve \eqref{eq:eq_x} but have to consider common cases for $\lambda$ less than, equal to, and greater than 0.
We begin with $\lambda = 0$ then we have $X(x) = Ax +B$.
Our boundary conditions
\begin{align*}
u_x(0, t) + 2u(0, t) &= 0, \\
u_x(1, t) - 2u(1, t) &= 0,
\end{align*}
where $ t > 0$.
Using $A + 2(A(0) + B) = 0 \implies A = -2B$
We also need $A - 2(A(1) + B) = 0 \implies - A - 2B = 0 \implies A = -2B$, again.
Hence,
$$X(x) = A \left( x - \frac 1 2 \right). $$
Continuing on with the case where $\lambda >  0$.
Then we have
$X(x) = A\cos(\sqrt \lambda x) + B\sin(\sqrt \lambda x) $
which using our same boundary conditions now gives
\begin{align*}
u_x(0, t) + 2u(0, t) &= 0 \\
\cancel{-A \sqrt \lambda \sin(\sqrt \lambda 0)} + B \sqrt \lambda \cos(\sqrt \lambda 0) + 2 \left(  A\cos(\sqrt \lambda 0) + \cancel{ B\sin(\sqrt \lambda 0)} \right) &= 0 \\
B \sqrt \lambda + 2A &= 0 \implies
A &= - \frac {\sqrt \lambda} 2 B
\end{align*}
and then for $u_x(1, t) - 2u(1, t) = 0$ we have
\begin{align*}
-A \sqrt \lambda \sin(\sqrt \lambda ) + B \sqrt \lambda \cos(\sqrt \lambda ) - 2 \left(  A\cos(\sqrt \lambda) + B\sin(\sqrt \lambda ) \right) &= 0 \\
-A \sqrt \lambda \sin(\sqrt \lambda ) + B \sqrt \lambda \cos(\sqrt \lambda ) - 2 A\cos(\sqrt \lambda) - 2B\sin(\sqrt \lambda ) &= 0 \\
(- 2B - A \sqrt \lambda) \sin(\sqrt \lambda ) + (B \sqrt \lambda - 2A) \cos(\sqrt \lambda ) &= 0 \\
(B \sqrt \lambda - 2A) \cos(\sqrt \lambda ) &= (2B + A \sqrt \lambda) \sin(\sqrt \lambda ) \\
\frac {B \sqrt \lambda - 2A}{2B + A \sqrt \lambda} &= \tan(\sqrt \lambda ) \\
\frac {2 B \sqrt \lambda}{2B + A \sqrt \lambda} &= \tan(\sqrt \lambda ) \\
\frac {2 B \sqrt \lambda}{2B - \frac 1 2 B  \lambda} &= \tan(\sqrt \lambda ) \\
\frac {2 \sqrt \lambda}{2 - \frac 1 2  \lambda} &= \tan(\sqrt \lambda ) \\
\frac {4 \sqrt \lambda}{4 - \lambda} &= \tan(\sqrt \lambda ).
\end{align*}

\noindent
Now for the final case where  $\lambda < 0$ we have $X(x) = A\E^{\sqrt \lambda x} + B \E^{ - \sqrt \lambda x}$.
Now we have to use our boundary conditions so
\begin{align*}
\sqrt \lambda A\E^{\sqrt \lambda 0} - \sqrt \lambda B \E^{ - \sqrt \lambda 0} + 2\left(  A\E^{\sqrt \lambda 0} + B \E^{ - \sqrt \lambda 0} \right) &= 0 \\
\sqrt \lambda A - \sqrt \lambda B + 2 (A + B) &= 0 \\
(2 + \sqrt \lambda)A + (2 - \sqrt \lambda) B &= 0 \\
A = \frac {\sqrt \lambda - 2}{\sqrt \lambda + 2} B.
\end{align*}
Next we get
\begin{align*}
\sqrt \lambda A\E^{\sqrt \lambda} - \sqrt \lambda B \E^{ - \sqrt \lambda} - 2\left(  A\E^{\sqrt \lambda} + B \E^{ - \sqrt \lambda} \right) &= 0 \\
(\sqrt \lambda - 2)A \E^{\sqrt \lambda}  &= (\sqrt \lambda + 2) B \E^{ - \sqrt \lambda} \\
(\sqrt \lambda - 2)\frac {\sqrt \lambda - 2}{\sqrt \lambda + 2} B \E^{\sqrt \lambda}  &= (\sqrt \lambda + 2) B \E^{ - \sqrt \lambda} \\
(\sqrt \lambda - 2)^2 B \E^{\sqrt \lambda}  &= (\sqrt \lambda + 2)^2 B \E^{ - \sqrt \lambda}
\end{align*}
Next we carefully FOIL out the quadratic terms and move everything to one side
\begin{align*}
(\sqrt \lambda - 2)^2 B \E^{\sqrt \lambda}  &= (\sqrt \lambda + 2)^2 B \E^{ - \sqrt \lambda} \\
(\lambda -4 \sqrt \lambda + 4) B \E^{\sqrt \lambda}  &= (\lambda +4 \sqrt \lambda + 4) B \E^{ - \sqrt \lambda} \\
(\lambda -4 \sqrt \lambda + 4) B \E^{\sqrt \lambda} - (\lambda +4 \sqrt \lambda + 4) B \E^{ - \sqrt \lambda} &= 0 \\
\lambda B \E^{\sqrt \lambda} - \lambda B \E^{- \sqrt \lambda} -4 \sqrt \lambda B \E^{\sqrt \lambda} - 4\sqrt \lambda B \E^{- \sqrt \lambda} + 4 B \E^{\sqrt \lambda} - 4 B \E^{- \sqrt \lambda} &= 0 \\
\lambda B 2 \sinh \sqrt \lambda -4 \sqrt \lambda B (2\cosh \sqrt \lambda) + 4 B (2 \sinh \sqrt \lambda) &= 0 \\
\lambda B 2 \sinh \sqrt \lambda + 4 B (2 \sinh \sqrt \lambda) &= 4 \sqrt \lambda B (2\cosh \sqrt \lambda)  \\
(\lambda + 4 )B 2 \sinh \sqrt \lambda &= 4 \sqrt \lambda ( B 2\cosh \sqrt \lambda)  \\
\frac{B 2 \sinh \sqrt \lambda}{B 2\cosh \sqrt \lambda} &= \frac{4 \sqrt \lambda }{\lambda + 4 }  \\
\tanh \sqrt \lambda &= \frac{4 \sqrt \lambda }{\lambda + 4 }.
\end{align*}


\newpage

\item
\begin{enumerate}
\item Consider the following IBVP:
$$
\begin{cases}
(x^2\phi^\prime)^\prime + \lambda \phi = 0, & 1 < x < 2, \\
\phi(1) = 0 = \phi(2).
\end{cases}
$$
Figure out $p, q, w, h_1$ and $h_2$.
Write down the properties satisfied by eigenvalues and eigenfunctions by Sturm-Liouville theorem. (e.g. orthogonality of eigenfunctions, completeness of basis, etc.)
Solve the eigenpairs $\left\{ (\lambda_k, \phi_k \right\}$. \\

\noindent
\textit{Solution:} \\
\textbf{TODO: }

\newpage


\item Then, use the eigenpairs to solve the following IBVP:
$$
\begin{cases}
u_t = (x^2u_x)_x - u, &1 < x< 2, t> 0, \\
u(1, t) = u(2, t) = 0, &t > 0, \\
u(x, 0) = f(x), &x \in (1, 2)
\end{cases}
$$
(Hint: for Euler's ODE: $aX^{\prime\prime} + bX^\prime + cX = 0$, we have the ansatz $X = x^r$ and the characteristic root equation $ar(r- 1) + br + x = 0$.
If $r_1 \neq r_2$, then $X = c_1x^{r_1} + c_2x^{r_2}$;
if $r_1 = r_2 = r$, then $X = c_1x^r + c_2x^r \log x$;
if $r = \nu + \I \mu$ is a complex root, then $X = c_1x^\nu \cos (\mu \log x) + c_2 x^\nu \sin(\mu \log x)$. ) \\

\noindent
\textit{Solution:} \\
\textbf{TODO} \\

\end{enumerate}

\newpage

\item Consider the following BVP:
$$
\begin{cases}
x^2y^{\prime\prime} + xy^\prime + (x^2\lambda^2 - n^2)y = 0, &x \in (0, L), \\
y^\prime(0) = 0 or y(0) = 0, y(L) = 0,
\end{cases}
$$
where $\lambda$ and $n$ are real numbers.
$L > 0$ is a constant, as well.

\begin{enumerate}
\item rewrite the BVP as the sturm-liouville form and write down the definition of $p, q, w, h_1,$ and $h_2$: \\

\noindent
\textit{Solution:} \\
\textbf{TODO:} \\

\item write down the orthogonality conditions satisfied by the eigenfunction. \\

\noindent
\textit{Solution:} \\
\textbf{TODO:} \\

\item We have the fact that
$$
J_n(x) := \sum_{k = 0}^\infty\frac {(-1)^k}{k!(n - k)! } \left( \frac  x 2 \right)^{n + 2k}, \quad n = 0, 1, ...
$$
are solutions to 
$$
\begin{cases}
xy^{\prime\prime} + y^\prime + (x - \frac{n^2} x )y = 0, &x \in (0, \infty), \\
y(\infty) = 0.
\end{cases}
$$
$J_n$ is named the $n$-th order Bessel function.
Plot the figures of $J_0, J_1,$ and $J_2$ by matlab or python.
We have the facts that $J_n(x)$ has infinitely many zeros $\nu_{nm}, m = 1, ...$ and $J_n$ is bounded as $r \rightarrow 0$.
By using $J_n(x)$ and $nu_{nm}$ to find all eigenpairs $\left\{\lambda_{nm}^2, y_{nm}(x) \right\}_{m = 1}^\infty$ to the given BVP. \\

\noindent
\textit{Solution:} \\
\textbf{TODO:} \\

\item Solve the following IBVP:
$$
\begin{cases}
u_t = \Delta u, &(x, y) \in B_a(0), t> 0, \\
u(x, t) = 0, &(x, y) \in \partial B_a(0), t> 0, \\
u(x, 0) = u_0(x, y), &(x, y) \in B_a(0),
\end{cases}
$$
where $B_a(0) \subset \mathbb R^2$ is the disc with radius $a > 0$.
(Hint: for (d), recall in the polar coordinate $(r, \theta)$, $\Delta u = u_{rr} + \frac 1 r u_r + \frac 1 {r^2} u_{\theta\theta}$.
Use separation of variables $u(r, \theta, t) = R(r)\Theta(\theta)T(t)$ to solve the IBVP.
$\Theta$ mode is 2$\pi$-periodic.) \\

\noindent
\textit{Solution:} \\
\textbf{TODO:} \\

\end{enumerate}

\newpage

\item Solve the following signaling problem: 
$$
\begin{cases}
u_t + cu_x = 0, & 0< x< + \infty, \\
u(0, t) = g(t), u(x, 0) = 0, &x \geq0,
\end{cases}
$$
where $c > 0$ is a constant. \\

\noindent
\textit{Solution:} \\
\textbf{TODO: } \\

\newpage

\item Olver 2.2.17 \\

\noindent
\textit{Solution:} \\
\textbf{TODO: } \\
\newpage


\item Olver 2.2.31 \\

\noindent
\textit{Solution:} \\
\textbf{TODO: } \\
\newpage

\item 
\begin{enumerate}
\item Solve the ODE:
$$
\frac {du}{ds} + u = 2\E^x
$$
by the integral factor method.
And use the same technique to solve the following damping heat equation:
$$
\begin{cases}
\nu_t = \nu_{xx} = \nu, &x\in(0, \pi), t> 0, \\
\nu(0, t) = \nu(\pi, t) = 0. &t > 0, \\
\nu(x, 0) = \nu_0(x), &x \in (0, \pi).
\end{cases}
$$

\noindent
(Hint: test the BHS of ODE against $\E^x$ and then integrate to solve it, where $\E^x$ is called the integral factor.) \\

\noindent
\textit{Solution:} \\
\textbf{TODO: } \\

\item Solving the following transport equation:
$$
u_t + tu_x = u, -\infty < x < + \infty, t > 0,
$$
with initial condition
$$
u(x, 1) = f(x), 0 \leq x \leq 1,
$$
where $f$ is continuous.
Compute $u(x, t)$ by the method of characteristics and find the subregion in $-\infty < x < + \infty, t> 0$, where the data on $t = 1$ determines this solution.
Plot this subregion. \\

\noindent
\textit{Solution:} \\
\textbf{TODO: } \\


\end{enumerate}

\end{enumerate}

\end{document}

%%% Local Variables:
%%% mode: latex
%%% TeX-master: t
%%% End:
