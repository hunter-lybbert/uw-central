\documentclass[10pt]{amsart}
\usepackage[margin=1.4in]{geometry}
\usepackage{amssymb,amsmath,enumitem,url, bm}
\usepackage{graphicx,subfig}
\graphicspath{ {./images/} }
\usepackage{cancel}

\newcommand{\D}{\mathrm{d}}
\newcommand{\I}{\mathrm{i}}
\DeclareMathOperator{\E}{e}
\DeclareMathOperator{\OO}{O}
\DeclareMathOperator{\oo}{o}
\DeclareMathOperator{\erfc}{erfc}
\DeclareMathOperator{\real}{Re}
\DeclareMathOperator{\imag}{Im}
\usepackage{tikz}
\usepackage[framemethod=tikz]{mdframed}
\theoremstyle{nonumberplain}

\mdtheorem[innertopmargin=-5pt]{sol}{Solution}
%\newmdtheoremenv[innertopmargin=-5pt]{sol}{Solution}

\begin{document}
\pagestyle{empty}

\newcommand{\mline}{\vspace{.2in}\hrule\vspace{.2in}}

\noindent
\text{Hunter Lybbert} \\
\text{Student ID: 2426454} \\
\text{05-22-25} \\
\text{AMATH 503} \\
% header containing your name, student number, due date, course, and the homework number as a title.

\title{\bf {Homework 6} }


\maketitle
\noindent
Exercises come from \textit{Introduction to Partial Differential Equations by Peter J. Olver} as well as supplemented by instructor provided exercises.
\mline
\begin{enumerate}[label={\bf {\arabic*}:}]

\item Solve the following wave equations by using D'Alambert's formula:
$$u_{tt} - 4u_{xx} = 0, -\infty < x < \infty, t > 0,$$
\begin{enumerate}
\item $u(x, 0) = \E^x, u_t(x, 0) = \sin(x)$. \\

\noindent
\textit{Solution:} \\
In order to use D'Alambert's formula we need to identify that
\begin{align*}
c &= 2, \\
u(x, 0) &= \E^x = f(x), \\
u_t(x, 0) &= \sin x = g(x).
\end{align*}
Therefore, applying the formula
$$
u(x, t) = \frac 1 2 \Big[ f(x - ct) + f(x + ct) \Big] + \frac 1{2c} \int_{x - ct}^{x + ct} g(z) dz
$$
we have
\begin{align*}
u(x, t) &= \frac 1 2 \Big[ f(x - ct) + f(x + ct) \Big] + \frac 1{2c} \int_{x - ct}^{x + ct} g(z) dz \\
	&= \frac 1 2 \Big[ \E^{(x - 2t)} + \E^{(x + 2t)} \Big] + \frac 1{4} \int_{x - 2t}^{x + 2t} \sin(z) dz
\end{align*}
Let's now calculate the integral on the right
\begin{align*}
\int_{x - ct}^{x + ct} \sin(z) dz &= -\cos(z) \Big|_{x - 2t}^{x + 2t} = -\cos(x + 2t) - (-\cos(x - 2t)) = \cos(x - 2t) - \cos(x + 2t)
\end{align*}
Therefore our final solution is 
$$
u(x, t) = \frac 1 2 \Big[ \E^{(x - 2t)} + \E^{(x + 2t)} \Big] + \frac 1{4} \Big[ \cos(x - 2t) - \cos(x + 2t) \Big]
$$
\qed \\


\item $u(x, 0) = \sin(x), u_t(x, 0) = \cos(2x)$. \\

\noindent
\textit{Solution:} \\
This time we have $f(x) = \sin(x)$ and $g(x) = \cos(2x)$ while $c = 2$ still.
Therefore the integral we need to calculate is
\begin{align*}
\int_{x - ct}^{x + ct} \cos(2z) dz &= \frac 1 2 \sin (2z) \Big|_{x - 2t}^{x + 2t} \\
	&= \frac 1 2 \Big( \sin(2x + 4t) - \sin(2x - 4t) \Big)
\end{align*}
Therefore, by D'Alambert's formula we have
$$
u(x, t) = \frac 1 2 \Big[ \sin(x - 2t) + \sin(x + 2t) \Big] + \frac 1 8 \Big[ \sin(2x + 4t) - \sin(2x - 4t) \Big]
$$
\qed \\
\newpage

\end{enumerate}

\newpage


\item Olver: 2.4.11 (c) \\
Solve the forced IVP
$$
\begin{cases}
u_{tt} - 4 u_{xx} = \cos 2t, &-\infty < x < \infty, t \geq 0\\
u(0, x) = \sin x, \\
u_t(0, x) = \cos x,
\end{cases}
$$

\noindent
\textit{Solution:} \\
Similar to problem 1 we want to identify that the functions $f$, $g$, and $F$ and the constant $c$ to use \textbf{Theorem 2.18} from Olver.
This time we also want to identify the force $F$, all together we have
\begin{align*}
c &= 2 \\
f(x) &= \sin x \\
g(x) &= \cos x \\
F(x, t) &= \cos 2t.
\end{align*}
Which gives us
\begin{align*}
u(x, t) &= \frac 1 2 \Big[ f(x - ct) + f(x + ct) \Big] + \frac 1{2c} \int_{x - ct}^{x + ct} g(z) dz + \frac 1{2c} \int_0^t \int_{x - c (t - s)}^{x + c(t - s)} F(y, s) \: dy \: ds \\
	&= \frac 1 2 \Big[ \sin(x - 2t) + \sin(x + 2t) \Big] + \frac 1 4 \int_{x - 2t}^{x + 2t} \cos (z) dz + \frac 1 4 \int_0^t \int_{x - 2(t - s)}^{x + 2(t - s)} \cos (2s) \: dy \: ds
\end{align*}
We will now calculate the necessary integrals beginning first with the integral over $\cos z$
$$
\int_{x - 2t}^{x + 2t} \cos (z) dz
	= \sin z \big|_{x - 2t}^{x + 2t}
	= \sin (x + 2t) - \sin(x - 2t)
$$
Next the integral over $\cos 2s$
\begin{align*}
\int_0^t \int_{x - 2(t - s)}^{x + 2(t - s)} \cos (2s) \: dy \: ds
	&= \int_0^t  \cos (2s) \int_{x - 2(t - s)}^{x + 2(t - s)} \: dy \: ds \\
	&= \int_0^t \cos (2s) y \Big|_{x - 2(t - s)}^{x + 2(t - s)} \: ds \\
	&= \int_0^t \cos (2s)\Big[(x + 2(t - s)) - (x - 2(t - s)) \Big] \: ds \\
	&= \int_0^t \cos (2s)\Big[ x + 2(t - s)  - x + 2(t - s) \Big] \: ds \\
	&= \int_0^t \cos (2s)4(t - s) \: ds \\
	&= 4\left[ t \int_0^t \cos (2s)ds - \int_0^t s \cos (2s) ds \right] \\
	&= 4\left[ \frac t 2 \sin (2t) - \int_0^t s \cos (2s) ds \right]
\end{align*}
Using integration by parts on the remaining integral we have
\begin{align*}
\int_0^t s \cos (2s) ds
	&= \frac 1 2 s \sin (2s) \Big|_0^t  - \int_0^t \frac 1 2 \sin (2s) ds \\
	&= \frac 1 2 t \sin (2t)  +  \frac 1 2 \cos (2s) \Big|_0^t \\
	&= \frac 1 2 t \sin (2t)  +  \frac 1 2 \cos (2t) - \frac 1 2 \\
	&= \frac 1 2 \left( t \sin (2t)  +  \cos (2t) - 1 \right).
\end{align*}
Combining these integral back up the chain of equalities we have the final solution
\begin{align*}
u(x, t) = \frac 1 2 \Big[ \sin(x - 2t) + \sin(x + 2t) \Big] + \frac 1 4 \Big[ \sin (x + 2t) - \sin(x - 2t) \Big] + \frac 1 2 \Big[ 1 - \cos (2t) \Big]
\end{align*}
\qed \\


\newpage


\item Separation of variables to solve
$$
\begin{cases}
u_{tt} = u_{xx} + \E^{-t}\sin(x), & 0 < x < \pi, t > 0 \\
u(x, 0) = \sin(3x), u_t(x, 0) = 0, & 0 < x < \pi, \\
u(0, t) = 1, u(\pi, t) = 0, & t > 0.
\end{cases}
$$

\noindent
\textit{Solution:} \\
\textbf{TODO}

\newpage

\item (Bonus question) Solve the following wave equation
$$
\begin{cases}
u_{tt} - 4 u_{xx} = 0, & 0 < x < \infty, 0 < t < \infty \\
u(0, t) = 1, & t > 0, \\
u(x, 0) = x, u_t(x, 0) = \E^x, & x \geq 0.
\end{cases}
$$

\noindent
\textit{Solution:} \\
\textbf{TODO}

\newpage


\item Separation of variables to solve
$$
\begin{cases}
u_{xx} + u_{yy} = 0, & 0 < x < \pi, 0 < y < \pi \\
u(0, y) = u_x(\pi, y) = u(x, 0) = 0 \\
u(x, \pi) = \sin \left( \frac x 2 \right) - 2 \sin \left( \frac {3x}{2} \right).
\end{cases}
$$

\noindent
\textit{Solution:} \\
\textbf{TODO}

\newpage

\item Olver: 4.3.34 (b) Solve the following boundary value problems for the Laplace equation on the annulus $1 < r < 2$ with
$$
\begin{cases}
u_{rr} + \frac 1 ru_{r} + \frac 1 {r^2} u_{\theta\theta} = 0 &\textbf{Is this right?} \\
u(1, \theta) = 0, u(2, \theta) = \cos \theta, \\
1 \leq r < 2, 0 \leq \theta < 2\pi 
\end{cases}
$$

\noindent
\textit{Solution:} \\
\textbf{TODO}

\newpage


\item (Bonus) Consider the following Laplace equation
$$
\begin{cases}
u_{rr} + \frac 1 ru_{r} + \frac 1 {r^2} u_{\theta\theta} = 0, & 0 \leq r < 1, 0 \leq \theta < 2\pi \\
u_r(1, \theta) + u(1, \theta) = \cos(2 \theta)
\end{cases}
$$
Use the method of separation of variables to find a solution. \\

\noindent
\textit{Solution:} \\
\textbf{TODO}

\newpage

\end{enumerate}

\end{document}

%%% Local Variables:
%%% mode: latex
%%% TeX-master: t
%%% End:
