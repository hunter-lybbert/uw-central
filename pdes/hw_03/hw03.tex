\documentclass[10pt]{amsart}
\usepackage[margin=1.4in]{geometry}
\usepackage{amssymb,amsmath,enumitem,url, bm}
\usepackage{graphicx,subfig}
\graphicspath{ {./images/} }
\usepackage{cancel}

\newcommand{\D}{\mathrm{d}}
\newcommand{\I}{\mathrm{i}}
\DeclareMathOperator{\E}{e}
\DeclareMathOperator{\OO}{O}
\DeclareMathOperator{\oo}{o}
\DeclareMathOperator{\erfc}{erfc}
\DeclareMathOperator{\real}{Re}
\DeclareMathOperator{\imag}{Im}
\usepackage{tikz}
\usepackage[framemethod=tikz]{mdframed}
\theoremstyle{nonumberplain}

\mdtheorem[innertopmargin=-5pt]{sol}{Solution}
%\newmdtheoremenv[innertopmargin=-5pt]{sol}{Solution}

\begin{document}
\pagestyle{empty}

\newcommand{\mline}{\vspace{.2in}\hrule\vspace{.2in}}

\noindent
\text{Hunter Lybbert} \\
\text{Student ID: 2426454} \\
\text{04-24-25} \\
\text{AMATH 503} \\
% header containing your name, student number, due date, course, and the homework number as a title.

\title{\bf {Homework 3} }


\maketitle
\noindent
Exercises come from \textit{Introduction to Partial Differential Equations by Peter J. Olver} as well as supplemented by instructor provided exercises.
\mline
\begin{enumerate}[label={\bf {\arabic*}:}]
\item Olver: 3.2.6 (a,c,e) \\

\noindent
\textit{Solution:} \\
\textbf{TODO:}

\newpage


\item Olver: 3.3.2 and 3.3.3 \\

\noindent
\textit{Solution:} \\
\textbf{TODO:}

\newpage


\item Olver: 3.2.55 \\
\begin{enumerate}
\item Find the complex Fourier series for $x\E^{\I x}$ \\

\noindent
\textit{Solution:} \\
First of all we define the complex Fourier series for a piecewise continuous real or complex function $f$ is the doubly infinite series
$$
f(x) \sim \sum_{k=-\infty}^{\infty} c_k \E^{\I k x}
$$
where the $c_k$ are given by
$$
c_k = \langle f, \E^{\I k x} \rangle = \frac 1 {2 \pi} \int_{-\pi}^{\pi} f(x) \E^{-\I k x} dx.
$$
Therefore, the bulk of our work here is to establish what the coefficients $c_k$ need to be.
In other words we need to calculate
\begin{align*}
c_k = \langle x\E^{\I x}, \E^{\I k x} \rangle
	&= \frac 1 {2 \pi} \int_{-\pi}^{\pi} x\E^{\I x} \E^{-\I k x} dx \\
	&= \frac 1 {2 \pi} \int_{-\pi}^{\pi} x\E^{\I x (1 - k)}dx.
\end{align*}
I believe integration by parts would be useful.
Let $u = x$ and let $dv = \E^{\I x (1 - k)}dx$ these then also give rise to $du = dx$ and $v = \frac 1 {\I(1 - k)} \E^{\I x (1 - k)}$, respectively.
Then we have
\begin{align*}
\int u dv &= uv - \int v du \\
\frac 1 {2 \pi} \int_{-\pi}^{\pi} x\E^{\I x (1 - k)}dx
	&= \frac 1 {2 \pi} \Bigg[ \left. \frac x {\I(1 - k)} \E^{\I x (1 - k)}\right|_{-\pi}^{\pi} - \int_{-\pi}^{\pi} \frac 1 {\I(1 - k)} \E^{\I x (1 - k)} dx \Bigg] \\
\end{align*}
Let's take the right hand piece by piece in order to keep the calculations clean.
First with the $uv$ term
\begin{align*}
uv &= \left. \frac x {\I(1 - k)} \E^{\I x (1 - k)}\right|_{-\pi}^{\pi} \\
	&= \frac \pi {\I(1 - k)} \E^{\I \pi (1 - k)} - \frac {(-\pi)} {\I(1 - k)} \E^{- \I \pi (1 - k)} \\
	&= \frac \pi {\I(1 - k)} \left( \E^{\I \pi (1 - k)} + \E^{- \I \pi (1 - k)} \right) \\
	&= \frac {2 \pi \cos (\pi (1 - k)) } {\I(1 - k)} \\
	&= - \frac {2 \pi \I \cos (\pi (1 - k)) } {1 - k}.
\end{align*}
Now we proceed with the integral on the right hand side
\begin{align*}
\int v du &= \int_{-\pi}^{\pi} \frac 1 {\I(1 - k)} \E^{\I x (1 - k)} dx \\
	&= \left. \frac 1 {(\I(1 - k))^2} \E^{\I x (1 - k)} \right|_{-\pi}^{\pi} \\
	&= \frac 1 {(\I(1 - k))^2} \E^{\I \pi (1 - k)} - \frac 1 {(\I(1 - k))^2} \E^{- \I \pi (1 - k)} \\
	&= \frac 1 {\I^2(1 - k)^2} \left(  \E^{\I \pi (1 - k)} - \E^{- \I \pi (1 - k)} \right) \\
	&= - \frac 1 {(1 - k)^2} \left(  2 \I \sin (\pi (1 - k)) \right) \\
	&= 0
\end{align*}
where the final equality holds due to the fact that $\sin(\pi (1 - k))$ is always 0 since $1 - k$ is an integer.
Using these in our initial IBP step we have
\begin{align*}
\frac 1 {2 \pi} \int_{-\pi}^{\pi} x\E^{\I x (1 - k)}dx
	&= \frac 1 {2 \pi} \Bigg[ \left. \frac x {\I(1 - k)} \E^{\I x (1 - k)}\right|_{-\pi}^{\pi} - \int_{-\pi}^{\pi} \frac 1 {\I(1 - k)} \E^{\I x (1 - k)} dx \Bigg] \\
	&= \frac 1 {2 \pi} \Bigg[ - \frac {2 \pi \I \cos (\pi (1 - k)) } {1 - k} \Bigg] \\
	&= - \frac {\I \cos (\pi (1 - k)) } {1 - k}.
\end{align*}
Thus we have calculated the $c_k$ to be
$$
c_k = - \frac {\I \cos (\pi (1 - k)) } {1 - k}
$$
and thus our Fourier series of the function $x \E^{\I x}$ is given by
$$
f(x) \sim \sum_{k=-\infty}^{\infty} - \frac {\I \cos (\pi (1 - k)) } {1 - k} \E^{\I k x}
$$
\qed \\

\item Use your result to write down the real Fourier series for $x\cos x$ and $x \sin s$ \\
\noindent
\textit{Solution:} \\
I am not confident on how to go from part a) to the results that are requested here but I do think it will be helpful to know
$a_k = c_k + c_{-k}$ and $b_k = \I (c_k - c_{-k})$.

\end{enumerate}

\newpage


\item Olver: 3.4.6 \\

\noindent
\textit{Solution:} \\
\textbf{TODO:}

\newpage


\item Olver: 3.5.29 \\

\noindent
\textit{Solution:} \\
\textbf{TODO:}

\newpage


\item Olver: 3.5.43 \\

\noindent
\textit{Solution:} \\
\textbf{TODO:}

\newpage

\item We consider the complex orthonormal basis
$$
\varphi_n = \frac 1 {\sqrt{2 \pi}} \E^{\I n x}
$$
where $n = 0, 1, -1, 2, -2, ...$.
Consider the function $f_a(x) = \E^{ax}$ with real number $a \neq 0$ and compute the Fourier coefficient
$$
\langle f_a, \varphi_n \rangle = \frac 1 {\sqrt{2 \pi}} \int_{-\pi}^{\pi} f_a(x) \E^{- \I n x} dx.
$$
Then prove the formula
$$
\sum_{n = 1}^\infty \frac {1}{a^2 + n^2} = \frac \pi {2a} \coth (\pi a)  - \frac 1 {2a^2}
$$
(Hint: Plancherel’s formula: the relation between $L^2$ norm of coefficients and $\langle f_a, f_a \rangle$.)
\\

\noindent
\textit{Solution:} \\
\textbf{TODO: }
\end{enumerate}

\end{document}

%%% Local Variables:
%%% mode: latex
%%% TeX-master: t
%%% End:
