\documentclass[10pt]{amsart}
\usepackage[margin=1.4in]{geometry}
\usepackage{amssymb,amsmath,enumitem,url, bm}
\usepackage{graphicx,subfig}
\graphicspath{ {./images/} }
\usepackage{cancel}

\newcommand{\D}{\mathrm{d}}
\newcommand{\I}{\mathrm{i}}
\DeclareMathOperator{\E}{e}
\DeclareMathOperator{\OO}{O}
\DeclareMathOperator{\oo}{o}
\DeclareMathOperator{\erfc}{erfc}
\DeclareMathOperator{\real}{Re}
\DeclareMathOperator{\imag}{Im}
\usepackage{tikz}
\usepackage[framemethod=tikz]{mdframed}
\theoremstyle{nonumberplain}

\mdtheorem[innertopmargin=-5pt]{sol}{Solution}
%\newmdtheoremenv[innertopmargin=-5pt]{sol}{Solution}

\begin{document}
\pagestyle{empty}

\newcommand{\mline}{\vspace{.2in}\hrule\vspace{.2in}}

\noindent
\text{Hunter Lybbert} \\
\text{Student ID: 2426454} \\
\text{04-24-25} \\
\text{AMATH 503} \\
% header containing your name, student number, due date, course, and the homework number as a title.

\title{\bf {Homework 3} }


\maketitle
\noindent
Exercises come from \textit{Introduction to Partial Differential Equations by Peter J. Olver} as well as supplemented by instructor provided exercises.
\mline
\begin{enumerate}[label={\bf {\arabic*}:}]
\item Olver: 3.2.6 (a,c,e) \\

\noindent
\textit{Solution:} \\
\textbf{TODO:}

\newpage


\item Olver: 3.3.2 and 3.3.3 \\
\begin{itemize}
\item 3.3.2 Find the Fourier series for the function $f(x) = x^3$.
If you differentiate your series, do you recover the Fourier series for $f^\prime(x) = 3x^2$?
If not, explain why not. \\

\noindent
\textit{Solution:} \\
We begin by calculating the coefficients $a_k$ and $b_k$.
\begin{align*}
a_k = \langle x^3, \cos kx \rangle &= \frac 1 \pi \int_{-\pi}^\pi x^3 \cos kx dx \\
	&= \frac 1 \pi \int_{-\pi}^\pi x^3 \cos kx dx
\end{align*}

\item 3.3.3 Repeat exercise 3.3.2 but starting with $f(x) = x^4$. \\
\noindent
\textit{Solution:} \\
\textbf{TODO:}

\end{itemize}

\noindent
\textit{Solution:} \\
\textbf{TODO:}

\newpage


\item Olver: 3.2.55 \\
\begin{enumerate}
\item Find the complex Fourier series for $x\E^{\I x}$ \\

\noindent
\textit{Solution:} \\
First of all we define the complex Fourier series for a piecewise continuous real or complex function $f$ is the doubly infinite series
$$
f(x) \sim \sum_{k=-\infty}^{\infty} c_k \E^{\I k x}
$$
where the $c_k$ are given by
$$
c_k = \langle f, \E^{\I k x} \rangle = \frac 1 {2 \pi} \int_{-\pi}^{\pi} f(x) \E^{-\I k x} dx.
$$
Therefore, the bulk of our work here is to establish what the coefficients $c_k$ need to be.
In other words we need to calculate
\begin{align*}
c_k = \langle x\E^{\I x}, \E^{\I k x} \rangle
	&= \frac 1 {2 \pi} \int_{-\pi}^{\pi} x\E^{\I x} \E^{-\I k x} dx \\
	&= \frac 1 {2 \pi} \int_{-\pi}^{\pi} x\E^{\I x (1 - k)}dx.
\end{align*}
I believe integration by parts would be useful.
Let $u = x$ and let $dv = \E^{\I x (1 - k)}dx$ these then also give rise to $du = dx$ and $v = \frac 1 {\I(1 - k)} \E^{\I x (1 - k)}$, respectively.
Then we have
\begin{align*}
\int u dv &= uv - \int v du \\
\frac 1 {2 \pi} \int_{-\pi}^{\pi} x\E^{\I x (1 - k)}dx
	&= \frac 1 {2 \pi} \Bigg[ \left. \frac x {\I(1 - k)} \E^{\I x (1 - k)}\right|_{-\pi}^{\pi} - \int_{-\pi}^{\pi} \frac 1 {\I(1 - k)} \E^{\I x (1 - k)} dx \Bigg] \\
\end{align*}
Let's take the right hand piece by piece in order to keep the calculations clean.
First with the $uv$ term
\begin{align*}
uv &= \left. \frac x {\I(1 - k)} \E^{\I x (1 - k)}\right|_{-\pi}^{\pi} \\
	&= \frac \pi {\I(1 - k)} \E^{\I \pi (1 - k)} - \frac {(-\pi)} {\I(1 - k)} \E^{- \I \pi (1 - k)} \\
	&= \frac \pi {\I(1 - k)} \left( \E^{\I \pi (1 - k)} + \E^{- \I \pi (1 - k)} \right) \\
	&= \frac {2 \pi \cos (\pi (1 - k)) } {\I(1 - k)} \\
	&= - \frac {2 \pi \I \cos (\pi (1 - k)) } {1 - k}.
\end{align*}
Now we proceed with the integral on the right hand side
\begin{align*}
\int v du &= \int_{-\pi}^{\pi} \frac 1 {\I(1 - k)} \E^{\I x (1 - k)} dx \\
	&= \left. \frac 1 {(\I(1 - k))^2} \E^{\I x (1 - k)} \right|_{-\pi}^{\pi} \\
	&= \frac 1 {(\I(1 - k))^2} \E^{\I \pi (1 - k)} - \frac 1 {(\I(1 - k))^2} \E^{- \I \pi (1 - k)} \\
	&= \frac 1 {\I^2(1 - k)^2} \left(  \E^{\I \pi (1 - k)} - \E^{- \I \pi (1 - k)} \right) \\
	&= - \frac 1 {(1 - k)^2} \left(  2 \I \sin (\pi (1 - k)) \right) \\
	&= 0
\end{align*}
where the final equality holds due to the fact that $\sin(\pi (1 - k))$ is always 0 since $1 - k$ is an integer.
Using these in our initial IBP step we have
\begin{align*}
\frac 1 {2 \pi} \int_{-\pi}^{\pi} x\E^{\I x (1 - k)}dx
	&= \frac 1 {2 \pi} \Bigg[ \left. \frac x {\I(1 - k)} \E^{\I x (1 - k)}\right|_{-\pi}^{\pi} - \int_{-\pi}^{\pi} \frac 1 {\I(1 - k)} \E^{\I x (1 - k)} dx \Bigg] \\
	&= \frac 1 {2 \pi} \Bigg[ - \frac {2 \pi \I \cos (\pi (1 - k)) } {1 - k} \Bigg] \\
	&= - \frac {\I \cos (\pi (1 - k)) } {1 - k}.
\end{align*}
Notice, since $\cos(\ell \pi) = \begin{cases} 1, &\text{if $\ell$ is even} \\ -1, &\text{if $\ell$ is odd} \end{cases}$, therefore, when $k$ is odd $1 - k$ is even but if $k$ is even then $1 - k$ is odd.
Hence
\begin{align*}
- \frac {\I \cos (\pi (1 - k)) } {1 - k} &= - \frac {\I (-1)^{(1 - k)} } {1 - k} \\
	&= \frac {\I (-1)^{(2 - k)} } {1 - k} \\
	&= \frac {\I (-1)^{k} } {1 - k}
\end{align*}
Thus we have calculated the $c_k$ to be
$$
c_k = \frac {\I (-1)^{k} } {1 - k}
$$
and thus our Fourier series of the function $x \E^{\I x}$ is given by
$$
f(x) \sim \sum_{k=-\infty}^{\infty} \frac {\I (-1)^{k} } {1 - k} \E^{\I k x}
$$
\qed \\

\item Use your result to write down the real Fourier series for $x\cos x$ and $x \sin s$ \\
\noindent
\textit{Solution:} \\
Notice we can rewrite the previous Fourier series as
\begin{align*}
f(x) &\sim \sum_{k=-\infty}^{\infty} \frac {\I (-1)^{k} } {1 - k} \E^{\I k x} \\
x \E^{\I x} &\sim \sum_{k=-\infty}^{\infty} \frac {\I (-1)^{k} } {1 - k} \E^{\I k x} \\
x \cos x + \I x\sin x &\sim \sum_{k=-\infty}^{\infty} \Bigg[ \frac {\I (-1)^{k} } {1 - k} \cos kx - \frac { (-1)^{k} } {1 - k}\sin kx \Bigg].
\end{align*}
Pairing up the real and imaginary parts of this we get
$$
x \cos x \sim \sum_{k=-\infty}^{\infty}  \frac { (-1)^{k + 1} } {1 - k}\sin kx
$$
and
$$
x \sin x \sim \sum_{k=-\infty}^{\infty} \frac {(-1)^{k} } {1 - k} \cos kx.
$$
However, for the real Fourier series we want the indices to start at $k=1$ rather than $-\infty$.
Thus we have
\begin{align*}
\sum_{k=-\infty}^{\infty}  \frac { (-1)^{k + 1} } {1 - k}\sin kx
	&= \sum_{k=1}^{\infty}  \Bigg[ \frac { (-1)^{k + 1} } {1 - k}\sin kx + \frac { (-1)^{-k + 1} } {1 + k}\sin (-kx) \Bigg] \\
	&= \sum_{k=1}^{\infty}  \frac { (-1)^{k + 1} \Big( \sin kx (1 + k) - \sin kx (1 - k) \Big)} {1 - k^2} \\
	&= \sum_{k=1}^{\infty}  \frac { (-1)^{k + 1} 2 k \sin kx} {1 - k^2}
\end{align*}
However, we actually want to exclude $k= 1$ to avoid dividing by zero.
Also it is helpful to note that we have already removed $k = 0$ since $\sin 0 = 0$
Thus
$$
x \cos x \sim \sum_{k=2}^{\infty}  \frac { (-1)^{k + 1} 2 k \sin kx} {1 - k^2}.
$$
Finally for $x \sin x$ we have
\begin{align*}
\sum_{k=-\infty}^{\infty} \frac {(-1)^{k} } {1 - k} \cos kx
	&= 1 + \sum_{k=1}^{\infty} \Bigg( \frac {(-1)^{k} } {1 - k} \cos kx + \frac {(-1)^{-k} } {1 + k} \cos (- kx) \Bigg) \\
	&= 1 + \sum_{k=1}^{\infty} (-1)^{k} \cos kx \Bigg( \frac {1} {1 - k}  + \frac { 1 } {1 + k} \Bigg) \\
	&= 1 + \sum_{k=1}^{\infty} \frac {(-1)^{k} \cos kx } {1 - k^2}.
\end{align*}
And thus
$$
x \sin x \sim 1 + \sum_{k=1}^{\infty} \frac {(-1)^{k} \cos kx } {1 - k^2}.
$$
\textbf{TODO: revisit this and verify how to handle the $k=1$ case.}


\end{enumerate}

\newpage


\item Olver: 3.4.6 Write down formulas for the Fourier series of both even and odd functions on $[-\ell, \ell]$. \\

\noindent
\textit{Solution:} \\
\textbf{TODO:}

\newpage


\item Olver: 3.5.29 Let $f(x) \in L^2[a, b]$ be square integrable.
Which constant function $g(x) \equiv c$ best approximates $f$ in the least squares sense? \\

\noindent
\textit{Solution:} \\
\textbf{TODO:}

\newpage


\item Olver: 3.5.43 For each $n = 1, 2, ..., $ define the function
$$
f_n(x) = \begin{cases} 1, \quad &\frac k m \leq x \leq \frac {k + 1} m, \\ 0, \quad & \text{otherwise}\end{cases},
$$
where $n = \frac 1 2 m (m + 1) + k$ and $0 \leq k \leq m$.
Show first that $m, k$ are uniquely determined by $n$.
Then prove that, on the interval $[0, 1]$, the sequence $f_n(x)$ converges in norm to 0 but does not converge pointwise \textit{anywhere}! \\

\noindent
\textit{Solution:} \\
\textbf{TODO:}

\newpage

\item We consider the complex orthonormal basis
$$
\varphi_n = \frac 1 {\sqrt{2 \pi}} \E^{\I n x}
$$
where $n = 0, 1, -1, 2, -2, ...$.
Consider the function $f_a(x) = \E^{ax}$ with real number $a \neq 0$ and compute the Fourier coefficient
$$
\langle f_a, \varphi_n \rangle = \frac 1 {\sqrt{2 \pi}} \int_{-\pi}^{\pi} f_a(x) \E^{- \I n x} dx.
$$
Then prove the formula
$$
\sum_{n = 1}^\infty \frac {1}{a^2 + n^2} = \frac \pi {2a} \coth (\pi a)  - \frac 1 {2a^2}
$$
(Hint: Plancherel’s formula: the relation between $L^2$ norm of coefficients and $\langle f_a, f_a \rangle$.)
\\

\noindent
\textit{Solution:} \\
We begin by calculating the Fourier coefficient as requested
\begin{align*}
\langle f_a, \varphi_n \rangle &= \frac 1 {\sqrt{2 \pi}} \int_{-\pi}^{\pi} f_a(x) \E^{- \I n x} dx \\
	&= \frac 1 {\sqrt{2 \pi}} \int_{-\pi}^{\pi} \E^{ax} \E^{- \I n x} dx \\
	&= \frac 1 {\sqrt{2 \pi}} \int_{-\pi}^{\pi} \E^{(a - \I n) x} dx \\
	&= \frac 1 {\sqrt{2 \pi}} \Bigg( \left. \frac 1 {(a - \I n)} \E^{(a - \I n) x} \right|_{-\pi}^{\pi} \Bigg) \\
	&= \frac 1 {\sqrt{2 \pi}} \Bigg( \frac 1 {(a - \I n)} \E^{(a - \I n) \pi} - \frac 1 {(a - \I n)} \E^{- (a - \I n) \pi} \Bigg) \\
	&= \frac 1 {\sqrt{2 \pi}(a - \I n)} \big( \E^{(a - \I n) \pi} - \E^{- (a - \I n) \pi} \big) \\
	&= \frac 1 {\sqrt{2 \pi}(a - \I n)} \big( \E^a \E^{- \I n \pi} - \E^{- a} \E^{ \I n \pi} \big) \\
	&= \frac 1 {\sqrt{2 \pi}(a - \I n)} \Bigg( \frac {\E^{2a} \E^{- \I n \pi}}{\E^a} - \frac {\E^{ \I n \pi}}{\E^{a}}\Bigg) \\
	&= \frac 1 {\sqrt{2 \pi}(a - \I n)\E^a} \Bigg( \E^{2a} \Big( \cos{n \pi} - \cancelto{0}{\I \sin n \pi} \Big) - \Big( \cos{n \pi} + \cancelto{0}{\I \sin n \pi} \Big)\Bigg) \\
	&= \frac {(\E^{2a} - 1)\cos n \pi }{\sqrt{2 \pi}(a - \I n)\E^a} \\
	&= \frac {(\E^{2a} - 1)(-1)^n }{\sqrt{2 \pi}(a - \I n)\E^a}
\end{align*}
Next, let's prove the formula provided.
\textbf{TODO: }
\end{enumerate}

\end{document}

%%% Local Variables:
%%% mode: latex
%%% TeX-master: t
%%% End:
