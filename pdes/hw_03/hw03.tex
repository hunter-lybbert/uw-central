\documentclass[10pt]{amsart}
\usepackage[margin=1.4in]{geometry}
\usepackage{amssymb,amsmath,enumitem,url, bm}
\usepackage{graphicx,subfig}
\graphicspath{ {./images/} }
\usepackage{cancel}

\newcommand{\D}{\mathrm{d}}
\newcommand{\I}{\mathrm{i}}
\DeclareMathOperator{\E}{e}
\DeclareMathOperator{\OO}{O}
\DeclareMathOperator{\oo}{o}
\DeclareMathOperator{\erfc}{erfc}
\DeclareMathOperator{\real}{Re}
\DeclareMathOperator{\imag}{Im}
\usepackage{tikz}
\usepackage[framemethod=tikz]{mdframed}
\theoremstyle{nonumberplain}

\mdtheorem[innertopmargin=-5pt]{sol}{Solution}
%\newmdtheoremenv[innertopmargin=-5pt]{sol}{Solution}

\begin{document}
\pagestyle{empty}

\newcommand{\mline}{\vspace{.2in}\hrule\vspace{.2in}}

\noindent
\text{Hunter Lybbert} \\
\text{Student ID: 2426454} \\
\text{04-24-25} \\
\text{AMATH 503} \\
% header containing your name, student number, due date, course, and the homework number as a title.

\title{\bf {Homework 3} }


\maketitle
\noindent
Exercises come from \textit{Introduction to Partial Differential Equations by Peter J. Olver} as well as supplemented by instructor provided exercises.
\mline
\begin{enumerate}[label={\bf {\arabic*}:}]
\item Olver: 3.2.6 (a,c,e) \\

\noindent
\textit{Solution:} \\
\textbf{TODO:}

\newpage


\item Olver: 3.3.2 and 3.3.3 \\

\noindent
\textit{Solution:} \\
\textbf{TODO:}

\newpage


\item Olver: 3.2.55 \\

\noindent
\textit{Solution:} \\
\textbf{TODO:}

\newpage


\item Olver: 3.4.6 \\

\noindent
\textit{Solution:} \\
\textbf{TODO:}

\newpage


\item Olver: 3.5.29 \\

\noindent
\textit{Solution:} \\
\textbf{TODO:}

\newpage


\item Olver: 3.5.43 \\

\noindent
\textit{Solution:} \\
\textbf{TODO:}

\newpage

\item We consider the complex orthonormal basis
$$
\varphi_n = \frac 1 {\sqrt{2 \pi}} \E^{\I n x}
$$
where $n = 0, 1, -1, 2, -2, ...$.
Consider the function $f_a(x) = \E^{ax}$ with real number $a \neq 0$ and compute the Fourier coefficient
$$
\langle f_a, \varphi_n \rangle = \frac 1 {\sqrt{2 \pi}} \int_{-\pi}^{\pi} f_a(x) \E^{- \I n x} dx.
$$
Then prove the formula
$$
\sum_{n = 1}^\infty \frac {1}{a^2 + n^2} = \frac \pi {2a} \coth (\pi a)  - \frac 1 {2a^2}
$$
(Hint: Plancherel’s formula: the relation between $L^2$ norm of coefficients and $\langle f_a, f_a \rangle$.)
\\

\noindent
\textit{Solution:} \\
\textbf{TODO: }
\end{enumerate}

\end{document}

%%% Local Variables:
%%% mode: latex
%%% TeX-master: t
%%% End:
