\documentclass[10pt]{amsart}
\usepackage[margin=1.4in]{geometry}
\usepackage{amssymb,amsmath,enumitem,url, bm}
\usepackage{graphicx,subfig}
\graphicspath{ {./images/} }

\newcommand{\D}{\mathrm{d}}
\newcommand{\I}{\mathrm{i}}
\DeclareMathOperator{\E}{e}
\DeclareMathOperator{\OO}{O}
\DeclareMathOperator{\oo}{o}
\DeclareMathOperator{\erfc}{erfc}
\DeclareMathOperator{\real}{Re}
\DeclareMathOperator{\imag}{Im}
\usepackage{tikz}
\usepackage[framemethod=tikz]{mdframed}
\theoremstyle{nonumberplain}

\mdtheorem[innertopmargin=-5pt]{sol}{Solution}
%\newmdtheoremenv[innertopmargin=-5pt]{sol}{Solution}

\begin{document}
\pagestyle{empty}

\newcommand{\mline}{\vspace{.2in}\hrule\vspace{.2in}}

\noindent
\text{Hunter Lybbert} \\
\text{Student ID: 2426454} \\
\text{04-09-25} \\
\text{AMATH 503} \\
% header containing your name, student number, due date, course, and the homework number as a title.

\title{\bf {Homework 1} }


\maketitle
\noindent
Exercises come from \textit{Introduction to Partial Differential Equations by Peter J. Olver} as well as supplemented by instructor provided exercises.
\mline
\begin{enumerate}[label={\bf {\arabic*}:}]
\item Olver 1.1 \\
\textit{Solution:}
\begin{enumerate}
\item $\frac {du}{dx} + xu = 1:$ \quad Ordinary equilibrium differential equation of the first order.
\item $\frac {\partial u}{\partial t} + u \frac{\partial u }{\partial x} = x:$ \quad Partial dynamic differential equation of the first order.
\item $u_{tt} = 9u_{xx}:$ \quad Partial dynamic differential equation of the second order.
\item $\frac {\partial u}{\partial t} = \frac{\partial^2 u }{\partial x^2} + \frac{\partial u }{\partial x}:$ \quad Partial dynamic differential equation of the second order.
\item $- \frac{\partial^2 u }{\partial x^2} - \frac{\partial^2 u }{\partial y^2} = x^2 + y^2:$ \quad Partial equilibrium differential equation of the second order.
\item $\frac{\partial^2 u }{\partial t^2} + 3u = \sin t:$ \quad Ordinary dynamic differential equation of the second order.
\item $u_{xx} + u_{yy} + u_{zz} + (x^2 + y^2 + z^2)u = 0:$ \quad Partial equilibrium differential equation of the second order.
\item $u_{xx} = x + u^2:$ \quad Ordinary equilibrium differential equation of the second order.
\item $\frac{\partial u }{\partial t} + \frac{\partial^3 u }{\partial x^3} + u \frac{\partial u }{\partial x} = 0:$ \quad Partial dynamic differential equation of the third order.
\item $\frac{\partial^2 u }{\partial x^2} + \frac{\partial^2 u }{\partial y \partial z} = u:$ \quad Partial equilibrium differential equation of the second order.
\item $u_{tt} = u_{xxxx} + 2u_{xxyy} + u_{yyyy}:$ \quad Partial dynamic differential equation of the fourth order.
\end{enumerate}
\qed \\

\item Olver 1.17 \\
\textit{Solution:}
\begin{enumerate}
\item $u_t = x^2u_{xx} + 2xu_x:$ \quad homogeneous linear
\item $-u_{xx} = u_{yy} = \sin u:$ \quad nonlinear
\item $u_{xx} + 2yu_{yy} = 3:$ \quad inhomogeneous linear
\item $u_t + uu_x = 3u:$ \quad nonlinear
\item $\E^yu_x = \E^xu_y:$ \quad homogeneous linear
\item $u_t = 5u_{xxx} + x^2u + x:$ \quad inhomogeneous linear
\end{enumerate}
\qed \\
\newpage

\item Olver 1.22 \\
\begin{enumerate}
\item Prove that the Laplacian $\Delta = \partial_x^2 + \partial_y^2$ defines a linear differential operator. \\

\textit{Solution:} We need to show that for some appropriate functions $u$, $v$ and two scalars $a, b \in \mathbb R$
$$
\Delta [au + bv] = a\Delta [u] + b \Delta [v].
$$
We will do this directly,
\begin{align*}
\Delta [au + bv] = (\partial_x^2 + \partial_y^2)(au + bv) &= (\partial_x^2 + \partial_y^2)au + (\partial_x^2 + \partial_y^2)bv \\
	&= \partial_x^2au + \partial_y^2au + \partial_x^2bv + \partial_y^2bv \\
	&= a \partial_x^2 u + a \partial_y^2 u + b \partial_x^2 v + b \partial_y^2 v \\
	&= a u_{xx} + a u_{yy} + b v_{xx} + b v_{yy} \\
	&= a (u_{xx} + u_{yy}) + b (v_{xx} + v_{yy}) \\
	&= a (\partial_x^2 u + \partial_y^2 u) + b (\partial_x^2 v + \partial_y^2 v) \\
	&= a (\partial_x^2 + \partial_y^2) u + b (\partial_x^2 + \partial_y^2) v \\
	&= a \Delta [u] + b \Delta [v].
\end{align*}
\qed \\

\item Write out the Laplace equation $\Delta[u] = 0$ and the Poisson equation $-\Delta[u] = f$. \\

\textit{Solution:} The Laplace equation is
$$
\Delta [u] = (\partial_x^2 + \partial_y^2)u = u_{xx} + u_{yy} = 0
$$
and the Poisson equation is 
$$
- \Delta [u] = - (\partial_x^2 + \partial_y^2)u = - u_{xx} - u_{yy} = f.
$$
\qed \\

\end{enumerate}
\newpage

\item We derive the advection-diffusion equation from the microscopic view.
Define $u(x, t)$ as the density of the particles at location $x$ and time $t$.
Define the probability of jumping from the left as $p(x - \Delta x \rightarrow x, t) \approx \frac 1 2 + \Delta x$
when $\Delta x$ is small, and the probability of jumping from the right as $q(x + \Delta x \rightarrow x, t) \approx \frac 1 2 - \Delta x $) with small $\Delta x$.
Assume $D := \lim_{\Delta x, \Delta t \rightarrow 0} \frac {\big(\Delta x\big)^2}{\Delta t}$.
Establish the equation of $u(x, t)$ in the continuum limit. \\
\textit{Solution:} \\
We begin by taylor expanding $u(x, t + \Delta t), u(x - \Delta x, t), \text{ and } u(x + \Delta x , t)$
\begin{align*}
u(x, t + \Delta t) &= u(x, t) + u_t \Delta t + \mathcal O\big((\Delta t)^2\big) \\
u(x - \Delta x, t) &= u(x, t) - u_x \Delta x + \frac 1 2 u_{xx} (\Delta x)^2 + \mathcal O\big((\Delta x)^3\big) \\
u(x + \Delta x , t) &= u(x, t) + u_x \Delta x + \frac 1 2 u_{xx} (\Delta x)^2 + \mathcal O\big((\Delta x)^3\big).
\end{align*}
Additionally, we have the following relationship for the evolution of the system in one time step
\begin{align*}
u(x, t + \Delta t) &= p(x - \Delta x \rightarrow x, t) u(x + \Delta x, t) + q(x + \Delta x \rightarrow x, t) u (x - \Delta x, t) \\
u(x, t + \Delta t) &\approx \left( \frac 1 2 + \Delta x \right) u(x + \Delta x, t) + \left( \frac 1 2 - \Delta x \right) u (x - \Delta x, t).
\end{align*}
Combining this with the taylor expansions from earlier we have
\begin{align*}
u(x, t) + u_t \Delta t &\approx \left( \frac 1 2 + \Delta x \right) \left( u(x, t) + u_x \Delta x + \frac 1 2 u_{xx} (\Delta x)^2 + \mathcal O\big((\Delta x)^3\big)\right) \\
	& \quad + \left( \frac 1 2 - \Delta x \right) \left( u(x, t) - u_x \Delta x + \frac 1 2 u_{xx} (\Delta x)^2 + \mathcal O\big((\Delta x)^3\big) \right) \\
u(x, t) + u_t \Delta t &\approx \frac 1 2 \left( u(x, t) + u_x \Delta x + \frac 1 2 u_{xx} (\Delta x)^2 + \mathcal O\big((\Delta x)^3\big)\right)  + \Delta x \left( u(x, t) + u_x \Delta x + \frac 1 2 u_{xx} (\Delta x)^2 + \mathcal O\big((\Delta x)^3\big)\right) \\
	& + \frac 1 2 \left( u(x, t) - u_x \Delta x + \frac 1 2 u_{xx} (\Delta x)^2 + \mathcal O\big((\Delta x)^3\big) \right) - \Delta x \left( u(x, t) - u_x \Delta x + \frac 1 2 u_{xx} (\Delta x)^2 + \mathcal O\big((\Delta x)^3\big) \right) \\
u_t \Delta t &\approx \frac 1 2 u_{xx} (\Delta x)^2 + \Delta x \left( u(x, t) + u_x \Delta x + \frac 1 2 u_{xx} (\Delta x)^2 + \mathcal O\big((\Delta x)^3\big)\right) \\
	& - \Delta x \left( u(x, t) - u_x \Delta x + \frac 1 2 u_{xx} (\Delta x)^2 + \mathcal O\big((\Delta x)^3\big) \right) \\
u_t \Delta t &\approx \frac 1 2 u_{xx} (\Delta x)^2 + 2 u_x (\Delta x)^2 \\
u_t &\approx \left( \frac 1 2 u_{xx} + 2 u_x \right) \frac {(\Delta x)^2}{\Delta t} \\
u_t &= D \left( \frac 1 2 u_{xx} + 2 u_x \right).
\end{align*}
This is the differential equation for the equation $u(x, t)$ in the continuum limit. \\
\qed \\

\newpage

\item 
\begin{enumerate}
\item Consider the following boundary value problem (BVP). \\
$$
\begin{cases}
X^{\prime\prime}(x) + \lambda X = 0, \quad x \in (0, L) \\
X(0)  = X(L) = 0, \\
\end{cases}
$$
where $L > 0$ is a constant.
Solve the eigenpair:
$$
(X_k, \lambda_k) = \left\{ \sin \left( \frac {k \pi x}{L} \right), \left(\frac {k \pi}{L} \right)^2 \right\}_{k=1}^\infty
$$
\textit{Solution:} \\
I begin by rewriting the second order ODE as a 2D first order ODE.
Let $X_1 = X$ and $X_2 = X^\prime$ then we have
$$
\begin{bmatrix}
X_1 \\
X_2
\end{bmatrix}^\prime =
\begin{bmatrix}
X_2 \\
-\lambda X_1
\end{bmatrix} =
\begin{bmatrix}
0 &1 \\
-\lambda &0
\end{bmatrix}
\begin{bmatrix}
X_1 \\
X_2
\end{bmatrix}.
$$
Using this system we can solve for the eigenvalues of this system, that is we want to solve for $\gamma$ (since $\lambda$ is already in use in this function we choose a stand in variable) in the following equation
$$
(-\gamma)^2 - (-\lambda \cdot 1) = \gamma^2 + \lambda = 0
$$
Therefore the eigenvalues are $\gamma = \pm \I \sqrt{\lambda}$.
Since the eigenvalues are both imaginary we know that the solution will be of the form
$$
X(x) = C_1 \cos(\sqrt \lambda x) + C_2 \sin(\sqrt \lambda x).
$$
Now using the boundary values we can determine
\begin{align*}
X(0) &= C_1 \cos(\sqrt \lambda 0) + C_2 \sin(\sqrt \lambda 0) \\
0 &= C_1.
\end{align*}
Furthermore, we need
\begin{align*}
X(L) &= C_2 \sin(\sqrt \lambda L) \\
0 &= C_2 \sin(\sqrt \lambda L) \\
0 &= \sin(\sqrt \lambda L)
\end{align*}
assuming $C_2 \neq 0$ to avoid arriving at the uninteresting trivial solution.
We know $\sin(x) = 0$ where $x = \pi k$ for $k \in \mathbb Z$.
Thus we need $\sqrt \lambda L = \pi k$ which gives us $\lambda = \left( \frac {\pi k } L \right)^2$.
Finally, our general solution currently is the following
$$
X_k(x) = \sin \left( \frac{\pi k x} L \right)
$$
where $\lambda_k = \left( \frac {\pi k } L \right)^2$. \\
\textbf{TODO: Do the other cases.}
\qed \\
\newpage

\item Consider the following boundary value problem (BVP). \\
$$
\begin{cases}
X^{\prime\prime}(x) + \lambda X = 0, \quad x \in (0, L) \\
X^\prime(0)  = X^\prime(L) = 0, \\
\end{cases}
$$
where $L > 0$ is a constant.
Solve the eigenpair:
$$
(X_k, \lambda_k) = \left\{ \cos \left( \frac {k \pi x}{L} \right), \left(\frac {k \pi}{L} \right)^2 \right\}_{k=0}^\infty
$$
\textit{Solution:} \\
We reuse much of the work for $X(x)$ from part (a), however the final steps using the boundary values will vary slightly.
Beginning from the solution following the form
$$
X(x) = C_1 \cos(\sqrt \lambda x) + C_2 \sin(\sqrt \lambda x),
$$
we now apply boundary conditions.
We first need this time to calculate
$$
X^\prime(x) = - C_1 \sqrt \lambda \sin(\sqrt \lambda x) + C_2 \sqrt \lambda \cos(\sqrt \lambda x)
$$
Plugging in our boundary conditions we have
\begin{align*}
X^\prime(0) &= - C_1 \sqrt \lambda \sin(\sqrt \lambda 0) + C_2 \sqrt \lambda \cos(\sqrt \lambda 0) \\
0 &= C_2.
\end{align*}
Once more, we have
\begin{align*}
X^\prime(L) &= - C_1 \sqrt \lambda \sin(\sqrt \lambda L) \\
0 &= - C_1 \sqrt \lambda \sin(\sqrt \lambda L)
\end{align*}
which only holds when $\sqrt \lambda = \frac {\pi k} L $ for $k \in \mathbb Z$.
This in total gives the solution
$$
X(x) = C_1 \cos \left(\frac {\pi k x} L \right)
$$
where $\lambda = \left ( \frac {\pi k } L \right )^2$.
\qed \\

\end{enumerate}

\newpage

\item Consider the following IBVP in a rectangle: \\
\begin{align*}
\begin{cases}
u_t = \Delta u, &(x, y) \in (0, L_1) \times (0, L_2), t > 0 \\
\partial_{\bm n} u(x, y, t) = 0, &(x, y) \in \partial \big( (0, L_1) \times (0, L_2) \big), t > 0 \\
u(x, y, 0) = u_0(x, y) \geq 0, \not \equiv 0 &(x, y) \in (0, L_1) \times (0, L_2)
\end{cases}
\end{align*}
where $\bm n$ denotes the unit outer normal derivative and $L_1, L_2 > 0$ are given constants.
Solve to get the general solution.
Recall that $\Delta = \partial_{xx} + \partial_{yy}$. \\
\textit{Solution:} \\
We have
$$
u_t = u_{xx} + u_{yy},
$$
assuming we have the ability to use separation of variables we let $u(x, y, t)$ take the form
$u(x, y, t) = X(x)Y(y)T(t)$.
Then plugging this in we have:
\begin{align*}
X(x)Y(y)T^\prime(t) &= X^{\prime\prime}(x)Y(y)T(t) + X(x)Y^{\prime\prime}(y)T(t) \\
\frac {T^\prime(t)}{T(t)} &= \frac{X^{\prime\prime}(x)}{X(x)} + \frac{Y^{\prime\prime}(y)}{Y(y)} = -\lambda.
\end{align*}
Where the $-\lambda$ came from following the example in class. \\
\textbf{TODO: use 5 b) to skip the cases}

\end{enumerate}

\end{document}

%%% Local Variables:
%%% mode: latex
%%% TeX-master: t
%%% End:
