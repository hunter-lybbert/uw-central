\documentclass[10pt]{amsart}
\usepackage[margin=1.4in]{geometry}
\usepackage{amssymb,amsmath,enumitem,url}
\usepackage{graphicx,subfig}
\graphicspath{ {./images/} }

\newcommand{\D}{\mathrm{d}}
\newcommand{\I}{\mathrm{i}}
\DeclareMathOperator{\E}{e}
\DeclareMathOperator{\OO}{O}
\DeclareMathOperator{\oo}{o}
\DeclareMathOperator{\erfc}{erfc}
\DeclareMathOperator{\real}{Re}
\DeclareMathOperator{\imag}{Im}
\usepackage{tikz}
\usepackage[framemethod=tikz]{mdframed}
\theoremstyle{nonumberplain}

\mdtheorem[innertopmargin=-5pt]{sol}{Solution}
%\newmdtheoremenv[innertopmargin=-5pt]{sol}{Solution}

\begin{document}
\pagestyle{empty}

\newcommand{\mline}{\vspace{.2in}\hrule\vspace{.2in}}

\noindent
\text{Hunter Lybbert} \\
\text{Student ID: 2426454} \\
\text{04-03-25} \\
\text{AMATH 503} \\
% header containing your name, student number, due date, course, and the homework number as a title.

\title{\bf {Homework 1} }


\maketitle
\noindent
Exercises come from \textit{Introduction to Partial Differential Equations by Peter J. Olver} as well as supplemented by instructor provided exercises.
\mline
\begin{enumerate}[label={\bf {\arabic*}:}]
\item Olver 1.1 \\
\textit{Solution:} \\
\textbf{TODO} \\

\item Olver 1.17 \\
\textit{Solution:} \\
\textbf{TODO} \\

\item Olver 1.22 \\
\textit{Solution:} \\
\textbf{TODO} \\

\item We derive the advection-diffusion equation from the microscopic view.
Define $u(x, t)$ as the density of the particles at location $x$ and time $t$.
Define the probability of jumping from the left as $p(x - \Delta x \rightarrow x, t) \approx \frac 1 2 + \Delta x$
when $\Delta x$ is small, and the probability of jumping from the right as $q(x + \Delta x \rightarrow x, t) \approx \frac 1 2 - \Delta x $) with small $\Delta x$.
Assume $D := \lim_{\Delta x, \Delta t \rightarrow 0} \frac {\big(\Delta x\big)^2}{\Delta t}$.
Establish the equation of $u(x, t)$ in the continuum limit.
\\
\textit{Solution:} \\
\textbf{TODO} \\

\item Consider the following boundary value problem (BVP). \\
$$
\begin{cases}
X^\prime\prime(x) + \lambda X = 0 \\
X(0)  = X(L) = 0 \\
\end{cases}
$$
where $L > 0$ is a constant.
Solve the eigenpair:
$$
(X_k, Y_k) = \left\{ \sin \left( \frac {k \pi x}{L}, \frac {k \pi}{L} \right)^2 \right\}_{k=0}^\infty
$$
\textit{Solution:} \\
\textbf{TODO} \\

\item Finish writing up information for question in class tomorrow. \\
\textit{Solution:} \\
\textbf{TODO} \\

\end{enumerate}

\end{document}

%%% Local Variables:
%%% mode: latex
%%% TeX-master: t
%%% End:
