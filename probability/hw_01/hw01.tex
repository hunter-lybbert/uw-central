\documentclass[11pt]{article}
\include{amsmath}
\usepackage{dsfont}													% gives you \mathds{} font

\title{AMATH 561 Autumn 2024 \\ Problem Set 1}
%\author{}
\date{Due: Mon 10/7 at 10am}

\begin{document}
\maketitle

{\it Note: Submit electronically to Canvas. }
\\

\noindent {\bf 1.} Describe the probability space for the following experiments: a) a biased coin is tossed three times; b)  two balls are drawn without replacement from an urn which originally contained two blue and two red balls.
\\

\noindent {\bf 2.} (No translation-invariant random integer). Show that there is no probability measure $P$ on the integers $\mathds{Z}$ with the discrete
$\sigma$-algebra $2^{\mathds{Z}}$ with the translation-invariance property $P(E + n) = P(E)$ for every event $E \in 2^{\mathds{Z}}$ and every integer $n$. $E+n$ is obtained by adding $n$ to every element of $E$.
\\

\noindent {\bf 3.}  (No translation-invariant random real). Show that there is no probability measure $P$ on the reals $\mathds{R}$ with the Borel
$\sigma$-algebra $\mathcal{B}(\mathds{R})$ with the translation-invariance property $P(E + x) = P(E)$ for every event $E \in \mathcal{B}(\mathds{R})$ and every real $x$. Borel $\sigma$-algebra $\mathcal{B}(\mathds{R})$ is the $\sigma$-algebra generated by intervals $(a,b] \subset \mathds{R}$.
\\

\noindent {\bf 4.} Let $\Omega=\mathds{R}$, $\mathcal{F}=$ all subsets of $\mathds{R}$ so that $A$ or $A^c$ is countable. Let $P(A)=0$ in the first case and $P(A)=1$ in the second. Show that $(\Omega, \mathcal{F}, P)$ is a probability space.



\end{document}  
