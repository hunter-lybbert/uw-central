\documentclass[10pt]{amsart}
\include{amsmath}
\usepackage{dsfont}													% gives you \mathds{} font
\usepackage{amssymb}

\newcommand{\D}{\mathrm{d}}

\begin{document}

\noindent
\text{Hunter Lybbert} \\
\text{Student ID: 2426454} \\
\text{10-07-24} \\
\text{AMATH 561}
\title{Problem Set 3}
\maketitle

\noindent {\bf 1.} Give an example of a probability space $(\Omega, \mathcal{F},P)$, a random variable $X$ and a function $f$ such that $\sigma(f(X))$ is strictly smaller than $\sigma(X)$ but $\sigma(f(X)) \neq \{\emptyset,\Omega\}$. Give a function $g$ such that $\sigma(g(X))=\{\emptyset,\Omega\}$. Hint: Look at finite sample spaces with a small number of elements. \\
\textit{Solution:} \\
Let our probability space be two independent coin tosses, such that $\Omega = \big\{HH, TT, HT, TH\big\}$.
Define a random variable $X$ such that $X(\omega)$ be the number of heads in the outcome $\omega$ with $\omega \in \Omega$.
Therefore
\begin{align*}
X(HH) &= 2, \\
X(TT) &= 0, \\
X(HT) &= 1, \:\: \text{and}\\
X(TH) &= 1.
\end{align*}
Now $\sigma(X)$ can be written as
$$
\sigma(X) = \bigg\{ \{HH\}, \{TH, HT\}, \{TT\}, \{TT, HH\}, \{HH, HT, TH\}, \{TT, HT, TH\}, \Omega, \emptyset \bigg\}
$$
\\
\textbf{Part one} \\
Random variable $X$ and $f$ such that $\sigma(f(X)) \subsetneq \sigma(X)$ and $\sigma(f(X))$ is not the trivial $\sigma$-algebra. \\

\noindent
Define $f(x)$ as follows
$$
f(x)= \begin{cases}
 	0, & x \leq 0 \\
	1, & x > 0.
\end{cases}
$$
Let's look at the possible pre-images of $f(X)$ with respect to a few cases of Borel sets.
For convenience, I will define $\hat X = f(X)$.
Now let's look at some cases for the pre-image\\
\textit{Case 1:} $0 \in B$ but $1 \not\in B$
\begin{align*}
\hat X^{-1} (B) = \left\{ \omega : \hat X(\omega) \in B \right\} = \left\{ \omega : X(\omega) \in (-\infty, 0] \ \right\} = \{ TT \}
\end{align*}
\textit{Case 2:} $0 \not \in B$ but $1 \in B$
\begin{align*}
\hat X^{-1} (B) = \left\{ \omega : \hat X(\omega) \in B \right\} = \left\{ \omega : X(\omega) \in (0, \infty) \ \right\} = \{ TH, HT, HH \}
\end{align*}
\textit{Case 3:} $0 \in B$ and $1 \in B$
\begin{align*}
\hat X^{-1} (B) = \left\{ \omega : \hat X(\omega) \in B \right\} = \left\{ \omega : X(\omega) \in (-\infty, \infty) \ \right\} = \Omega
\end{align*}
\textit{Case 4:} $0 \not\in B$ and $1 \not\in B$
\begin{align*}
\hat X^{-1} (B) = \left\{ \omega : \hat X(\omega) \in B \right\} = \emptyset.
\end{align*}
Therefore,
\begin{align*}
\sigma(f(X)) = \bigg\{
	\{ TT \},
	\{ TH, HT, HH \},
	\Omega,
	\emptyset
\bigg\} \neq \left\{\emptyset, \Omega \right\}
\end{align*}
And thus we have $\sigma(f(X)) \subsetneq \sigma(X)$. \\
\textbf{Part two}
Now also give a function $g$ such that $\sigma(g(X))$ is the trivial $\sigma$-algebra, $\left\{ \emptyset, \Omega\right\}$.

\noindent
Define $g(x)$ to be a constant $c \in \mathbb R$ such that $g(x) = c$ for all $x \in \mathbb R$.
Once again, for convenience we define $\tilde X = g(X)$.
Let's go through a few cases of what the pre-image may be for any Borel set\\
\textit{Case 1:} $c \in B$
\begin{align*}
\tilde X^{-1} (B) = \left\{ \omega : \tilde X(\omega) \in B \right\} = \left\{ \omega : X(\omega) \in (-\infty, \infty) \ \right\} = \Omega
\end{align*}
\textit{Case 2:} $c \not \in B$
\begin{align*}
\tilde X^{-1} (B) = \left\{ \omega : \tilde X(\omega) \in B \right\} = \emptyset.
\end{align*}
Therefore,
\begin{align*}
\sigma(g(X)) = \left\{ \Omega, \emptyset \right\}.
\end{align*}
\qed
\\

\noindent {\bf 2.} Give an example of events $A$, $B$, and $C$, each of probability strictly between 0 and 1, such that
$P(A\cap B)=P(A)P(B), P(A\cap C)=P(A)P(C)$, and $P(A\cap B\cap C)=P(A)P(B)P(C)$ but $P(B\cap C)\neq P(B)P(C)$. Are $A$, $B$ and $C$ independent? Hint: You can let $\Omega$ be a set of eight equally likely points. \\
\textit{Solution:} \\
Let $\Omega = \{1, 2, 3, 4, 5, 6, 7, 8\}$.
Define events $A$, $B$, and $C$ as follows
\begin{align*}
A = \{1, 2, 3, 4\} \\
B = \{1, 2, 5, 7\} \\
C = \{1, 3, 6, 8\}.
\end{align*}
Then we have 
$$P(A \cap B) = P(\{ 1, 2 \}) = \frac 1 4$$
and
$$P(A)P(B) = P(\{1, 2, 3, 4\})P(\{1, 2, 5, 7\}) =  \frac 1 2 \cdot \frac 1 2 = \frac 1 4.$$
Additionally, we have 
$$P(A \cap C) = P(\{ 1, 3 \}) = \frac 1 4$$
and
$$P(A)P(C) = P(\{1, 2, 3, 4\})P(\{1, 3, 6, 8\}) =  \frac 1 2 \cdot \frac 1 2 = \frac 1 4.$$
Finally, we have 
$$P(A \cap B \cap C) = P(\{ 1\}) = \frac 1 8$$
and
$$P(A)P(B)P(C) = P(\{1, 2, 3, 4\})P(\{1, 2, 5, 7\})P(\{1, 3, 6, 8\}) =  \frac 1 2 \cdot \frac 1 2 \cdot \frac 1 2 = \frac 1 8.$$
Notice we also get
$$P(B \cap C) = P(\{ 1 \}) = \frac 1 8$$
which is not equal to
$$P(B)P(C) = P(\{1, 2, 5, 7\})P(\{1, 3, 6, 8\}) =  \frac 1 2 \cdot \frac 1 2 = \frac 1 4.$$
In class we said two events $E$ and $E^\prime$ are independent if $P(E\cap E^\prime) = P(E)P(E^\prime)$.
Therefore we have shown that $A$ and $B$ are independent, $A$ and $C$ are independent but $B$ and $C$ are not independent.
\qed
\\

\noindent {\bf 3.} Let $(\Omega, \mathcal{F},P)$ be a probability space such that $\Omega$ is countably infinite, and $\mathcal{F}=2^{\Omega}$. Show that it is impossible for there to exist a countable collection of events $A_1, A_2,... \in \mathcal{F}$ which are independent, such that $P(A_i)=1/2$ for each $i$. Hint: First show that for each $\omega \in \Omega$ and each $n\in \mathds{N}$, we have $P({\omega})\leq 1/2^n$. Then derive a contradiction. \\
\textit{Solution:} \\
Literally just use the hint...
\\

\noindent {\bf 4.}  (a) Let $X \geq 0$ and $Y \geq 0$  be independent random variables with distribution functions $F$ and $G$. Find the distribution function of $XY$. \\
\textit{Solution:} \\
These are not explicitly dealing with discrete or continuous.
Definitely review lecture notes.
Since these are independent try using the formulae from the lecture on 10-16-24.
\\

(b) If $X \geq 0$ and $Y \geq 0$ are independent continuous random variables with density functions $f$ and $g$, find the density function of $XY$. \\
\textit{Solution:} \\
Notice these are continuous and you're dealing with densities.
\\

(c) If $X$ and $Y$ are independent exponentially distributed random variables with parameter $\lambda$, find the density function of $XY$.\\
\textit{Solution:} \\
TBD
\\
\end{document}  
