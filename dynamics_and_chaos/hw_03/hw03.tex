\documentclass[10pt]{amsart}
\usepackage[margin=1.4in]{geometry}
\usepackage{amssymb,amsmath,enumitem,url}
\usepackage{graphicx,subfig}
\usepackage{bm}
\graphicspath{ {./images/} }

\newcommand{\D}{\mathrm{d}}
\newcommand{\I}{\mathrm{i}}
\DeclareMathOperator{\E}{e}
\DeclareMathOperator{\OO}{O}
\DeclareMathOperator{\oo}{o}
\DeclareMathOperator{\erfc}{erfc}
\DeclareMathOperator{\real}{Re}
\DeclareMathOperator{\imag}{Im}
\usepackage{tikz}
\usepackage[framemethod=tikz]{mdframed}
\theoremstyle{nonumberplain}

\mdtheorem[innertopmargin=-5pt]{sol}{Solution}
%\newmdtheoremenv[innertopmargin=-5pt]{sol}{Solution}

\begin{document}
\pagestyle{empty}

\newcommand{\mline}{\vspace{.2in}\hrule\vspace{.2in}}

\noindent
\text{Hunter Lybbert} \\
\text{Student ID: 2426454} \\
\text{01-30-25} \\
\text{AMATH 502} \\
% header containing your name, student number, due date, course, and the homework number as a title.

\title{\bf {Homework 3} }


\maketitle
\noindent
Exercises come from a pdf provided by instructor.
\mline
\begin{enumerate}[label={\bf {\arabic*}:}]
\item For each of the following vector fields, find and classify all the fixed points, and
sketch the phase portrait on the circle. \\
\begin{enumerate}

\item $\dot \theta = 1 + 2 \cos \theta$ \\
\textit{Solution:} \\
We need to find where $\dot \theta = 0$.
Therefore we need 
\begin{align*}
1 + 2 \cos \theta &= 0 \\
\cos \theta &= -\frac 1 2.
\end{align*}
Which if we are assuming $\theta \in [0, 2 \pi)$ then we have fixed points at $\theta^* = \frac {2 \pi} 3, \frac {4 \pi} 3$.
See Figure \ref{fig:f1} for the phase portrait on the circle. \textbf{TODO: Add picture} \\
\qed \\

\item $\dot \theta = \sin \theta + \cos \theta$ \\
\textit{Solution:} \\
Once again we need to find where $\dot \theta = 0$.
Therefore we are looking for $\theta$ s.t.
\begin{align*}
\sin \theta + \cos \theta &= 0 \\
\cos \theta &= -\sin \theta.
\end{align*}
Assuming $\theta \in [0, 2 \pi)$ then this only holds for the following values of $\theta$ and thus we have fixed points at $\theta^* = \frac {3 \pi} 4, \frac {7 \pi} 4$.
See Figure \ref{fig:f2} for the phase portrait on the circle. \textbf{TODO: Add picture} \\
\qed \\

\item $\dot \theta = \sin 4 \theta$ \\
\textit{Solution:} \\
Once again we need to find where $\dot \theta = 0$.
Therefore we are looking for $\theta$ s.t.
\begin{align*}
\sin 4 \theta &= 0.
\end{align*}
Assuming $\theta \in [0, 2 \pi)$ then this only holds for the following values of $\theta$ and thus we have fixed points at $\theta^* = \frac {k \pi} 8$ for $k \in \{0, 1, 2, ..., 7\}$.
See Figure \ref{fig:f3} for the phase portrait on the circle. \textbf{TODO: Add picture} \\
\qed \\

\end{enumerate}

\newpage

\item There is a lot of context and build up to this problem in the assignment pdf which I will not transcribe here.
The important details will be included in their respective parts. \\

\begin{enumerate}

\item Graph the following function on the domain $-\pi \leq \phi < \pi$
\begin{align*}
f(\phi) = \begin{cases}
2\phi, & |\phi| \leq \frac \pi 2 \\
2 {\rm sgn}(\theta)\pi - 2\phi & |\phi| > \frac \pi 2.
\end{cases}
\end{align*}

\textit{Solution:} \\
See Figure \ref{fig:f4} for the graphs drawn by hand of the function $f(\phi)$.
\textbf{TODO: Add picture} \\
\qed \\

\item rite the dynamical system for the phase difference $\phi = \theta_s - \theta_f$ in terms of the dimensionless time $\tau = At$ and the dimensionless parameter $\mu = \frac {\Omega - \omega} A$. \\
\textit{Solution:} \\
We are also going to want to use the substitution $d \tau = A d t$.
Additionally, we are starting from the dynamical systems for $\theta_f$ and $\theta_s$ given by
$$\dot \theta_f = \omega + A f(\theta_s - \theta_f) \quad {\rm and }\quad \dot \theta_s = \Omega.$$
Let's start by manipulating these a bit
\begin{align}
\dot \theta_f &= \omega + A f(\theta_s - \theta_f) \nonumber \\
\frac {d \theta_f}{dt} &= \omega + A f(\theta_s - \theta_f) \nonumber \\
\frac {d \theta_f}{\frac 1 A d \tau} &= \omega + A f(\theta_s - \theta_f) \nonumber \\
A \frac {d \theta_f}{d \tau} &= \omega + A f(\theta_s - \theta_f) \nonumber \\
\frac {d \theta_f}{d \tau} &= \frac \omega A + f(\theta_s - \theta_f).
\label{eq:eq1}
\end{align}
Let's pause here, and manipulate this thing
\begin{align*}
\mu &= \frac {\Omega - \omega} A \\
\mu &= \frac \Omega A - \frac \omega A \\
\mu + \frac \omega A &= \frac \Omega A \\
\frac \omega A &= \frac \Omega A - \mu
\end{align*}
Now we can combine this result with equation \eqref{eq:eq1} as follows
\begin{align*}
\frac {d \theta_f}{d \tau} &= \frac \Omega A - \mu + f(\theta_s - \theta_f) \\
\frac {d \theta_f}{d \tau} &= \frac {\dot \theta_s} A - \mu + f(\theta_s - \theta_f) \\
\frac {d \theta_f}{d \tau} - \frac {d \theta_s} {A d t} &= - \mu + f(\theta_s - \theta_f) \\
\frac {d \theta_f}{d \tau} - \frac {d \theta_s} {d \tau} &= - \mu + f(\theta_s - \theta_f) \\
\frac {d \theta_f}{d \tau} - \frac {d \theta_s} {d \tau} &= - \mu + f(\theta_s - \theta_f) \\
\frac {d \theta_s} {d \tau} - \frac {d \theta_f}{d \tau} &= \mu - f(\theta_s - \theta_f) \\
\frac {d \theta_s - d \theta_f} {d \tau} &= \mu - f(\theta_s - \theta_f) \\
\frac {d \phi} {d \tau} &= \mu - f(\phi) \\
\end{align*}
We now have a the desired dynamical system. \\
\qed \\

\item Find the values of $\mu$ for which the firefly will be phase-locked to the stimulus. \\
\textit{Solution:} \\
\begin{align*}
\frac {d \phi} {d \tau} &= \mu - f(\phi) \\
0 &= \mu - f(\phi) \\
f(\phi) &= \mu
\end{align*}
Therefore whenever $\mu \in [-\pi, \pi]$ we have a fixed point. \\
\qed \\

\item Using the definition of $\mu$ and your answer from part (c), find the range of
frequencies of the stimulus $\Omega$ for which the firefly will be phase-locked to the
stimulus. \\
\textit{Solution:} \\
\begin{align*}
-\pi \leq \mu \leq \pi \\
-\pi \leq \frac {\Omega - \omega} A \leq \pi \\
-A\pi + \omega \leq \Omega \leq A\pi + \omega
\end{align*}
Therefore, $\Omega \in [-A\pi + \omega, A\pi + \omega]$. \\
\qed \\

\item What kind of bifurcation occurs at $\mu = \pm \pi$? Does this look like the usual
form for this type of bifurcation? If not, why not? \\
\textit{Solution:} \\
The bifurcation that occurs is a saddle node bifurcation, but it is not in the usual form.
For one, we typically see it as more of a sideways parabolic shape which we don't have here.
Secondly, we typically see the fixed points arise and continue to exist along their respective branches going off to infinity one way or the other.
However, this time we have the case where the two fixed points or branches arise and then eventually return to one another and finally disappearing again.
See Figure \ref{fig:f8} for further details on this odd saddle node bifurcation example.
\textbf{TODO: add picture of bifurcation} \\

\item Assuming the firefly is phase-locked to the stimulus, find a formula for the
phase difference $\phi*$ (i.e. the stable fixed point). \\
\textit{Solution:} \\
\textbf{TODO} \\

\end{enumerate}

\newpage

\item Plot the phase portrait and classify the fixed point of the following linear systems. 
put the system in matrix form. \\

\begin{enumerate}

\item $\dot x = y, \quad \dot y = -2x - 3y$ \\
\textit{Solution:} \\
We can put this in the form of a 2 dimensional first order system as follows
\begin{align*}
\begin{bmatrix}
x \\ y
\end{bmatrix}^\prime
&= \begin{bmatrix}
0 & 1 \\
-2 & -3
\end{bmatrix} \begin{bmatrix}
x \\ y
\end{bmatrix}.
\end{align*}
Let's begin by finding the fixed points.
For a fixed point we need $y = 0$ and $-2x -3y = 0$.
Therefore we need $-2x = 0$ which implies the fixed point is when $(x^*,y^*) = (0, 0)$.
Furthermore, let's compute the eigenvalues of this system
\begin{align*}
\begin{bmatrix} 0 & 1 \\ -2 & -3 \end{bmatrix}
\begin{bmatrix} x \\ y \end{bmatrix}
	&= \lambda \begin{bmatrix} x \\ y \end{bmatrix} \\
\begin{bmatrix} 0 & 1 \\ -2 & -3 \end{bmatrix}
\begin{bmatrix} x \\ y \end{bmatrix} - \lambda \begin{bmatrix} x \\ y \end{bmatrix}
	&= \bm { 0 } \\
\left( \begin{bmatrix} 0 & 1 \\ -2 & -3 \end{bmatrix}
 - \lambda I \right)\begin{bmatrix} x \\ y \end{bmatrix}
	&= \bm { 0 } \\
\begin{bmatrix} -\lambda & 1 \\ -2 & -3 - \lambda \end{bmatrix}
\begin{bmatrix} x \\ y \end{bmatrix}
	&= \bm { 0 } \\
\end{align*}
This implies that we need
\begin{align*}
\det{ \left( \begin{bmatrix} -\lambda & 1 \\ -2 & -3 - \lambda \end{bmatrix} \right)} &= 0 \\
-\lambda(-3 -\lambda) + 2 &= 0 \\
\lambda^2 + 3\lambda + 2 &= 0.
\end{align*}
Thus we have
\begin{align*}
\lambda &= \frac{-3 \pm \sqrt{9 - 4(2)}} 2 \\
	&= \frac{-3 \pm 1} 2 \\
	&= -1, -2.
\end{align*}
Now we can calculate the eigenvectors!
First for $\lambda = -1$
\begin{align*}
\begin{bmatrix} 1 & 1 \\ -2 & -2 \end{bmatrix}
\begin{bmatrix} x \\ y \end{bmatrix}
	&= \bm { 0 } \\
\end{align*}
\textbf{TODO: calculate eigenvalues and eigenvectors} \\

\item $\dot x = 3x - 4y, \quad \dot y = x - y$ \\
\textit{Solution:} \\
We can put this in the form of a 2 dimensional first order system as follows
\begin{align*}
\begin{bmatrix}
x \\ y
\end{bmatrix}^\prime
&= \begin{bmatrix}
3 & -4 \\
1 & -1
\end{bmatrix} \begin{bmatrix}
x \\ y
\end{bmatrix}
\end{align*}
Implying
\textbf{TODO: calculate eigenvalues and eigenvectors} \\

\item $\ddot x + 2\dot x - x = 0$ \\
\textit{Solution:} \\
We can put this second order system in the form of a 2 dimensional first order system as follows.
Set $x_1 = x$ and $x_2 = \dot x$
\begin{align*}
\begin{bmatrix}
x_1 \\ x_2
\end{bmatrix}^\prime
&= \begin{bmatrix}
0 & 1 \\
1 & -2
\end{bmatrix} \begin{bmatrix}
x_1 \\ x_2
\end{bmatrix}
\end{align*}
\textbf{TODO: calculate eigenvalues and eigenvectors} \\

\end{enumerate}

\newpage

\item There is a lot of context and build up to this problem in the assignment pdf which I will not transcribe here.
The important details will be included in their respective parts. \\

\begin{enumerate}

\item Rewrite the second-order differential equation
$$L \ddot I + \dot I R + \frac I C = 0$$
as a 2-D linear dynamical system. \\
\textit{Solution:} \\
If we let $I_1 = I$ and $I_2 = \dot I$ then we have
\begin{align*}
\begin{bmatrix}
I_1 \\ I_2
\end{bmatrix}^\prime
&= \begin{bmatrix}
0 & 1 \\
-\frac 1 {LC} & -\frac R L
\end{bmatrix} \begin{bmatrix}
I_1 \\ I_2
\end{bmatrix}
\end{align*}
\textbf{TODO} \\

\item Show that the origin is asymptotically stable if $R > 0$ and neutrally stable if $R = 0$. \\
\textit{Solution:} \\
In order to classify it in these scenarios we want to calculate the eigenvalues of the matrix representing our DS.
We first need to calculate the characteristic equation as follows
\begin{align*}
-\lambda \left(- \frac R L - \lambda \right) + \frac 1 {LC} &= 0 \\
\lambda^2 + \lambda \frac R L + \frac 1 {LC} &= 0.
\end{align*}
This gives us
\begin{equation}
\lambda = \frac {-R/L \pm \sqrt{R^2/L^2 - 4/(LC)} }{2}.
\label{eq:eq2}
\end{equation}
Now if $R > 0$, then the origin is asymptotically stable since the real part of the eigenvalues will be negative.
However, if the real part of the eigenvalues is $0$ or rather the eigenvalues are purely imaginary, then the origin will be neutrally stable.
\textbf{TODO: Check logic with someone else.} \\

\item Classify the fixed point at the origin depending on whether $R^2C - 4L$ is positive, negative, or zero. \\
\textit{Solution:} \\
Referring back to equation (\ref{eq:eq2}) we can see that the value of interest here is within a scalar of the discriminant which determines whether or not there is any imaginary component to the eigenvalues.
See the following minor rewriting of equation (\ref{eq:eq2})
\begin{align*}
\lambda &= \frac {-R/L \pm \sqrt{R^2/L^2 - 4/(LC)} }{2} \\
\lambda &= \frac {-R/L \pm \sqrt{(R^2 - 4L/C)/L^2} }{2} \\
\lambda &= \frac {-R/L \pm \sqrt{(R^2C - 4L)/(L^2C)} }{2}.
\end{align*}
Since we are given in the problem statement that $C > 0$ and $L > 0$, then $L^2C > 0$.
Therefore, the sign of the discriminant and therefore it's real or imaginaryness is determined purely by the value of $R^2C - 4L$.
\textbf{TODO: Finish analyzing how this impacts the stability of fixed points.} \\

\end{enumerate}

\newpage

\item From pdf \\

\begin{enumerate}

\item (Not graded) \\
\textit{Solution:} \\
\textbf{TODO} \\

\item Do the following for $a > 1$ and $ a < 1$ \\

\begin{enumerate}

\item Determine \\
\textit{Solution:} \\
\textbf{TODO} \\

\item Verify \\
\textit{Solution:} \\
\textbf{TODO} \\

\end{enumerate}

\item Summarize \\
\textit{Solution:} \\
\textbf{TODO} \\

\end{enumerate}

\end{enumerate}

\end{document}

%%% Local Variables:
%%% mode: latex
%%% TeX-master: t
%%% End:
