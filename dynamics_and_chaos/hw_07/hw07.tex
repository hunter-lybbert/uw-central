\documentclass[10pt]{amsart}
\usepackage[margin=1.4in]{geometry}
\usepackage{amssymb,amsmath,enumitem,url}
\usepackage{graphicx,subfig}
\graphicspath{ {./images/} }

\newcommand{\D}{\mathrm{d}}
\newcommand{\I}{\mathrm{i}}
\DeclareMathOperator{\E}{e}
\DeclareMathOperator{\OO}{O}
\DeclareMathOperator{\oo}{o}
\DeclareMathOperator{\erfc}{erfc}
\DeclareMathOperator{\real}{Re}
\DeclareMathOperator{\imag}{Im}
\usepackage{tikz}
\usepackage[framemethod=tikz]{mdframed}
\theoremstyle{nonumberplain}

\mdtheorem[innertopmargin=-5pt]{sol}{Solution}
%\newmdtheoremenv[innertopmargin=-5pt]{sol}{Solution}

\begin{document}
\pagestyle{empty}

\newcommand{\mline}{\vspace{.2in}\hrule\vspace{.2in}}

\noindent
\text{Hunter Lybbert} \\
\text{Student ID: 2426454} \\
\text{03-04-25} \\
\text{AMATH 502} \\
% header containing your name, student number, due date, course, and the homework number as a title.

\title{\bf {Homework 7} }


\maketitle
\noindent
Exercises come from the assignment sheet provided by the professor on canvas.
\mline
\begin{enumerate}[label={\bf {\arabic*}:}]
\item A powerful tool for numerically finding the roots of an equation $g(x) = 0$ is \textit{Newton's Method}.
Newton's method says to construct a map $x_{n + 1} = f(x_n)$, where 
$$
f(x_n) = x_n - \frac{g(x_n)}{g^\prime(x_n)}
$$. 
\begin{enumerate}

\item A simple root of the function $g(x)$ is defined as a value $x$ for which $g(x) = 0$ and $g^\prime(x) \neq 0$.
Show that the simple roots of $g(x)$ are fixed points of the Newton Map. \\

\textit{Solution:} \\
Let's first assume $x^*$ is a simple root.
Therefore, $g(x^*) = 0$ and $g^\prime(x^*) \neq 0$, for notation let $g^\prime(x^*) = a$ where $a \neq 0$.
This also implies that 
\begin{align}
f(x^*) &= x^* - \frac{g(x^*)}{g^\prime(x^*)} \nonumber \\
f(x^*) &= x^* - \frac 0 a \nonumber \\
f(x^*) &= x^*.
\label{eq:eq1}
\end{align}
Notice, the definition of a fixed point in a discrete time system is $f(x_n) = x_n$ which is exactly what we are left with in \eqref{eq:eq1}.
Therefore, $x^*$ is a fixed point. \\
\qed \\

\item Show that these fixed points are \textit{superstable}, which means that the linear stability analysis shows \textit{zero} growth for perturbations $(f^\prime(x^*) = 0).$

\textit{Solution:} \\
\textbf{TODO} \\

\end{enumerate}

\item Consider the map $x_{n + 1} = 3x_n - x_n^3$.
This well-studied map is an example of a cubic map and is known to exhibit chaos.
\begin{enumerate}

\item Find all the fixed points and classify their stability. \\

\textit{Solution:} \\
\textbf{TODO} \\

\item In Figure 1, you are given the cobweb diagrams for $x_0 = 1.9$ and $x_0 = 2.1$.
Show analytically that if $|x| \leq 2$, then $|f(x)| \leq 2$, where $f(x) = 3x - x^3$.
Then show that if $|x| > 2$, $|f(x)| > |x|$.
Use this to explain the behavior in cobweb diagrams for $x_0 = 1.9$ and $x_0 = 2.1$.  \\

\textit{Solution:} \\
\textbf{TODO} \\

\item Show that (2, -2) (repeating) is a 2 cycle.
This 2 cycle is analogous to a boundary that we defined when we were doing phase-plane analysis.
What would you call this 2-cycle? (Not a limit cycle or a periodic orbit). \\

\textit{Solution:} \\
\textbf{TODO} \\

\end{enumerate}

\item Consider a 1D ODE
\begin{equation}
\dot x = f(x), \quad x \in \mathbb R.
\label{eq:eq2}
\end{equation}
The most basic method for solving this ODE numerically is to use the Forward Euler method,
\begin{equation}
x_{n + 1} = x_n + hf(x_n),
\label{eq:eq3}
\end{equation}
where $h > 0$ is a chosen step size.
This method comes from discretizing the derivative, as discussed in class. \\

\begin{enumerate}
\item Show that fixed points of the ODE \eqref{eq:eq2} correspond to fixed points of the Forward Euler map \eqref{eq:eq3}. \\

\textit{Solution:} \\
\textbf{TODO} \\

\item Show that stability of the fixed points of the ODE \eqref{eq:eq2} do not necessarily agree with the stability of the fixed points of the Forward Euler map \eqref{eq:eq3}. \\

\textit{Solution:} \\
\textbf{TODO} \\

\item Give a condition which guarantees stability of fixed points of the Forward Euler map \eqref{eq:eq2}. 
Comment on this condition: how must we generally choose the step size $h$ in order to find equilibrium solutions of the ODE \eqref{eq:eq3} using the Forward Euler method? \\

\textit{Solution:} \\
\textbf{TODO} \\

\item It is common to see the Forward Euler solution oscillating about the true solution when solving numerically.
Give a condition involving $f^\prime(x)$ and $h$ for which the numerical solution oscillates about a fixed point of the ODE \eqref{eq:eq2} (hint: when did we have oscillations for the linear discrete-time dynamical systems?).
Given this condition, why is it common to see oscillations in the Forward-Euler solution (hint: see above problem)? \\

\textit{Solution:} \\
\textbf{TODO} \\

\item Consider a linear ODE, 
\begin{equation}
\dot x = kx, \quad k \in \mathbb R.
\label{eq:eq4}
\end{equation}
Give a condition on $h$ and $k$ for which 2-cycles (the non-fixed point 2 cycles) exist for the Forward-Eualer map when solving this ODE.
Show that these 2 cycles are neutrally stable.
Comment on your results (in particular, when $h$ and $k$ match your condition, what happens to the numerical solution for any initial condition you use?). \\

\textit{Solution:} \\
\textbf{TODO} \\

\end{enumerate}
\end{enumerate}

\end{document}

%%% Local Variables:
%%% mode: latex
%%% TeX-master: t
%%% End:
