\documentclass[10pt]{amsart}
\usepackage[margin=1.4in]{geometry}
\usepackage{amssymb,amsmath,enumitem,url}
\usepackage{graphicx,subfig}
\graphicspath{ {./images/} }

\newcommand{\D}{\mathrm{d}}
\newcommand{\I}{\mathrm{i}}
\DeclareMathOperator{\E}{e}
\DeclareMathOperator{\OO}{O}
\DeclareMathOperator{\oo}{o}
\DeclareMathOperator{\erfc}{erfc}
\DeclareMathOperator{\real}{Re}
\DeclareMathOperator{\imag}{Im}
\usepackage{tikz}
\usepackage[framemethod=tikz]{mdframed}
\theoremstyle{nonumberplain}

\mdtheorem[innertopmargin=-5pt]{sol}{Solution}
%\newmdtheoremenv[innertopmargin=-5pt]{sol}{Solution}

\begin{document}
\pagestyle{empty}

\newcommand{\mline}{\vspace{.2in}\hrule\vspace{.2in}}

\noindent
\text{Hunter Lybbert} \\
\text{Student ID: 2426454} \\
\text{03-13-25} \\
\text{AMATH 502} \\
% header containing your name, student number, due date, course, and the homework number as a title.

\title{\bf {Homework 8} }


\maketitle
\noindent
Exercises come from the assignment sheet provided by the professor on canvas.
\mline
\begin{enumerate}[label={\bf {\arabic*}:}]
\item See assignment document for background on the \textit{Sierpinski Carpet}.
\begin{enumerate}

\item Show that the (Lebesgue) measure of the resulting fractal is 0.
Justify your work. \\

\textit{Solution:} \\
\textbf{TODO} \\

\item Find the similarity dimension of the limiting fractal. Show and explain your work. \\

\textit{Solution:} \\
\textbf{TODO} \\

\item Show that the box-counting dimension of this fractal is the same as the similarity dimension. (You may not be able to complete this until box-counting dimension is defined in class). \\

\textit{Solution:} \\
\textbf{TODO} \\

\item Show that there are uncountably many points on the interior of the limiting fractal (i.e. infinitely many points without x = 0, x = 1, y = 0, or y = 1).
Hint: Try showing this in just one dimension, e.g., show that there are uncountably many points (x, y) with a fixed value of y, perhaps y = 0.5 is a good choice. \\

\textit{Solution:} \\
\textbf{TODO} \\

\newpage

\end{enumerate}

\item In this problem we construct what is called a middle-halves Cantor set.
Consider the following Cantor set construction. Start with the interval [0, 1], then remove the middle half. Continue this process for each sub-interval. \\
\begin{enumerate}

\item Draw $S_1$ and $S_2$. \\

\textit{Solution:} \\
\textbf{TODO} \\

\item Find the similarity dimension of the set. \\

\textit{Solution:} \\
\textbf{TODO} \\

\item Find the measure of the set. \\

\textit{Solution:} \\
\textbf{TODO} \\

\end{enumerate}

\newpage

\item See assignment sheet for details of the \textit{fat fractal}. \\

\begin{enumerate}
\item Find the (Lebesgue) measure of this cantor set. Show your work. \\

\textit{Solution:} \\
\textbf{TODO} \\

\item Is the fractal self similar?
Justify your answer. \\
\textbf{Hint:} Can you find the similarity dimension of this set?
What happens when you try?) \\
\textbf{Note:} You may find this part to be difficult.
If you are struggling with it, you may want to skip it for now and come back to it later. 
\\
\end{enumerate}
 
 \newpage
 
\item The tent map on the interval [0, 1] is defined by $x_{n + 1} = f(x_n)$, where 
$$
f(x) = \begin{cases}
rx,  &0 \leq x \leq \frac 1 2 \\
r(1 - x), &\frac 1 2 < x \leq 1
\end{cases}.
$$
Assume that $r > 2$.
Then some points get mapped outside of the interval [0, 1].
If $f(x_0) > 1$ then we say that $x_0$ has ``escaped" after one iteration.
Similarly, if $f^n(x_0) > 1$ for some finite $n$ and $n$ is the smallest integer for which this is true, then we say $x_0$ has escaped after $n$ iterations.

\begin{enumerate}
\item Find the set of initial conditions x0 that escape after one iteration. \\
\item Find the set of initial conditions x0 that escape after two iterations. \\
\item Describe the set of x0 that never escape. This is called the invariant set. 
Hint: First look at what happens for r = 3. Does this look like a set you 
recognize? \\
\item Find the box dimension of the invariant set (for general r, not r = 3). \\
\item This is just a note, you don’t need to answer anything here. After completing
this problem, you will have shown that the invariant set of this chaotic map
forms a fractal. Cool! This invariant set is called a strange repeller because it
is a fractal set that repels all nearby points that are not in the set and points
in the set are part of a chaotic orbit.

\end{enumerate}

\end{enumerate}

\end{document}

%%% Local Variables:
%%% mode: latex
%%% TeX-master: t
%%% End:
