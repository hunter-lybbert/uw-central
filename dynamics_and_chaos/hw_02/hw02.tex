\documentclass[10pt]{amsart}
\usepackage[margin=1.4in]{geometry}
\usepackage{amssymb,amsmath,enumitem,url}
\usepackage{graphicx,subfig}
\graphicspath{ {./images/} }

\newcommand{\D}{\mathrm{d}}
\newcommand{\I}{\mathrm{i}}
\DeclareMathOperator{\E}{e}
\DeclareMathOperator{\OO}{O}
\DeclareMathOperator{\oo}{o}
\DeclareMathOperator{\erfc}{erfc}
\DeclareMathOperator{\real}{Re}
\DeclareMathOperator{\imag}{Im}
\usepackage{tikz}
\usepackage[framemethod=tikz]{mdframed}
\theoremstyle{nonumberplain}

\mdtheorem[innertopmargin=-5pt]{sol}{Solution}
%\newmdtheoremenv[innertopmargin=-5pt]{sol}{Solution}

\begin{document}
\pagestyle{empty}

\newcommand{\mline}{\vspace{.2in}\hrule\vspace{.2in}}

\noindent
\text{Hunter Lybbert} \\
\text{Student ID: 2426454} \\
\text{01-23-25} \\
\text{AMATH 502} \\
% header containing your name, student number, due date, course, and the homework number as a title.

\title{\bf {Homework 2} }


\maketitle
\noindent
Exercises come from \textit{Nonlinear Dynamics and Chaos by Steven H. Strogatz}
\mline
\begin{enumerate}[label={\bf {\arabic*}:}]
\item 2.6.1 Explain this paradox: a simple harmonic oscillator $m\ddot {x} = -k x$ is a system which oscillates in one dimension (along the $x$-axis). But the text says one-dimensional systems can't oscillate. \\

\noindent
\textit{Solution:} \\
Not a formal method of finding a solution but I can see that if $x = \sin \left(  \frac k m t \right)$ then
\begin{align*}
\dot x &= \frac k m \cos \left(  \frac k m t \right) \\
\ddot x &= - \frac {k^2} {m^2} \sin \left(  \frac k m t \right) \\
\ddot x &= - \frac {k^2} {m^2} x \\
m^2\ddot x &= - k^2 x
\end{align*}
Therefore to adjust the constants $k$ and $m$ so they agree with the original solution we need to actually have $x = \sin \big(  \sqrt {k/m} \: t \big)$.
Furthermore, we could similarly have arrived at a similar solution with $x = \cos \big(  \sqrt {k/m} \: t \big)$.
Therefore, we have $x = c_1 \sin \big(  \sqrt {k/m} \: t \big) + c_2 \cos \big(  \sqrt {k/m} \: t \big)$. 
Now as for how this helps me determine that this is not a contradiction I have no idea. \\
\textbf{TODO: Finish. Perhaps it's because ... nah idk. } \\

\newpage

\item 3.1.1 $\dot x = 1 + rx + x^2$ \\
Sketch all the qualitatively different vector fields that occur as $r$ is varied.
Show that a saddle-node bifurcation occurs at a critical value of $r$, to be determined.
Finally sketch the bifurcation diagram of fixed points $x^*$ vs. $r$. \\

\noindent
\textit{Solution:} \\
Well we know that we have fixed points at
$$x = \frac {-r \pm \sqrt {r^2 - 4}}{2}$$
\textbf{TODO} \\

\item 3.1.5 (Unusual bifurcations) In discussing the normal form of the saddle-node bifurcation, we mentioned the assumption that $a = \left. \partial f / \partial r \right|_{(x^*, r_c)} \neq 0.$
To see what can happen if $a = \left. \partial f / \partial r \right|_{(x^*, r_c)} = 0$, sketch the vector fields for the following examples, and then plot the fixed points as a function of $r$. \\

\begin{enumerate}
\item $\dot x = r^2 - x^2$ \\
\textit{Solution:} \\
\textbf{TODO} \\

\item $\dot x = r^2 + x^2$ \\
\textit{Solution:} \\
\textbf{TODO} \\
\end{enumerate}


\item 3.2.3 $\dot x = x - rx(1 - x)$ \\
Sketch all the qualitatively different vector fields that occur as $r$ is varied.
Show that transcritical bifurcation occurs at a critical value of $r$, to be determined.
Finally, sketch the bifurcation diagram of fixed points $x^*$ vs. $r$. \\

\noindent
\textit{Solution:} \\
\textbf{TODO} \\

\item 3.4.11 (An interesting bifurcation diagram) Consider $\dot x = rx - \sin x$ \\
\begin{enumerate}

\item For the case $r = 0$, find and classify all the fixed points, and sketch the vector field. \\

\noindent
\textit{Solution:} \\
\textbf{TODO} \\

\item Show that when $r > 1$, there is only one fixed point.
What kind of fixed point is it? \\

\noindent
\textit{Solution:} \\
\textbf{TODO} \\

\item As $r$ decreases from $\infty$ to $0$, classify \textit{all} the bifurcations that occur. \\

\noindent
\textit{Solution:} \\
\textbf{TODO} \\

\item For $0 < r << 1$, find an approximate formula for values of $r$ at which bifurcations occur. \\

\noindent
\textit{Solution:} \\
\textbf{TODO} \\

\item Now classify all the bifurcations that occur as $r$ decreases from 0 to $-\infty$. \\

\noindent
\textit{Solution:} \\
\textbf{TODO} \\

\item Plot the Bifurcation diagram for $-\infty < r < \infty$, and indicate the stability of the various branches of fixed points. \\

\noindent
\textit{Solution:} \\
\textbf{TODO} \\

\end{enumerate}

\item 3.5.8 (Nondimensionalizing the subcritical pitchfork) The first-order system $\dot u = au + bu^3 - c u^5$, where $b, c > 0$, has a subcritical pitchfork bifurcation at $a = 0$.
Show that this equation can be rewritten as
$$\frac {dx}{d\tau} = r x + x^3 - x^5$$
where $x = u/U$, $\tau = t/T$, and $U$, $T$, and $r$ are to be determined in terms of $a$, $b$, and $c$. \\

\noindent
\textit{Solution:} \\
\textbf{TODO} \\

\newpage

\item 3.7.3 (A model of a fishery) The equation
$$\dot N = rN\left( 1 - \frac N K \right) - H$$
provides an extremely simple model of a fishery.
In the absence of fishing, the population is assumed to grow logistically. 
The effects of fishing are modeled by the term $-H$, which says that fish are caught or ``harvested" at a constant rate $H > 0$, independent of their population $N$.
(This assumes that the fisherman aren't worried about fishing the population dry--they simply catch the same number of fish every day.) \\

\begin{enumerate}

\item Show that the system can be rewritten in dimensionless form as 
$$
\frac {dx}{d\tau} = x(1 - x) - h
$$
for suitably defined dimensionless quantities $x$, $\tau$, and $h$ \\

\noindent
\textit{Solution:} \\
\textbf{TODO} \\

\item Plot the vector field for different values of $h$ \\

\noindent
\textit{Solution:} \\
\textbf{TODO} \\

\item Show that a bifurcation occurs at a certain value of $h_c$, and classify this bifurcation. \\

\noindent
\textit{Solution:} \\
\textbf{TODO} \\

\item Discuss the long-term behavior of the fish population for $h < h_c$ and $h > h_c$. 
Give the biological interpretation in each case. \\

\noindent
\textit{Solution:} \\
\textbf{TODO} \\

\end{enumerate}

\end{enumerate}

\end{document}

%%% Local Variables:
%%% mode: latex
%%% TeX-master: t
%%% End:
