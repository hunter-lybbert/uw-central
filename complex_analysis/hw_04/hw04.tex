\documentclass[10pt]{amsart}
\usepackage[margin=1.4in]{geometry}
\usepackage[usenames,dvipsnames,cmyk]{xcolor} %load first
\usepackage{cancel}
\usepackage{graphicx,subfig}
\graphicspath{ {./images/} }

\usepackage{amssymb,amsmath,enumitem,url}

\newcommand{\D}{\mathrm{d}}
\newcommand{\I}{\mathrm{i}}
\DeclareMathOperator{\E}{e}
\DeclareMathOperator{\OO}{O}
\DeclareMathOperator{\oo}{o}
\DeclareMathOperator{\erfc}{erfc}
\DeclareMathOperator{\real}{Re}
\DeclareMathOperator{\imag}{Im}
\usepackage{tikz}
\usepackage[framemethod=tikz]{mdframed}
\theoremstyle{nonumberplain}

\mdtheorem[innertopmargin=-5pt]{sol}{Solution}
%\newmdtheoremenv[innertopmargin=-5pt]{sol}{Solution}
\definecolor{MichiganBlue}{HTML}{00274C}
\definecolor{MichiganYellow}{HTML}{FFCB05}  
\definecolor{NicePurple}{RGB}{75,56,76} %PrincePurple
\definecolor{NiceRed}{RGB}{230,37,52}
\definecolor{MidnightBlue}{rgb}{0.1, 0.1, 0.44}
\usepackage[colorlinks=true, linkcolor=MidnightBlue, citecolor=MidnightBlue, urlcolor=MidnightBlue]{hyperref}

\begin{document}
\pagestyle{empty}

\newcommand{\mline}{\vspace{.2in}\hrule\vspace{.2in}}

\noindent
\text{Hunter Lybbert} \\
\text{Student ID: 2426454} \\
\text{10-21-24} \\
\text{AMATH 567} \\

\title{\bf { Homework 4} }


\maketitle
\noindent
Collaborators*: Nate Ward, Erin Szalda-Petree, Hailey Sparks, Laura Thomas, Cooper Simpson, Jenny Jin \\
\\
\tiny
\text{*Listed in no particular order. And anyone I discussed at least part of one problem with is considered a collaborator.}
\normalsize
\mline
\noindent
\textbf{TODO:} Now that you know the `$\I$' symbol exists, consider changing each of the `$i$' characters to `$\I$' characters.
\begin{enumerate}[label={\bf {\arabic*}:}]
\item From A\&F: 2.4.2 c, e.\\
Evaluate the integral $\oint_Cf(z)\D z$, where C is the unit circle enclosing the origin, and $f(z)$ is given as follows: \\
c) $$f(z) = \frac {1} {\bar z}$$
\textit{Solution:} \\
We want to evaluate
$$
\oint_C\frac {1} {\bar z} \D z
$$
on the parameterized unit circle $z = \E^{i\theta}$ where $\theta \in [0, 2\pi)$, where $\bar z = \E^{-i\theta}$ on the unit circle.
Note, before we do the substitution we need $\D z = i\E^{i\theta} \D \theta$. Now our integral is
\begin{align*}
\oint_C\frac {1} {\bar z} \D z &= \int_0^{2\pi}\frac {1} {\E^{-i\theta}} i\E^{i \theta}\D \theta \\
					  &= \int_0^{2\pi} i \E^{2i \theta} \D \theta \\
					  &= \left( \left. \frac 1 2 \E^{2i\theta} \right|_0^{2\pi}\right) \\
					  &= \frac 1 2 \E^{4\pi i} - \frac 1 2 \E^{0} \\
					  &= \frac 1 2 - \frac 1 2 \\
					  &= 0
\end{align*}
\qed
\\
e) $$f(z) = \E^{\bar z}$$
\textit{Solution:} \\
We will use the same substitutions from the previous part
\begin{align*}
\oint_C\E^{\bar z} \D z &= \int_0^{2\pi}\E^{\E^{-i\theta}} i\E^{i \theta}\D \theta \\
				  &= \int_0^{2\pi} \sum_{j=0}^{\infty} \frac{\left(\E^{-i\theta}\right)^j}{j!} i\E^{i\theta} \D \theta \\
				  &= \sum_{j=0}^{\infty} \int_0^{2\pi} i \frac{\left(\E^{-i\theta}\right)^j}{j!} {\E^{i\theta}} \D \theta.
\end{align*}
We are justified in reordering the integral of the infinite sum to be the infinite sum of the integrals since the original series converges absolutely.
Notice we can pull out the first term where $j=1$ to get,
$$
\sum_{j=0}^{\infty} \int_0^{2\pi} i \frac{\left(\E^{-i\theta}\right)^j}{j!} {\E^{i\theta}} \D \theta
= \int_0^{2\pi} i \frac{\left(\E^{-i\theta}\right)^0}{0!} {\E^{i\theta}} \D \theta
+ \int_0^{2\pi} i \frac{\left(\E^{-i\theta}\right)^1}{1!} {\E^{i\theta}} \D \theta
+ \sum_{j=2}^{\infty} \int_0^{2\pi} i \frac{\left(\E^{-i\theta}\right)^j}{j!} {\E^{i\theta}} \D \theta.
$$
We need to look more closely at this first two terms, observe
\begin{align*}
\int_0^{2\pi} i \frac{\left(\E^{-i\theta}\right)^0}{0!} {\E^{i\theta}} \D \theta
= \int_0^{2\pi} i \E^{i\theta} \D \theta
= \left. \E^{i\theta} \right|_0^{2\pi}
= \E^{i2\pi} - \E^{0}
= 0
\end{align*}
and
\begin{align*}
\int_0^{2\pi} i \frac{\left(\E^{-i\theta}\right)^1}{1!} \E^{i\theta} \D \theta = \int_0^{2\pi} i \frac{\E^{-i\theta}\E^{i\theta}}{1!} \D \theta = \int_0^{2\pi} i \frac{1}{1} \D \theta = i\int_0^{2\pi}\D \theta = 2\pi i.
\end{align*}
Now, I will focus on the integral inside the sum where $j = 2, 3, 4, ...$
\begin{align*}
\int_0^{2\pi} i \frac{\left(\E^{-i\theta}\right)^j}{j!} {\E^{i\theta}} \D \theta &= \int_0^{2\pi} i \frac{\E^{-i\theta j}\E^{i\theta}}{j!} \D \theta \\
	&= \int_0^{2\pi} i \frac{\E^{-i\theta j + i\theta}}{j!} \D \theta \\
	&= \int_0^{2\pi} i \frac{\E^{i\theta\left( - j + 1\right)}}{j!} \D \theta \\
	&= \int_0^{2\pi} \frac{i\E^{i\theta\left(1 - j\right)}}{j!} \D \theta \\
	&= \left. \frac{1}{1 - j} \frac{i\E^{i\theta\left(1 - j\right)}}{j!} \right|_0^{2 \pi} \\
	&= \frac{1}{1 - j} \frac{i\E^{i2\pi\left(1 - j\right)}}{j!} - \frac{1}{1 - j} \frac{i\E^0}{j!} \\
	&= \frac{i}{\left(1 - j\right)j!}\left(\E^{i2\pi\left(1 - j\right)} - 1 \right) \\
	&= \frac{i}{\left(1 - j\right)j!}\left(1 - 1 \right) \\
	&= 0. \\
\end{align*}
I want to clarify why $\E^{i2\pi\left(1 - j\right)} = 1$.
Since $j \in \{2, 3, ...\}$, then $1 - j$ is an integer and we have $\E^{i2\pi \ell}$ where $\ell \in \mathbb Z$, thus $\E^{i2\pi \ell} = 1$.
Now we return to the original problem
$$
\sum_{j=0}^{\infty} \oint_0^{2\pi} i \frac{\left(\E^{-i\theta}\right)^j}{j!} {\E^{i\theta}} \D \theta = 0 + 2\pi i + \sum_{j=2}^{\infty} 0 = 2\pi i.
$$
Now we have completed the requisite task.
\qed
\\

\item From A\&F: 2.4.4 a, b.
Use the principal branch where the argument is in $[-\pi,\pi)$.
Discuss any ambiguities. 
Use the principal branch of $\log(z)$ and $z^{\frac{1}{2}}$ where the argument is in $[-\pi,\pi)$ to evaluate the following: \\
a) $$\int_{-1}^{1}\log z \D z$$
\textit{Solution:} \\
We want to parameterize this once again using $z = r\E^{i\theta}$ where $\theta \in [-\pi,\pi)$. Now our integral is
\begin{align*}
\int_{-1}^{1}\log z \D z &= \int_{-\pi}^{0}\log \left(\E^{i\theta}\right) i \E^{i\theta} \D \theta \\
	&= \int_{-\pi}^{0} i\theta i \E^{i\theta} \D \theta.
\end{align*}
Let's use integration by parts, woohoo! We will assign the substitutions as follows:
\begin{align*}
u &= i\theta \\
\D u &= i \D \theta\\
\\
\D v &= i\E^{i \theta} \D \theta \\
v &= \E^{i \theta}.
\end{align*}
Plugging this in we have
\begin{align*}
\int_{-\pi}^{0} i\theta i \E^{i\theta} \D \theta &= \left. i\theta \E^{i\theta}\right|_{-\pi}^0 - \int_{-\pi}^0 i \E^{i \theta} \D \theta \\
	&= \left(0 - \left(-i\pi \E^{-i\pi}\right)\right) - \left. \E^{i\theta}\right|_{-\pi}^0 \\
	&= 0 + i\pi \E^{-i\pi} - \left. \E^{i\theta}\right|_{-\pi}^0 \\
	&= i\pi \E^{-i\pi} - \left(\E^{0} - \E^{-i\pi} \right) \\
	&= - i\pi - \left(1 - \left( - 1 \right) \right) \\
	&= - i\pi - \left(2\right) \\
	&= - i\pi - 2. \\
\end{align*}
Ambiguities could arise when integrating along a contour that crosses a branch cut.
Since the parameterization I chose followed the circle of radius 1 in the lower half plane from $-\pi$ to $0$.
If one were to choose to integrate along a contour on the other side of the branch cut they would also arrive at an answer off by a sign from my solution.
\qed
\\

b) $$\int_{-1}^{1}z^{\frac{1}{2}} \D z$$
\textit{Solution:} \\
We want to parameterize this once again using $z = r\E^{i\theta}$ where $\theta \in [-\pi,\pi)$. Now our integral is
\begin{align*}
\int_{-1}^{1} z^{\frac{1}{2}} \D z &= \int_{-\pi}^{0} \left(\E^{i\theta}\right)^{\frac{1}{2}} i \E^{i\theta} \D \theta \\
	&= \int_{-\pi}^{0} i\E^{\frac{i\theta}{2}} \E^{i\theta} \D \theta \\
	&= \int_{-\pi}^{0} i\E^{\frac{i3}{2}\theta} \D \theta \\
	&= \left. \frac 2 3 \E^{\frac{i3}{2}\theta} \right|_{-\pi}^{0} \\
	&= \frac 2 3 - \frac 2 3 \E^{ - \frac{i3}{2} \pi}. \\
\end{align*}
Now remembering our branch cut limits $\theta$ to be within $[-\pi, \pi)$ we change the angle $-\frac 3 2 \pi$ to be $\frac 1 2 \pi$.
Hence,
\begin{align*}
\frac 2 3 - \frac 2 3 \E^{ - \frac{i3}{2} \pi}
	&= \frac 2 3 \left( 1 - \E^{ - \frac{i3}{2} \pi} \right) \\
	&= \frac 2 3 \left( 1 - \E^{ - \frac{i2}{2} \pi} \E^{ - \frac{i\pi}{2}} \right) \\
	&= \frac 2 3 \left( 1 - \left(-\E^{ - \frac{i\pi}{2}} \right) \right) \\
	&= \frac 2 3 \left( 1 - \E^{\frac{i\pi}{2}} \right) \\
	&= \frac 2 3 \left( 1 - i \right).
\end{align*}
Ambiguities could arise when integrating along a contour that crosses a branch cut.
Since the parameterization I chose followed the circle of radius 1 in the lower half plane from $-\pi$ to $0$.
If one were to choose to integrate along a contour on the other side of the branch cut they would also arrive at an answer off by a sign from my solution.
\qed
\\

\item From A\&F: 2.4.7 \\
Let $C$ be an open (upper) semicircle of radius $R$ with its center at the origin, and consider $\int_C f(z) \D z$.
 Let $f(z) = \frac{1}{z^2 + a^2}$ for a real $a > 0$.
Show that $\left| f(z) \right| \leq \frac{1}{R^2 - a^2}$, $R > a$, and
$$
\left| \int_C f(z) dz \right| \leq \frac{\pi R}{R^2 - a^2}, \quad R > a.
$$
\textit{Solution:} \\
First, we want to show
$$
\left| f(z) \right| \leq \frac{1}{R^2 - a^2}
$$
where $R > a > 0$ and a $\in \mathbb R$.
Let's consider the function more closely
\begin{align*}
f(z) =  \frac{1}{z^2 + a^2} &= \frac{1}{(x + \I y)^2 + a^2} \\
	&= \frac{1}{x^2 + 2ixy - y^2 + a^2} \\
	&= \frac{1}{x^2 - y^2 + a^2 + i2xy }. \\
\end{align*}
Notice, we can write the real and imaginary parts of the complex number in the denominator as functions $u(x, y)$ and $v(x, y)$.
Where $u(x, y) = x^2 - y^2 + a^2$ and $v(x, y) = 2xy$.
Now we get 
$$
f(z) = \frac{1}{x^2 - y^2 + a^2 + i2xy }	 = \frac{u - iv }{u - iv }\frac{1}{u + iv } = \frac{u - iv }{u^2 + v^2 }.
$$
Then we calculate
\begin{align*}
\left| f(z) \right| &= \left| \frac{u - iv }{u^2 + v^2 } \right| \\
	&= \left| \frac{u }{u^2 + v^2 } - i\frac{v}{u^2 + v^2 } \right| \\
	&= \sqrt{
		\left(\frac{u }{u^2 + v^2 }\right)^2 + \left(\frac{v }{u^2 + v^2 }\right)^2
	} \\
	&= \sqrt{
		\frac{u^2 }{\left( u^2 + v^2 \right)^2} + \frac{v^2 }{\left( u^2 + v^2 \right)^2}
	} \\
	&= \sqrt{
		\frac{u^2 + v^2 }{\left( u^2 + v^2 \right)^2}
	} \\
	&= \sqrt{
		\frac{1}{u^2 + v^2}
	} \\
	&= \frac{1}{\sqrt{u^2 + v^2}}.
\end{align*}
If we plug our substitution back in we see
\begin{align*}
\frac{1}{\sqrt{u^2 + v^2}} &= \frac{1}{\sqrt{\left(x^2 - y^2 + a^2 \right)^2 + \left(2xy \right)^2}} \\
	&= \frac{1}{\sqrt{\left(x^2 - y^2 + a^2 \right)\left(x^2 - y^2 + a^2 \right) + 4x^2y^2}} \\
	&= \frac{1}{
		\sqrt{
			x^4 - x^2y^2 + x^2a^2 -x^2y^2 + y^4 - y^2a^2 + x^2a^2 - y^2a^2 + a^4 + 4x^2y^2
		}
	} \\
	&= \frac{1}{
		\sqrt{
			x^4 + x^2a^2 + y^4 - y^2a^2 + x^2a^2 - y^2a^2 + a^4 + 2x^2y^2
		}
	}.
\end{align*}
Now we add zero in a particular fashion, namely $- 4x^2a^2 + 4x^2a^2 $, so we can regroup the terms and refactor to get closer to what we desire
\begin{align*}
	&= \frac{1}{
		\sqrt{
			x^4 + y^4 - y^2a^2 + 2x^2a^2 - y^2a^2 + a^4 + 2x^2y^2 +\left( - 4x^2a^2 + 4x^2a^2 \right)
		}
	} \\
	&= \frac{1}{
		\sqrt{
			x^4 + y^4 - 2y^2a^2 + a^4 + 2x^2y^2 - 2x^2a^2 + 4x^2a^2
		}
	} \\
	&= \frac{1}{\sqrt{\left(x^2 + y^2 - a^2\right)^2 + \left(2xa\right)^2}}.
\end{align*}
Using the fact that $\sqrt{a + b} \geq \sqrt{a}$ for $a, b > 0$, in our next step we get a smaller denominator which makes the overall expression greater or equal to the previous step. Note, equality only holds when $x=0$.
\begin{align*}
 \frac{1}{\sqrt{\left(x^2 + y^2 - a^2\right)^2 + \left(2xa\right)^2}}
 	&\leq \frac{1}{\sqrt{\left(x^2 + y^2 - a^2\right)^2}} \\
	&= \frac{1}{x^2 + y^2 - a^2} \\
	&= \frac{1}{|z|^2 - a^2} \\
	&= \frac{1}{R^2 - a^2}
\end{align*}
Therefore $\left| f(z) \right| \leq \frac{1}{R^2 - a^2}$. \\
\qed

\noindent
Next we wish to show that 
$$
\left| \int_C f(z) dz \right| \leq \frac{\pi R}{R^2 - a^2}, \quad R > a.
$$
By Theorem 2.4.2 from A\&F, if $f(z)$ is continuous on contour C, then
$$
\left| \int_C f(z) dz \right| \leq ML
$$
where $L$ is the length of $C$ and $M$ is an upper bound for $\left| f(z) \right|$.
We have that $C$ is continuous, since $a >0$ and $a < R$ and there are no singularities or weirdness with $f(z)$ on the specified contour.
So we have 
$$
M = \frac{1}{R^2 - a^2}
$$
as we calculated in the first part of this problem.
Additionally, we know the arc length of $C$ is easy to calculate because it is half the circumference of the circle with radius $R$.
To convince myself of this I will show the general arc length formula also provides this quick calculation.
Let our parameterization of this contour be $$z(\theta) = R\E^{i\theta}$$ where $\theta \in [0, \pi)$. Then $$z^\prime(\theta) = Ri\E^{i\theta} = -R\sin\theta + iR\cos\theta.$$
Therefore calculating arc length is as follows,
\begin{align*}
L &= \int_a^b |z^\prime(t)| \D t \\
	&= \int_{0}^{\pi} \left| -R\sin\theta + iR\cos\theta \right| \D \theta \\
	&= \int_{0}^{\pi} \sqrt{ R^2\sin^2\theta + R^2\cos^2\theta } \D \theta \\
	&= \int_{0}^{\pi} \sqrt{ R^2\left(\sin^2\theta + \cos^2\theta\right) } \D \theta \\
	&= \int_{0}^{\pi} R \D \theta \\
	&= \pi R.
\end{align*}
Which is the same as half the circumference ($\frac 1 2 2\pi R = \pi R$).
And thus
$$
\left| \int_C f(z) dz \right| \leq ML \leq \frac{1}{R^2 - a^2} \pi R = \frac{\pi R}{R^2 - a^2}.
$$
Hence, 
$$
\left| \int_C f(z) dz \right| \leq \frac{\pi R}{R^2 - a^2}
$$
as desired.
\qed
\\
\item From A\&F: 2.4.8 \\
Let $C$ be an arc of the circle $\left|z\right| = R$ with $(R > 1)$ of angle $\frac{\pi}{3}$.
Show that 
$$
\left| \int_C \frac{\D z}{z^3 + 1} \right| \leq \frac \pi 3 \left( \frac{R}{R^3 - 1} \right)
$$
and deduce
$$
\lim_{R \rightarrow \infty} \int_C \frac{\D z}{z^3 + 1} = 0
$$
\textit{Solution:} \\
Similar to the previous problem, we will utilize Theorem 2.4.2 from A\&F.
We are justified in this, since the contour $C$ is continuous on the arc of the circle $|z| = R$.
This time, our arc length of the contour $C$ is
$$
L = \frac 1 6 2\pi R = \frac{\pi}{3} R.
$$
Next, we need to calculate $M$ as the upper bound for $\left| \frac{1}{z^3 + 1} \right|$.
Let's use a simpler method than in problem 3. Let's get on with it
$$
\left| f(z) \right| = \left| \frac 1 {z^3 + 1} \right| = \frac 1 {\left|z^3 + 1\right|}.
$$
Notice, if we can get a lower bound for the denominator then we will have an upper bound for the whole expression. It falls out quickly using the parameterization $z = R\E^{i\theta}$ and applying the inverse triangle inequality.
\begin{align*}
\left| z^3 + 1\right| &= \left| R^3\E^{i3\theta} + 1\right| \\
	&= \left| R^3\E^{i3\theta} - \left(-1\right)\right| \\
	&\geq \left| \left|R^3\E^{i3\theta}\right| - \left|\left(-1\right)\right|\right| \\
	&= \left| R^3 - 1\right| \\
	&= R^3 - 1
\end{align*}
where the last equality holds because $R > 1$.
Therefore, we have our lower bound for the denominator
$$
\left| z^3 + 1\right| \geq R^3 - 1,
$$
and thus an upper bound for the expression
$$
\left|f(z)\right| = \frac 1 {\left| z^3 + 1\right|} \leq \frac 1 {R^3 - 1} = M.
$$
Finally, applying Theorem 2.4.2 from A\&F we have
$$
\left| \int_C f(z) dz \right| \leq ML \leq \frac {\pi}{3} R\frac{1}{R^3 - 1} = \frac {\pi}{3} \left(\frac{R}{R^3 - 1}\right).
$$
Hence,
$$
\left| \int_C f(z) dz \right| \leq \frac {\pi}{3} \left(\frac{R}{R^3 - 1}\right).
$$
\qed \\
We will now take the limit of both sides of this inequality as $R$ goes to $\infty$
$$
\lim_{R \rightarrow \infty} \left| \int_C f(z) dz \right|
\leq \lim_{R \rightarrow \infty} \frac {\pi}{3} \left(\frac{R}{R^3 - 1}\right) = \frac \infty \infty.
$$
Applying L'Hôpital's rule once, we have
$$
\lim_{R \rightarrow \infty} \frac {\pi}{3} \left(\frac{1}{3R^2 - 1}\right) = 0.
$$
Therefore,
$$
\lim_{R \rightarrow \infty} \left| \int_C f(z) dz \right| \leq 0.
$$
If the limit of the absolute value of something is less than or equal to 0, then the limit of that thing must be zero.
This is the case because the absolute value is a non-negative function, so the ``$\leq 0 $" must be just an equality. Therefore,
$$
\lim_{R \rightarrow \infty} \left| \int_C f(z) dz \right| = 0.
$$
Note this also implies $\lim_{R \rightarrow \infty} -\left| \int_C f(z) dz \right| = 0$.
Now we have that
$$-\left| \int_C f(z) dz \right| \leq \int_C f(z) dz \leq \left| \int_C f(z) dz \right| $$
and thus by the squeeze theorem,
$$\lim_{R \rightarrow \infty} \int_C f(z) dz = 0.$$
\qed
\\

\item From A\&F: 2.5.1 b, e \\
Evaluate $\oint_C f(z)\D z$, where $C$ is the unit circle centered at the origin, and $f(z)$ is given by the following: \\
b) $$f(z) = \E^{z^2}$$ \\
\textit{Solution:} \\
Let's first break $f(z)$ up into real and imaginary parts so we can define $u(x, y)$ and $v(x, y)$ and check if the Cauchy-Riemann (C-R) equations hold.
\begin{align*}
f(z) &= \E^{z^2} \\
	&= \E^{ x^2 + 2ixy - y^2} \\
	&= \E^{x^2}\E^{2ixy}\E^{- y^2} \\
	&= \E^{x^2}\E^{- y^2}\E^{2ixy} \\
	&= \E^{x^2}\E^{- y^2}\left( \cos (2xy) + i \sin (2xy)\right) \\
	&= \E^{x^2}\E^{- y^2}\cos (2xy) + i \E^{x^2}\E^{- y^2}\sin (2xy).
\end{align*}
Then we can assign $u(x, y) = \E^{x^2}\E^{- y^2}\cos (2xy)$ and $v(x, y) = \E^{x^2}\E^{- y^2}\sin (2xy)$.
Now let's calculate the necessary derivatives to verify if $f(z)$ is analytic.
We have
\begin{align*}
\frac {\partial u}{\partial x} &= \E^{-y^2}\left( 2x\E^{x^2}\cos (2xy) - \E^{x^2} \sin (2xy)2y\right) \\
	&= \E^{x^2}\E^{-y^2}\left( 2x\cos (2xy) - \sin (2xy)2y\right)
\end{align*}
and
\begin{align*}
\frac {\partial v}{\partial y} &= \E^{x^2}\left( (-2y)\E^{-y^2}\sin(2xy) + \E^{-y^2}\cos (2xy)2x\right) \\
	&= \E^{x^2}\E^{-y^2}\left( -2y\sin(2xy) + \cos (2xy)2x\right)
\end{align*}
which are equivalent.
Additionally, we get
\begin{align*}
\frac {\partial v}{\partial x} &= \E^{- y^2} \left( 2x\E^{x^2}\sin (2xy) + \E^{x^2}\cos (2xy)2y \right) \\
	&= \E^{x^2}\E^{- y^2} \left( 2x\sin (2xy) + \cos (2xy)2y \right)
\end{align*}
and
\begin{align*}
- \frac {\partial u}{\partial y} &= -\left( \E^{x^2} \left( -2y\E^{- y^2}\cos (2xy) - \E^{- y^2}\sin (2xy)2x \right) \right) \\
	&= - \E^{x^2} \E^{- y^2} \left( -2y\cos (2xy) - \sin (2xy)2x \right) \\
	&= \E^{x^2} \E^{-y^2} \left(2y\cos (2xy) + \sin (2xy)2x \right)
\end{align*}
which are also equal as desired.
Therefore the C-R equations hold and f(z) is analytic everywhere.
Now by the Theorem 2.5.2 from A\&F or Cauchy' Theorem, we can conclude
$$
\oint_C f(z)\D z = \oint_C \E^{z^2} \D z = 0.
$$
\qed 
\\
e) $$f(z) = \frac{1}{2z^2 + 1} $$ \\
\textit{Solution:} \\
We want to calculate the following integral
$$
\oint_C f(z)\D z = \oint_C \frac{1}{2z^2 + 1} \D z.
$$
First, let's break it down using partial fractions
$$
f(z) = \frac{1}{2z^2 + 1} = \frac{-\frac1 2 i}{\sqrt{2}z - i} + \frac{\frac1 2 i}{\sqrt{2}z + i}.
$$
Now to actually calculate the integral we look at
\begin{align*}
\oint_C \frac{1}{2z^2 + 1} \D z &= \oint_C \left(\frac{-\frac1 2 i}{\sqrt{2}z - i} + \frac{\frac1 2 i}{\sqrt{2}z + i} \right)\D z \\
	&= \oint_C \frac{-\frac1 2 i}{\sqrt{2}z - i}\D z + \oint_C\frac{\frac1 2 i}{\sqrt{2}z + i} \D z.
\end{align*}
Each of these integrals is analytic in the region we care about except for at $\frac i {\sqrt{2}}$ and $-\frac i {\sqrt{2}}$ where the denominators are $0$, respectively.
However, we need to eliminate these two singularities by deforming our contour around each of them.
It is important to establish the orientation of these contours now. I am choosing that my starting contour $C$ is clockwise, hence, the ensuing contours $C_1$ and $C_2$ will be counterclockwise.
We take our contour $C$ and deform it such that we reroute it through a channel $t_1$ to circle $C_1$ of radius $\epsilon_1$ around $\frac i {\sqrt{2}} $ and back up through channel $t_2$ returning to the original contour $C$.
Similarly, we deform $C$ to go around $- \frac i {\sqrt{2}}$, by passing through the channel $t_3$ around $- \frac i {\sqrt{2}}$ via $C_2$ (a circle centered at $- \frac i {\sqrt{2}}$ with radius $\epsilon_2$) and back to $C$ through $t_4$. See Figure \ref{fig:f1} for a rough cartoon of how we deform our contour.
\\
\begin{figure}[h]
	\centering
	\includegraphics{problem5contour}
	\caption{
		I am including this visual because it helps me make sense of how to eliminate the singularities and use Cauchy's Theorem (Theorem 2.5.2 A\&F) to calculate the integral. I am acknowledging that this graphic is of a poor resolution, I am still figuring out the best ways to include images in \LaTeX. Additionally I should have labeled that the vertical axis is the imaginary ($\Im$) axis and the horizontal axis is the real ($\Re$) axis.
	}\label{fig:f1}
\end{figure}

\noindent
Once we have constructed this deformed contour such that our function $f(z)$ is analytic in a simply connected domain $D$, and along our new contour in $D$, by Cauchy's Theorem we say (I acknowledge an abuse of notation by suppressing the integrand and differential terms and putting them once at the end)
$$
\left(\oint_C + \oint_{t_1} + \oint_{C_1} + \oint_{t_2} + \oint_{t_3}  + \oint_{C_2} + \oint_{t_4}\right) f(z) \D z = 0
$$
Notice the channel contours come in opposite pairs ($t_1$, $t_2$) and ($t_3$, $t_4$) such that $t_1$ cancels $t_2$ and similar for the other pair.
Hence,
$$
\left(\oint_C + \oint_{C_1} + \oint_{C_2} \right) f(z) \D z = 0
$$
Now let's plug our actual function in here and solve for what we are looking for.
\begin{align*}
\oint_C f(z) \D z + \oint_{C_1} f(z) \D z + \oint_{C_2} f(z) \D z &= 0 \\
\oint_C \frac{1}{2z^2 + 1} \D z + \oint_{C_1} \frac{1}{2z^2 + 1} \D z + \oint_{C_2} \frac{1}{2z^2 + 1} \D z &= 0 \\
\oint_C \frac{1}{2z^2 + 1} \D z
	+ \oint_{C_1} \left( \frac{-\frac1 2 i}{\sqrt{2}z - i} + \frac{\frac1 2 i}{\sqrt{2}z + i}\right) \D z
	+ \oint_{C_2} \left( \frac{-\frac1 2 i}{\sqrt{2}z - i} + \frac{\frac1 2 i}{\sqrt{2}z + i} \right) \D z &= 0 \\
\oint_C \frac{1}{2z^2 + 1} \D z
	+ \oint_{C_1} \frac{-\frac1 2 i}{\sqrt{2}z - i}  \D z
	+ \cancel{\oint_{C_1} \frac{\frac1 2 i}{\sqrt{2}z + i} \D z}
	+ \cancel{\oint_{C_2} \frac{-\frac1 2 i}{\sqrt{2}z - i}  \D z}
	+ \oint_{C_2} \frac{\frac1 2 i}{\sqrt{2}z + i} \D z &= 0. \\
\end{align*}
Note, contour $C_1$ is centered at $\frac{i}{\sqrt{2}}$ so when integrating the half of the partial fraction result with a singularity at $- \frac{i}{\sqrt{2}}$ the function is analytic in the right regions and by Cauchy's Theorem this integral is $0$.
Similarly for $C_2$ around $-\frac{i}{\sqrt{2}}$ the function with a singularity at $\frac{i}{\sqrt{2}}$.
This explains the two integrals I crossed out in the previous step.
We have now simplified our expression to
\begin{align*}
\oint_C \frac{1}{2z^2 + 1} \D z
	+ \oint_{C_1} \frac{-\frac1 2 i}{\sqrt{2}z - i}  \D z
	+ \oint_{C_2} \frac{\frac1 2 i}{\sqrt{2}z + i} \D z &= 0 \\
\oint_C \frac{1}{2z^2 + 1} \D z
 &= - \oint_{C_1} \frac{-\frac1 2 i}{\sqrt{2}z - i}  \D z
	- \oint_{C_2} \frac{\frac1 2 i}{\sqrt{2}z + i} \D z.
\end{align*}
Recall, that our contour's $C_1$ and $C_2$ are both counterclockwise, so we can actual reorient them (where $-C_1$ and $-C_2$ are clockwise) and get
\begin{align*}
\oint_C \frac{1}{2z^2 + 1} \D z &= - \oint_{C_1} \frac{-\frac1 2 i}{\sqrt{2}z - i}  \D z
	- \oint_{C_2} \frac{\frac1 2 i}{\sqrt{2}z + i} \D z \\
\oint_C \frac{1}{2z^2 + 1} \D z &= \oint_{-C_1} \frac{-\frac1 2 i}{\sqrt{2}z - i}  \D z
	+ \oint_{-C_2} \frac{\frac1 2 i}{\sqrt{2}z + i} \D z.
\end{align*}
We finally get to just compute each of these integrals.
I will use a parameterization for each of these where $ z= r\E^{i\theta}$ where are $r$ is $\epsilon_1$ and $\epsilon_2$ for $C_1$ and $C_2$ respectively. Let's look at each remaining integral one at a time

\begin{align*}
\oint_{-C_1} \frac{-\frac1 2 i}{\sqrt{2}z - i}  \D z
	&= \int_{0}^{2\pi} \frac{-\frac1 2 i}{\sqrt{2}\:\epsilon_1\E^{i\theta} - i} i\E^{i\theta}\D \theta \\
	&= \left. -\frac1 2 i \frac{1}{\sqrt{2}\:\epsilon_1} \log\left(\sqrt{2}\:\epsilon_1\E^{i\theta} - i\right) \right|_{0}^{2\pi} \\
	&= -\frac1 2 i \frac{1}{\sqrt{2}\:\epsilon_1} \log\left(\sqrt{2}\:\epsilon_1\E^{2\pi i} - i\right)
	- \left(-\frac1 2 i \frac{1}{\sqrt{2}\:\epsilon_1} \log\left(\sqrt{2}\:\epsilon_1\E^{0} - i\right) \right) \\
	&= -\frac i {\epsilon_1 2 \sqrt{2}} \log\left(\epsilon_1\sqrt{2} - i\right)
	+ \frac i {\epsilon_1 2 \sqrt{2}} \log\left(\epsilon_1\sqrt{2} - i\right) \\
	&= 0. \\
\end{align*}
Next,
\begin{align*}
\oint_{-C_2} \frac{\frac1 2 i}{\sqrt{2}z + i}  \D z
	&= \int_{0}^{2\pi} \frac{\frac1 2 i}{\sqrt{2}\:\epsilon_2\E^{i\theta} + i} i\E^{i\theta}\D \theta \\
	&= \left. \frac1 2 i \frac{1}{\sqrt{2}\:\epsilon_2} \log\left(\sqrt{2}\:\epsilon_2\E^{i\theta} + i\right) \right|_{0}^{2\pi} \\
	&= \frac1 2 i \frac{1}{\sqrt{2}\:\epsilon_2} \log\left(\sqrt{2}\:\epsilon_2\E^{2\pi i} + i\right)
	- \frac1 2 i \frac{1}{\sqrt{2}\:\epsilon_2} \log\left(\sqrt{2}\:\epsilon_2\E^{0} + i\right)  \\
	&= -\frac i {\epsilon_2 2 \sqrt{2}} \log\left(\epsilon_2\sqrt{2} + i\right)
	+ \frac i {\epsilon_2 2 \sqrt{2}} \log\left(\epsilon_2\sqrt{2} + i\right) \\
	&= 0. \\
\end{align*}
Therefore we have found that 
$$
\oint_C \frac{1}{2z^2 + 1} \D z = \oint_{-C_1} \frac{-\frac1 2 i}{\sqrt{2}z - i}  \D z + \oint_{-C_2} \frac{\frac1 2 i}{\sqrt{2}z + i} \D z = 0 + 0 = 0
$$
\qed
\\

\item Use the ideas from A\&F: 2.5.5 to evaluate $\int_0^\infty \E^{\I
    z^3 t} \D z$, $t > 0$.  Express the result in terms of $\int_0^\infty \E^{-
    r^3} \D r$. \\
The ideas we might need to use are ... it's actually really long! \\
\textit{Solution:}\\
We want to evaluate the integral (with $t > 0$)
$$\int_0^\infty \E^{\I z^3 t} \D z.$$
Let's first do a quick change of variables to simplify the problem a bit.
Let $w = z \sqrt[3]{t}$, additionally, $\D w = \D z \sqrt[3]{t}$.
Therefore,
$$
\int_{0}^{\infty} \E^{iz^3t} \D z = \frac{1} {\sqrt[3]{t}} \int_{0}^{\infty} \E^{iw^3} \D w.
$$
It will suffice to work with the integral on the right and multiply by the scalar $\frac 1 {\sqrt[3] t}$ at the end.
Instead of this real valued integral let's consider the following
$$
\oint_{C_{(R)}} \E^{iw^3} \D w,
$$
with the contour $C_{(R)}$ being the closed circular sector (think a slice of pizza) in the upper half plane with boundary points $0$, $R$, and $R\E^{\frac{i\pi}{6}}$.
For notational convenience moving forward, I will use $f(w) = \E^{iw^3}$.
\\

\noindent
I claim that $\oint_{C_{(R)}} f(w) \D w = 0$, since our integrand is analytic everywhere (including our contour and the region inside of it) by Cauchy's Theorem the integral will be 0.
I will now verify that our function $f(w)$ is analytic.
Firstly, the composition $g(h(z))$ of analytic functions $g(z)$ and $h(z)$, is analytic so long as the range of $h(z)$ does not include any points, branch cuts or regions in which $g(z)$ is not analytic.
Secondly, it will be sufficient to show $iw^3$ is analytic since $\E^z$ is analytic everywhere.
Let's verify this using the C-R equations
We begin by separating $iw^3$ into real and imaginary parts
\begin{align*}
iw^3 &= i (x + iy)^3 \\
	&= i(x^3 + i3x^2y - 3xy^2 -iy^3) \\
	&= ix^3 - 3x^2y - i3xy^2 + y^3) \\
	&=  - 3x^2y  + y^3 - i3xy^2 + ix^3 \\
	&=  - 3x^2y  + y^3 + i\left(-3xy^2 + x^3\right).
\end{align*}
We have $$u(x, y) = - 3x^2y  + y^3 \quad \text{and} \quad v(x, y) = -3xy^2 + x^3.$$
Now we compute the necessary partials (recall we need $u_x = v_y$ and $v_x = - u_y$)
\begin{align*}
u_x &= -6xy \\
v_y &= -6xy \\
v_x &= -3y^2 + 3x^2 \\
-u_y &= -\left( -3x^2 + 3y^2 \right) = 3x^2 - 3y^2.
\end{align*}
Seeing that these partials satisfy the C-R equations we can conclude that $iw^3$ is analytic.
Therefore $\E^{iw^3}$ is also analytic and our integral, $$\oint_{C_{(R)}} \E^{iw^3} \D w = 0$$ by Cauchy's Theorem. \\

\noindent
For further convenience let's break down our contour integral $\oint_{C_{(R)}} \E^{iw^3} \D w$ into the sum of 3 contour integrals, where the 3 new contours make up our original single contour
$$
\oint_{C_{(R)}} f(w) \D w = \oint_{C_{1}} f(w) \D w + \oint_{C_{2}} f(w) \D w + \oint_{C_{3}} f(w) \D w.
$$
To be clear, the contour $C_{(R)}$ traverses the circular sector in a counterclockwise fashion.
The contour $C_1$ denotes the section of the $C_{(R)}$ contour which traverses from the point $z = R$ to $z = R\E^{\frac{i\pi}{6}}$ along the arc of the circle centered at the origin with radius $R$.
Next, the contour $C_2$ denotes the section of the $C_{(R)}$ contour which traverses from $z = R\E^{\frac{i\pi}{6}}$ to the origin along a radius of the same circle.
Finally, let the contour $C_3$ denote the section of $C_{(R)}$ which traverses along the real axis from the origin to the point $z=R$. \\

\noindent
I am now going to show that $$\lim_{R \rightarrow \infty} \oint_{C_{1}} f(w) \D w = 0.$$
I will do so by following a similar routine as done in problem 4.
First we calculate an upper bound $M$ on $\left| f(w) \right| \leq M$.
Let's first rewrite $f(w)$ a bit
$$
f(w) = \E^{iw^3} = \E^{i\left(R\E^{i\theta} \right)^3} = \E^{iR^3\E^{i3\theta}} = \E^{iR^3\left( \cos3\theta +  i\sin3\theta \right)} = \E^{iR^3\cos3\theta}\E^{ -R^3\sin3\theta} = \frac{\E^{iR^3\cos3\theta}}{\E^{R^3\sin3\theta}}.
$$
Then we have
\begin{align*}
\left|f(w)\right|
	&= \left| \frac{\E^{iR^3\cos3\theta}}{\E^{ R^3\sin3\theta}} \right| \\
	&= \frac{\left| \E^{iR^3\cos3\theta}\right| }{\left| \E^{ R^3\sin3\theta}\right| } \\
	&= \frac{\left| \cos\left(R^3\cos3\theta\right) + i\sin\left(R^3\cos3\theta\right)\right| }{\left| \E^{ R^3\sin3\theta}\right| } \\
	&= \frac{\sqrt{\cos^2\left(R^3\cos3\theta\right) + \sin^2\left(R^3\cos3\theta\right)} }{\left| \E^{ R^3\sin3\theta}\right| } \\
	&= \frac{\sqrt{1} }{\left| \E^{ R^3\sin3\theta}\right| } \\
	&= \frac{1}{\left| \E^{ R^3\sin3\theta}\right| }.
\end{align*}
Notice, the expression remaining in the modulus, $\E^{ R^3\sin3\theta}$, is a real number  since it no longer has an imaginary parts. Therefore the modulus is a like the absolute value.
Furthermore, this expression $\E$ raised to the power of some number, is a nonnegative function so we really have
$$
\left|f(w)\right| = \frac{1}{\E^{ R^3\sin3\theta}}.
$$
We are still after the upper bound for this, so let's see if we can calculate a lower bound for the denominator which will give us an upper bound for the whole expression.
Once again, observe the fact that
$\sin 3\theta \geq \frac 3 \pi \theta$ when $\theta \in [0, \frac \pi 6]$.
Then we see
\begin{align*}
\sin 3\theta &\geq \frac 3 \pi \theta \\
R^3\sin 3\theta &\geq R^3\frac 3 \pi \theta \\
\E^{R^3\sin 3\theta} &\geq \E^{R^3\frac 3 \pi \theta} \\
\end{align*}
and thus
$$
\frac 1 {\E^{R^3\sin 3\theta}} \leq \frac 1 {\E^{R^3\frac 3 \pi \theta}}.
$$
To be very explicit with how this upper bound only depends on $R$ as we will take the limit while $R\rightarrow \infty$ later, see
$$
\frac 1 {\E^{R^3\frac 3 \pi \theta}} = \frac 1 {\E^{R^3}\E^{\frac 3 \pi \theta}} \leq \frac 1 {\E^{R^3}\E^{0}} = \frac 1 {\E^{R^3}}.
$$
Therefore, the upper bound on $\left|f(w)\right|$ is $M = \frac 1 {\E^{R^3}}$.
Since we have
$$\left| f(w) \right| \leq \frac 1 {\E^{R^3}}$$
and the arc length $L$ of our contour $C_1$ is $\frac \pi 6 R$ (calculated as $\frac 1 {12} 2\pi R$), then by theorem 2.4.2, this will give us
$$
\left| \oint_{C_1} f(w) \D w\right| \leq \frac 1 {\E^{R^3}} \frac \pi 6 R = \frac {\pi R} {6\E^{R^3}} .
$$
Finally, take the limit of both sides 
\begin{align*}
\lim_{R \rightarrow \infty}\left| \oint_{C_1} f(w) \D w\right| &\leq \lim_{R \rightarrow \infty} \frac {\pi R} {6\E^{R^3}} \\
\lim_{R \rightarrow \infty}\left| \oint_{C_1} f(w) \D w\right| &\leq \lim_{R \rightarrow \infty} \frac {\pi} {18R^2\E^{R^3}} \quad \text{applying L'Hôpital's rule} \\
\lim_{R \rightarrow \infty}\left| \oint_{C_1} f(w) \D w\right| &\leq 0
\end{align*}
which should be an equality since the modulus is a nonnegative function.
Following this same logic from problem 4, the squeeze theorem gives us that
$$
\lim_{R \rightarrow \infty} \oint_{C_1} f(w) \D w = 0.
$$
Since we have
\begin{align*}
\oint_{C_{(R)}} f(w) \D w &= 0 \\
\oint_{C_{1}} f(w) \D w + \oint_{C_{2}} f(w) \D w + \oint_{C_{3}} f(w) \D w &=0
\end{align*}
Now taking the limit as $R\rightarrow\infty$
\begin{align}
\lim_{R\rightarrow\infty}\left(\cancelto{0}{\oint_{C_{1}} f(w) \D w} + \oint_{C_{2}} f(w) \D w + \oint_{C_{3}} f(w) \D w \right) &=0 \nonumber \\
\lim_{R\rightarrow\infty}\left(\oint_{C_{2}} f(w) \D w + \oint_{C_{3}} f(w) \D w \right) &=0.
\label{eq:last}
\end{align}
Let's carefully parameterize the two remaining contour integrals. The contour $C_3$ along the real axis from $0$ to $R$ is essentially already a real valued integral letting $w = x$ we have
$$
\oint_{C_{3}} f(w) \D w = \oint_{C_{3}} \E^{iw^3} \D w = \int_{0}^{R} \E^{ix^3} \D x.
$$
Next the $C_2$ contour integral will be parameterized with $w = r\E^{\frac{i\pi}{6}}$
\begin{align*}
\oint_{C_{2}} f(w) \D w
	&= \oint_{C_{2}} \E^{iw^3} \D w \\
	&= \int_{R}^{0} \E^{i\left(r\E^{\frac{i\pi}{6}}\right)^3} \E^{\frac{i\pi}{6}} \D r \\
	&= \int_{R}^{0} \E^{ir^3\E^{\frac{i\pi}{2}}} \E^{\frac{i\pi}{6}} \D r \\
	&= \E^{\frac{i\pi}{6}} \int_{R}^{0} \E^{i^2r^3} \D r \\
	&= \E^{\frac{i\pi}{6}} \int_{R}^{0} \E^{-r^3} \D r \\
	&= - \E^{\frac{i\pi}{6}} \int_{0}^{R} \E^{-r^3} \D r.
\end{align*}
Therefore equation \eqref{eq:last} becomes
\begin{align*}
& \lim_{R\rightarrow\infty} \left(
	\oint_{C_{2}} f(w) \D w + \oint_{C_{3}} f(w) \D w
\right) \\
	= & \lim_{R\rightarrow\infty}  \left(
	- \E^{\frac{i\pi}{6}} \int_{0}^{R} \E^{-r^3} \D r + \int_{0}^{R} \E^{ix^3} \D x
\right) \\
= & \lim_{R\rightarrow\infty}  \left(
	\int_{0}^{R} \E^{ix^3} \D x
	- \E^{\frac{i\pi}{6}} \int_{0}^{R} \E^{-r^3} \D r 
\right) = 0.
\end{align*}
Hence,
$$
\int_{0}^{\infty} \E^{ix^3} \D x =  \E^{\frac{i\pi}{6}} \int_{0}^{\infty} \E^{-r^3} \D r.
$$
And multiplying both sides by our scale factor from the substitution at the beginning we have
\begin{align*}
\int_{0}^{\infty} \E^{iz^3t} \D z &= \frac{1} {\sqrt[3]{t}} \int_{0}^{\infty} \E^{iw^3} \D w \\
	&=  \E^{\frac{i\pi}{6}} \frac{1} {\sqrt[3]{t}}  \int_{0}^{\infty} \E^{-r^3} \D r
\end{align*}
as required.
\qed
\\

\item From A\&F: 2.5.6. \\
This is the exercise from the book: \\
``Consider the integral $$I = \int_{-\infty}^{\infty} \frac{\D x}{x^2 + 1}.$$
Show how to evaluate this integral by considering
$$\oint_{C_{(R)}} \frac{\D z}{z^2 + 1},$$
where $C_{(R)}$ is closed semicircle in the upper half plane with endpoints at $(-R, 0)$ and $(R, 0)$ plus the $x$-axis.
\textit{Hint:} use
$$\frac{1}{z^2 + 1} = -\frac{1}{2i}\left(\frac{1}{z + i} - \frac{1}{z - i}\right),$$
and show that the integral along the open semicircle in the upper half plane vanishes as $R \rightarrow \infty$.
Verify your answer by usual integration in real variables." \\

\noindent
Do this this exercise once, for the following integral (a general form of the specific case given in this book exercise)
  \begin{align*}
    I_\epsilon = \int_{-\infty}^\infty \frac{\epsilon \D x}{x^2 +
    \epsilon^2}, \quad \epsilon > 0.
  \end{align*}\\
\textit{Solution:} \\
Let's consider the complex valued contour integral
$$\oint_{C_{(R)}} \frac{\epsilon}{z^2 + \epsilon^2} \D z,$$
where $C_{(R)}$ is closed semicircle in the upper half plane with endpoints at $(-R, 0)$ and $(R, 0)$ plus the $x$-axis.
Before I break apart our function $f(z)$ using partial fractions, I want to analyze this integral a little more.
The interior of our contour $C_{(R)}$ contains one singularity of our function $f(z)$ at $z = i\epsilon$ (the other singularity $-i\epsilon$ is outside of our contour so we don't worry about it).
By Cauchy's theorem we know
$$
\oint_{C_{(R)}} f(z) \D z = \oint_{C_{(i\epsilon)}} f(z) \D z
$$
where the contour $C_{(i\epsilon)}$ is a counterclockwise traversal of the circle centered at $z = i\epsilon$ with radius little $r$.
Additionally, observe that we can break the contour $C_{(R)}$ into two pieces, $C_1$, traversing an open (upper) semicircle of radius $R$ with its center at the origin, and $C_2$ traversing directly across the real axis from $-R$ to $R$. We now have
\begin{align}
\oint_{C_{(R)}} f(z) \D z &= \oint_{C_{(i\epsilon)}} f(z) \D z \nonumber \\
\oint_{C_1} f(z) \D z + \oint_{C_2} f(z) \D z &= \oint_{C_{(i\epsilon)}} f(z) \D zs
\label{eq:step1}
\end{align}
Before digging in and calculating any of these integrals let's analyze this for a moment.
Notice, the integral $\oint_{C_2} f(z) \D z$ is similar to the integral we found a bound for in Problem 3.
Siting our results from the same problem we can find the upper bound is
$$\left| \oint_{C_2} f(z) \D z\right| \leq \frac{\epsilon \pi R}{R^2 - \epsilon^2}$$
and it goes to $0$ as $R \rightarrow \infty$.
Now siting the latter section of Problem 4, where we used the squeeze theorem to determine the value of the integral, we can also conclude that
$$
\oint_{C_2} f(z) \D z = 0.
$$
Putting this together in equation \eqref{eq:step1}, we see
\begin{align}
\oint_{C_1} f(z) \D z + \oint_{C_2} f(z) \D z &= \oint_{C_{(i\epsilon)}} f(z) \D z \nonumber \\
\oint_{C_1} f(z) \D z + 0 &= \oint_{C_{(i\epsilon)}} f(z) \D z \nonumber \\
\oint_{C_1} f(z) \D z &= \oint_{C_{(i\epsilon)}} f(z) \D z \nonumber \\
\int_{-R}^R \frac{\epsilon}{z^2 + \epsilon^2} \D z &= \oint_{C_{(i\epsilon)}} f(z) \D z.
\label{eq:step2}
\end{align}
Essentially, we have reduced our problem to a real valued integral on the left from $-R$ to $R$ (looking suspiciously similar to the integral we are supposed to compute from the beginning) and a single contour integral on the right.
Let's calculate the contour integral on the right, next.
Our first step is to break apart our function $f(z)$ using partial fractions
\begin{align*}
\oint_{C_{(i\epsilon)}} \frac{\epsilon}{z^2 + \epsilon^2} \D z &= \oint_{C_{(i\epsilon)}} \left(
		\frac{-\frac 1 2 i}{z - i\epsilon} + \frac{\frac 1 2 i}{z + i\epsilon}
	\right) \D z \\
	&= \oint_{C_{(i\epsilon)}} \frac{-\frac 1 2 i}{z - i\epsilon} \D z
	+ \oint_{C_{(i\epsilon)}} \frac{\frac 1 2 i}{z + i\epsilon} \D z
\end{align*}
Remember our contour $C_{(i\epsilon)}$ is centered at $z =i\epsilon$ so it does not hit the singularity ($-i\epsilon$) of the function in the integral on the right.
Then by Cauchy's theorem,
$$\oint_{C_{(i\epsilon)}} \frac{\frac 1 2 i}{z + i\epsilon} \D z = 0.$$
Let's use the parameterization $z = r\E^{i\theta} + i\epsilon$, where $r<\epsilon$ is the radius of our circle centered at $i\epsilon$ to calculate the other integral here
\begin{align*}
\oint_{C_{(i\epsilon)}} \frac{-\frac 1 2 i}{z - i\epsilon} \D z
	&= \int_0^{2\pi} \frac{-\frac 1 2 i}{r\E^{i\theta}} ir\E^{i\theta}\D \theta \\
	&= \int_0^{2\pi} \frac{-\frac 1 2 i\left(ir\E^{i\theta}\right)}{r\E^{i\theta}} \D \theta \\
	&= \int_0^{2\pi} -\frac1 2 i^2 \D \theta \\
	&= \frac1 2 \int_0^{2\pi} \D \theta \\
	&= \frac1 2 \left( \left. \theta \right|_0^{2\pi}\right) \\
	&= \frac1 2 \left( 2\pi - 0 \right) \\
	&= \pi
\end{align*}
Returning to equation \eqref{eq:step2}, we see
\begin{align}
\int_{-R}^R \frac{\epsilon}{z^2 + \epsilon^2} \D z &= \oint_{C_{(i\epsilon)}} f(z) \D z \nonumber \\
\int_{-R}^R \frac{\epsilon}{z^2 + \epsilon^2} \D z &= \oint_{C_{(i\epsilon)}} \frac{-\frac 1 2 i}{z - i\epsilon} \D z
	+ \oint_{C_{(i\epsilon)}} \frac{\frac 1 2 i}{z + i\epsilon} \D z \nonumber \\
\int_{-R}^R \frac{\epsilon}{z^2 + \epsilon^2} \D z	 &= \pi + 0 \nonumber \\
\int_{-R}^R \frac{\epsilon}{z^2 + \epsilon^2} \D z	 &= \pi.
\label{eq:step3}
\end{align}
Recall earlier we have actually already taken $R\rightarrow\infty$ and thus equation \eqref{eq:step3} actually becomes
$$
\int_{-\infty}^\infty \frac{\epsilon}{z^2 + \epsilon^2} \D z = \pi.
$$
Let's verify if this is indeed the same as our original real valued integral
$I_\epsilon = \int_{-\infty}^\infty \frac{\epsilon \D x}{x^2 + \epsilon^2}$, $\epsilon > 0.$
We will use a $u$ substitution, namely $u = \frac x \epsilon$ and thus $\D u = \frac 1 \epsilon \D x$.
Followed by a trig substitution, $u = \tan \theta$ and $\D u = \sec^2\theta \D \theta$. Here we go,
\begin{align*}
\int_{-\infty}^\infty \frac{\epsilon}{x^2 + \epsilon^2}  \D x
	&= \int_{-\infty}^\infty \frac{\epsilon}{(\epsilon u)^2 + \epsilon^2} \epsilon \D u \\
	&= \int_{-\infty}^\infty \frac{\epsilon^2}{\epsilon^2u^2 + \epsilon^2} \D u \\
	&= \int_{-\infty}^\infty \frac{1}{u^2 + 1} \D u \\
	&= \int_{-\infty}^\infty \frac{1}{u^2 + 1} \D u \\
	&= \int_{-\frac \pi 2}^{\frac \pi 2} \frac{\sec^2\theta}{\tan^2\theta + 1} \D \theta \\
	&= \int_{-\frac \pi 2}^{\frac \pi 2} \frac{\sec^2\theta}{\sec^2\theta} \D \theta \\
	&= \int_{-\frac \pi 2}^{\frac \pi 2} \D \theta \\
	&= \left. \theta \right|_{-\frac \pi 2}^{\frac \pi 2}\\
	&= \frac \pi 2 - \left(-\frac \pi 2\right) \\
	&= \pi.
\end{align*}
And thus we see, in this case it was equivalent to think about this real valued integral as the contour integral proposed at the beginning.
\qed
\\

\item Use a similar method to calculate
  $\int_{-\infty}^{\infty} \frac{d x}{1+x^4}$. \\
\textit{Solution:}\\

\noindent
Let's consider the complex valued contour integral 
$$
\oint_{C_{(R)}} \frac{1}{z^4 + 1} \D z.
$$
Let $C_{(R)}$ be the contour which is the closed semicircle in the upper half plane made up of two contours $C_1$ and $C_2$.
Define $C_1$ to be a straight line along the real axis from $-R$ to $R$ and $C_2$ being the path along the semicircle with radius $R$ moving along values of theta from $0$ to $\pi$.
To begin, I will use the formula for partial fractions from the problem 7 hint,
$
\frac{1}{z^2 + 1} = -\frac{1}{2i}\left(\frac{1}{z + i} - \frac{1}{z - i}\right).
$
Therefore
$$
\frac{1}{z^4 + 1} = -\frac{1}{2i}\left(\frac{1}{z^2 + i} - \frac{1}{z^2 - i}\right).
$$
Now we have
\begin{align*}
 \frac{1}{z^2 + i} &= -\frac{1}{2\sqrt{-i}}\left( \frac{1}{z + \sqrt{-i}} - \frac{1}{z - \sqrt{-i}} \right) \\
	&= -\frac{1}{2i\sqrt{i}}\left( \frac{1}{z + i\sqrt{i}} - \frac{1}{z - i\sqrt{i}} \right).
\end{align*}
and
$$
\frac{1}{z^2 - i} = -\frac{1}{2\sqrt{i}}\left( \frac{1}{z + \sqrt{i}} - \frac{1}{z - \sqrt{i}} \right).
$$
Thus we can break the function $f(z)$ down as follows
\begin{align*}
f(z) = \frac{1}{z^4 + 1} &= -\frac{1}{2i}
	\left[
		\left( -\frac{1}{2i\sqrt{i}}\left( \frac{1}{z + i\sqrt{i}} - \frac{1}{z - i\sqrt{i}} \right) \right)
		- \left( -\frac{1}{2\sqrt{i}}\left( \frac{1}{z + \sqrt{i}} - \frac{1}{z - \sqrt{i}} \right) \right)
	\right] \\ \\
	&= -\frac{1}{2i}
	\left[
		-\frac{1}{2i\sqrt{i}}\left( \frac{1}{z + i\sqrt{i}} - \frac{1}{z - i\sqrt{i}} \right)
		+ \frac{1}{2\sqrt{i}}\left( \frac{1}{z + \sqrt{i}} - \frac{1}{z - \sqrt{i}} \right)
	\right] \\ \\
	&= \left[
		\frac{1}{4i^2\sqrt{i}}\left( \frac{1}{z + i\sqrt{i}} - \frac{1}{z - i\sqrt{i}} \right)
		- \frac{1}{4i\sqrt{i}}\left( \frac{1}{z + \sqrt{i}} - \frac{1}{z - \sqrt{i}} \right)
	\right] \\ \\
	&= \left[
		- \frac{1}{4\sqrt{i}}\left( \frac{1}{z + i\sqrt{i}} - \frac{1}{z - i\sqrt{i}} \right)
		- \frac{1}{4i\sqrt{i}}\left( \frac{1}{z + \sqrt{i}} - \frac{1}{z - \sqrt{i}} \right)
	\right] \\ \\
	&= - \frac{1}{4\sqrt{i}} \left( \frac{1}{z + i\sqrt{i}}\right)
		+ \frac{1}{4\sqrt{i}} \left(\frac{1}{z - i\sqrt{i}} \right)
		- \frac{1}{4i\sqrt{i}} \left( \frac{1}{z + \sqrt{i}} \right)
		+ \frac{1}{4i\sqrt{i}} \left( \frac{1}{z - \sqrt{i}} \right). \\ \\
\end{align*}
Let's add a bit of notation here for convenience. Let $f(z)$ be
$$
f(z) = f_{(-i\sqrt{i})}(z) + f_{(i\sqrt{i})}(z) + f_{(-\sqrt{i})}(z) + f_{(\sqrt{i})}(z)
$$
where each $f_{(z_j)}(z)$ is the fraction with a pole at $z_j$. Specifically, they are given by
\begin{align*}
f_{(-i\sqrt{i})}(z) &= - \frac{1}{4\sqrt{i}} \left( \frac{1}{z + i\sqrt{i}}\right) \\
f_{(i\sqrt{i})}(z) &= \frac{1}{4\sqrt{i}} \left(\frac{1}{z - i\sqrt{i}} \right) \\
f_{(-\sqrt{i})}(z) &= - \frac{1}{4i\sqrt{i}} \left( \frac{1}{z + \sqrt{i}} \right) \\
f_{(\sqrt{i})}(z) &= \frac{1}{4i\sqrt{i}} \left( \frac{1}{z - \sqrt{i}} \right). \\ \\
\end{align*}
Now that we have broken down our function $f(z)$ in a helpful manner, let's analyze how we are going to evaluate the integral $\oint_{C_{(R)}} f(z) \D z$.
Following a similar procedure to problem 7, we deform this contour $C_{(R)}$ to cut out the branch points or singularities and combine this with Cauchy's theorem we have
\begin{align*}
\oint_{C_{(R)}} f(z) \D z + \oint_{-C_{(\sqrt{i})}} f(z) \D z + \oint_{-C_{(i\sqrt{i})}} f(z) \D z &= 0 \\
\oint_{C_{(R)}} f(z) \D z &= \oint_{C_{(\sqrt{i})}} f(z) \D z + \oint_{C_{(i\sqrt{i})}} f(z) \D z.
\end{align*}
\\
Where the contours $C_{(\sqrt{i})}$ and $C_{(i\sqrt{i})}$ represent counterclockwise circles of radius $r$ sufficiently small centered at each of the two singularities of $f(z)$ found on the interior of our contour $C$.
Drawing on the results which we have shown several of times on several problems at this point, we can conclude that as $R\rightarrow \infty$ the contour integral across the semicircle contour $C_2$ is going to go to 0.
We write this as
\begin{align*}
\oint_{C_{(R)}} f(z) \D z &= \oint_{C_{(\sqrt{i})}} f(z) \D z + \oint_{C_{(i\sqrt{i})}} f(z) \D z \\
\oint_{C_1} f(z) \D z + \oint_{C_2} f(z) \D z &= \oint_{C_{(\sqrt{i})}} f(z) \D z + \oint_{C_{(i\sqrt{i})}} f(z) \D z \\
\lim_{R\rightarrow\infty} \left( \oint_{C_1} f(z) \D z + \cancelto{0}{\oint_{C_2} f(z) \D z} \right)
	&= \lim_{R\rightarrow\infty} \left( \oint_{C_{(\sqrt{i})}} f(z) \D z + \oint_{C_{(i\sqrt{i})}} f(z) \D z \right) \\
\int_{-\infty}^{\infty} f(z) \D z
	&= \oint_{C_{(\sqrt{i})}} f(z) \D z + \oint_{C_{(i\sqrt{i})}} f(z) \D z. \\
\end{align*}
This final line also uses the fact that the integrals on the right hand side do not depend on $R$.
Notice the integral on the left is now just the real valued integral we started with that we want to evaluate.
Let's compute the two remaining contour integrals on the right.
We make use of the fact that our partial fraction decomposition allows us to break each of these contour integrals into several pieces. Additionally, notice that if our contour is around $z=i\sqrt{i}$ but the integrand is one of the parts of $f(z)$ which has no singularity inside that contour then by Cauchy's Theorem that particular integral is 0.
\begin{align*}
\oint_{C_{(\sqrt{i})}} f(z) \D z &= \oint_{C_{(\sqrt{i})}} \left(
	f_{(-\sqrt{i})}(z) + f_{(\sqrt{i})}(z) + f_{(-i\sqrt{i})}(z) + f_{(i\sqrt{i})}(z)
	\right) \D z \\
	&= \cancel{\oint_{C_{(\sqrt{i})}} f_{(-\sqrt{i})}(z) \D z}
	+ \oint_{C_{(\sqrt{i})}} f_{(\sqrt{i})}(z) \D z
	+ \cancel{\oint_{C_{(\sqrt{i})}} f_{(-i\sqrt{i})}(z) \D z}
	+ \cancel{\oint_{C_{(\sqrt{i})}} f_{(i\sqrt{i})}(z) \D z} \\
	&= \oint_{C_{(\sqrt{i})}} f_{(\sqrt{i})}(z) \D z.
\end{align*}
If $z_k$ is a singularity of $f(z)$ and $z_j$ is a different singularity of $f(z)$, we don't have to worry about computing the contour integral of the $f_{(z_k)}(z)$ if the contour is centered at $z_j$ since $z_k$ is not a singularity of the function $f_{(z_k)}(z)$.
Therefore $f_{(z_k)}(z)$ is analytic in the region enclosed by the circle contour centered at $z_j$ and therefore the contour is 0 by Cauchy's theorem.
Similarly we can say 
\begin{align*}
\oint_{C_{(i\sqrt{i})}} f(z) \D z &= \oint_{C_{(i\sqrt{i})}} f_{(i\sqrt{i})}(z) \D z.
\end{align*}
Hence we can compute the whole thing as
$$
\int_{-\infty}^{\infty} f(z) \D z = \oint_{C_{(i\sqrt{i})}} f_{(i\sqrt{i})}(z) \D z + \oint_{C_{(\sqrt{i})}} f_{(\sqrt{i})}(z) \D z.
$$
Let's calculate them one at a time so we can get the right substitutions going.
First we will follow the contour of a circle with radius $r$ centered at $z = \sqrt{i}$, therefore $z = r\E^{i\theta} - \sqrt{i})$, here we go
\begin{align*}
\oint_{C_{(\sqrt{i})}} f_{(\sqrt{i})}(z) \D z
	&= \oint_{C_{(\sqrt{i})}} \frac{1}{4i\sqrt{i}} \left( \frac{1}{z - \sqrt{i}} \right) \D z \\
	&= \int_0^{2\pi} \frac{1}{4\cancel{i}\sqrt{i}}\left(\frac{\cancel{i}\cancel{r\E^{i\theta}}}{\cancel{r\E^{i\theta}}} \right) \D \theta \\
	&= \frac{1}{4\sqrt{i}} \int_0^{2\pi} \D \theta \\
	&= \frac{2 \pi}{4\sqrt{i}} \\
	&= \frac{\pi}{2\sqrt{i}}. \\
\end{align*}
Next we will follow the contour of a circle with radius $r$ centered at $z = i\sqrt{i}$, therefore $z = r\E^{i\theta} + i\sqrt{i}$, here we go
\begin{align*}
\oint_{C_{(i\sqrt{i})}} f_{(i\sqrt{i})}(z) \D z &= \oint_{C_{(i\sqrt{i})}} \frac{1}{4\sqrt{i}} \left(\frac{1}{z - i\sqrt{i}} \right) \D z \\
	&= \int_0^{2\pi} \frac{1}{4\sqrt{i}}\left(\frac{i\cancel{r\E^{i\theta}}}{\cancel{r\E^{i\theta}}} \right) \D \theta \\
	&= \frac{i}{4\sqrt{i}} \int_0^{2\pi}  \D \theta \\
	&= \frac{2\pi i}{4\sqrt{i}} \\
	&= \frac{\pi i}{2\sqrt{i}} \frac {\sqrt{i}} {\sqrt{i}} \\
	&= \frac{\pi \sqrt{i}}{2}. \\
\end{align*}

Then we have,
\begin{align*}
\int_{-\infty}^{\infty} f(z) \D z
	&= \oint_{C_{(i\sqrt{i})}} f_{(i\sqrt{i})}(z) \D z + \oint_{C_{(\sqrt{i})}} f_{(\sqrt{i})}(z) \D z \\
	&= \frac{\pi \sqrt{i}}{2} + \frac{\pi}{2\sqrt{i}} \\
	&= \frac{\pi}{2}\left(\sqrt{i} + \frac{1}{\sqrt{i}}\right) \\
	&= \frac{\pi}{2}\left(\E^{i\frac{\pi}{4}} + \frac{1}{\sqrt{i}} \left(\frac{i\sqrt{i}}{i\sqrt{i}}\right)\right) \\
	&= \frac{\pi}{2}\left(\E^{i\frac{\pi}{4}} + (-1) i\sqrt{i}\right) \\
	&= \frac{\pi}{2}\left(\E^{i\frac{\pi}{4}} + \E^{-i\pi} \E^{i \frac \pi 2} \E^{i \frac \pi 4}\right) \\
	&= \frac{\pi}{2}\left(\E^{i\frac{\pi}{4}} + \E^{i\left(-\pi + \frac \pi 2 + \frac \pi 4\right)}\right) \\
	&= \frac{\pi}{2}\left(\E^{i\frac{\pi}{4}} + \E^{-i \frac \pi 4}\right) \\
	&= \frac{\pi}{2}\left(\frac 1 {\sqrt{2}} + \cancel{ i\frac 1 {\sqrt{2}}} +  \frac 1 {\sqrt{2}} - \cancel{ i\frac 1 {\sqrt{2}}}\right) \\
	&= \frac{\pi}{2}\left(\frac 2 {\sqrt{2}}\right) \\
	&= \frac{\pi}{\sqrt{2}}
\end{align*}
And thus
$$
\int_{-\infty}^{\infty} f(z) \D z = \int_{-\infty}^{\infty} \frac 1 {z^4 + 1} \D z = \frac{\pi}{\sqrt{2}}
$$
which matches our expected results from real valued integration of $\int_{-\infty}^{\infty} \frac 1 {x^4 + 1} \D x$.
\qed
\\

\item From A\&F: 2.6.1 a, e.\\
Evaluate the integrals $\oint_C f(z) \D z$, where $C$ is the unit circle centered at the origin and $f(z)$ is given by the following (use Eq. (1.2.19) as necessary): \\
a)
$$
\frac{\sin z}{z}
$$
\\
\textit{Solution:}\\
Notice $\sin(z)$ is analytic everywhere, then it satisfies our criteria to be analytic inside our contour $C$ (which is the unit circle centered at the origin).
Therefore, we use the Cauchy Integral Formula which is
$$
f^{(k)}(z) = \frac{k!}{2\pi i} \oint_C \frac{f(\xi)}{\left(\xi - z\right)^{k + 1}} \D \xi.
$$
We want to evaluate the integral 
$$
\oint_C \frac{\sin z}{z} \D z.
$$
Notice we can apply the Cauchy Integral Formula here to help simplify this task.
Observe,
\begin{align*}
f^{(0)}(0) &= \frac{0!}{2\pi i} \oint_C \frac{\sin z}{(z)^{0 + 1}} \D z \\
2\pi i f(0) &= \oint_C \frac{\sin z}{z} \D z \\
2\pi i \sin(0) &= \oint_C \frac{\sin z}{z} \D z \\
0 &= \oint_C \frac{\sin z}{z} \D z. \\
\end{align*}
Therefore 
$$
\oint_C \frac{\sin z}{z} \D z = 0.
$$
\qed
\\

e)
$$
\E^{z^2}\left(\frac{1}{z^2} - \frac{1}{z^3}\right)
$$
\\
\textit{Solution:}\\
Notice $\E^{z^2}$ is analytic everywhere, then it satisfies our criteria to be analytic inside our contour $C$ (which is the unit circle centered at the origin).
Therefore, we use the Cauchy Integral formula again, for each half of this function.
We write
$$
\oint_C \E^{z^2}\left(\frac{1}{z^2} - \frac{1}{z^3}\right) \D z
	= \oint_C \frac{\E^{z^2}}{z^2}\D z - \oint_C \frac{\E^{z^2}}{z^3} \D z.
$$
Now we can calculate each of these integrals separately.
Before we do so, note the following first and second derivatives of the function $f(z) = \E^{z^2}$
$$f^{(1)}(z) = 2z\E^{z^2} \quad f^{(2)}(z) = 2\E^{z^2} + 4z^2\E^{z^2}.$$
Beginning with $\oint_C \frac{\E^{z^2}}{z^2}\D z$ we see
\begin{align*}
f^{(1)}(0) = \frac{1!}{2\pi i} \oint_C \frac{\E^{z^2}}{z^2}\D z \\
\frac{2\pi i}{1!} f^{(1)}(0) = \oint_C \frac{\E^{z^2}}{z^2}\D z \\
2\pi i f^{(1)}(0) = \oint_C \frac{\E^{z^2}}{z^2}\D z \\
2\pi i\left( 2(0)\E^{(0)^2}\right) = \oint_C \frac{\E^{z^2}}{z^2}\D z \\
0 = \oint_C \frac{\E^{z^2}}{z^2}\D z.
\end{align*}
And now for $\oint_C \frac{\E^{z^2}}{z^3} \D z$ we have
\begin{align*}
f^{(2)}(0) = \frac{2!}{2\pi i} \oint_C \frac{\E^{z^2}}{z^3}\D z \\
\frac{2\pi i}{2!} f^{(2)}(0) = \oint_C \frac{\E^{z^2}}{z^3}\D z \\
\pi i f^{(2)}(0) = \oint_C \frac{\E^{z^2}}{z^3}\D z \\
\pi i \left(2\E^{(0)^2} + 4(0)^2\E^{(0)^2}\right) = \oint_C \frac{\E^{z^2}}{z^2}\D z \\
\pi i 2 = \oint_C \frac{\E^{z^2}}{z^2}\D z \\
2 \pi i = \oint_C \frac{\E^{z^2}}{z^2}\D z.
\end{align*}
Therefore
$$
\oint_C \E^{z^2}\left(\frac{1}{z^2} - \frac{1}{z^3}\right) \D z
	= \oint_C \frac{\E^{z^2}}{z^2}\D z - \oint_C \frac{\E^{z^2}}{z^3} \D z
	= 0 - 2 \pi i = - 2 \pi i.
$$
\qed
\end{enumerate}

\end{document}

%%% Local Variables:
%%% mode: latex
%%% TeX-master: t
%%% End:
