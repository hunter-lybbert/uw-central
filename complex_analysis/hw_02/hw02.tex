\documentclass[10pt]{amsart}
\usepackage[margin=1.4in]{geometry}
\usepackage[usenames,dvipsnames,cmyk]{xcolor} %load first

\usepackage{amssymb,amsmath,enumitem,url}

\newcommand{\D}{\mathrm{d}}
\newcommand{\I}{\mathrm{i}}
\DeclareMathOperator{\E}{e}
\DeclareMathOperator{\OO}{O}
\DeclareMathOperator{\oo}{o}
\DeclareMathOperator{\erfc}{erfc}
\DeclareMathOperator{\real}{Re}
\DeclareMathOperator{\imag}{Im}
\usepackage{tikz}
\usepackage[framemethod=tikz]{mdframed}
\theoremstyle{nonumberplain}

\mdtheorem[innertopmargin=-5pt]{sol}{Solution}
%\newmdtheoremenv[innertopmargin=-5pt]{sol}{Solution}
\definecolor{MichiganBlue}{HTML}{00274C}
\definecolor{MichiganYellow}{HTML}{FFCB05}  
\definecolor{NicePurple}{RGB}{75,56,76} %PrincePurple
\definecolor{NiceRed}{RGB}{230,37,52}
\definecolor{MidnightBlue}{rgb}{0.1, 0.1, 0.44}
\usepackage[colorlinks=true, linkcolor=MidnightBlue, citecolor=MidnightBlue, urlcolor=MidnightBlue]{hyperref}

\begin{document}
\pagestyle{empty}

\newcommand{\mline}{\vspace{.2in}\hrule\vspace{.2in}}

\noindent
\text{Hunter Lybbert} \\
\text{Student ID: 2426454} \\
\text{10-07-24} \\
\text{AMATH 567} \\

\title{\bf { Homework 2 } }


\maketitle
\noindent
Collaborators*: TBD \\
\\
\tiny
\text{*Listed in no particular order. And anyone I discussed at least part of one problem with is considered a colaborator.}
\normalsize

\mline
\begin{enumerate}[label={\bf {\arabic*}:}]
\item From A\&F: 1.2.12. \\
Show that a circle in the $z$ plane corresponds to a circle on the sphere.
(Note the remark following the reference to Figure 1.2.7 in Section 1.2.2)\\
Hints from Office Hours: \\
\begin{itemize}
\item Use 1.2.25 equations
\item perhaps start with a circle centered at the origin for the intuition
\item show the intersection of the plane with a sphere is a circle \\
\end{itemize}
\textit{Solution:} \\
\textbf{Incomplete} \\

\item From A\&F: 1.3.5.\\
Show that the functions $\Re(z)$ and $\Im(z)$ are nowhere differentiable. \\
\textit{Solution:} \\
\textbf{Part 1:} $f(z) = \Re(z)$ \\
Let's begin with the definition of the derivative 
\begin{eqnarray*}
\lim_{h \rightarrow 0} \frac{f(z + h) - f(z)}{h} &=& \lim_{h \rightarrow 0} \frac{f(x_z + x_h + i (y_z + y_h) ) - x_z}{h} \\
								 &=& \lim_{h \rightarrow 0} \frac{x_z + x_h - x_z}{x_h + iy_h} \\
								 &=& \lim_{h \rightarrow 0} \frac{x_h}{x_h + iy_h} = \frac{0}{0}.						
\end{eqnarray*}
Now we want to use L'Hôpital's rule so we create two cases fixing $y_h$ and $x_h$ each in turn.
Starting from where we left off
$$ \lim_{h \rightarrow 0} \frac{x_h}{x_h + iy_h} = \lim_{h \rightarrow 0} \frac{ \frac{d}{dx_h}x_h}{\frac{d}{dx_h}(x_h + iy_h)} = \lim_{h \rightarrow 0} \frac{1}{1} = 1 \: \text{when $y_h$ is fixed} $$
$$ \lim_{h \rightarrow 0} \frac{x_h}{x_h + iy_h} = \lim_{h \rightarrow 0} \frac{ \frac{d}{dy_h}x_h}{\frac{d}{dy_h}(x_h + iy_h)} = \lim_{h \rightarrow 0} \frac{0}{i} = 0 \: \text{when $x_h$ is fixed}. $$
Since we get a different result at an arbitrary $z$ depending on which $x_h$ and $y_h$ is currently fixed the limit does not exist and therefore $f(z) = \Re(z)$ is nowhere differentiable. \\
\qed

\noindent
\textbf{Part 2:} $f(z) = \Im(z)$ \\
For this one we will use a different method. Recall that the Cauchy-Riemann equations are a necessary condition that must hold if $f(z)$ is differentiable (A\&F pg. 33). Therefore we can show that the Cauchy-Riemann equations do not hold and therefore the function is non differentiable. We use the fact that $f(z) = u(x, y) + i v(x, y)$ and 
$$f(z) = \Im(z) = y \quad\text{since $z = x +iy$}$$
to get 
\begin{eqnarray*}
u(x, y) &=& y \\
v(x, y) &=& 0.
\end{eqnarray*}
Now we want to check if both of the following hold
$$u_x = v_y$$
$$v_x = - u_y.$$
Let's calculate them
$$u_x = 0, v_y = 0, v_x = 0, u_y = 1$$
Therefore the first condition holds
\begin{eqnarray*}
u_x &=& v_y \\
0 &=& 0 \\
\end{eqnarray*}
however, the second does not
\begin{eqnarray*}
v_x = - u_y \\
0 \neq - 1.
\end{eqnarray*}
Therefore $f(z) = \Im(z)$ is nowhere differentiable. \\
\qed

\item Consider the function
    \begin{align*}
      \varphi(z) = z + \sqrt{z^2 - 1}, \quad z > 1.
    \end{align*}
    Show that
    \begin{align*}
      \log \varphi(z) = \int_1^z \frac{\D x}{\sqrt{x^2 - 1}}.
    \end{align*}
Office Hour Hints:
\begin{itemize}
\item $z$ is real!
\end{itemize}
\textit{Solution:} \\
\textbf{Incomplete} \\
\item Find all zeroes of $\tan (z), z \in \mathbb{C}$. What can you
  conclude about the zeroes of $\tanh (z)=\sinh (z) / \cosh (z), z \in
  \mathbb C$?\\
\textit{Solution:} \\
\textbf{Incomplete} \\
\item Consider $f_\epsilon(z)=\epsilon /\left(\epsilon^2+z^2\right)$, where
  $\epsilon$ is a small positive number, and $z \in \mathbb{C} /\{i
  \epsilon,-i \epsilon\}$. Plot $\left|f_\epsilon(z)\right|$, for various
  values of $\epsilon$. Discuss the influence the singularities of a
  function in the complex plane have on its behavior on the real
  line. Compute
  \begin{align*}
    \int_{-\infty}^\infty f_\epsilon(x) \D x.
  \end{align*}\\
\textit{Solution:} \\
\textbf{Incomplete} \\
\item Visualizing complex functions is not as easy as visualizing
  real-valued functions, since we need 4 dimensions: two for the input,
  two for the output. Different visualizations are commonly used, such
  as showing 3-dimensional plots of the real and imaginary
  parts. Plotting the modulus is informational, but it eliminates a
  lot of information.
  \begin{itemize}
\item To see this, plot the real and imaginary part of the exponential function $\exp (z)=\exp (x+\I y)$, for $x \in[-1,1], y \in[-2 \pi, 2 \pi]$. Now plot the modulus over the same region, and compare.
\item A "new" popular way to do this is to plot the modulus of the
function with the color defined by the phase. The \href{https://dlmf.nist.gov/}{Digital Library of
Mathematical Functions} has lots of examples. Create a plot of the
$|\exp (z)|=|\exp (x+\I y)|$, for $x \in[-1,1]$, $y \in[-2 \pi, 2
\pi]$. colored by the argument. Experimenting with other functions is
highly encouraged! The book visual Complex Functions: An Introduction
with Phase Portraits by Elias Wegert (Birkhäuser, 2012) is a good
companion to our textbook, if you think geometrically.
\end{itemize}
\textit{Solution:} \\
\textbf{Incomplete} \\
\item From A\&F: 2.1.1 \\
Which of the following satisfy the Cauchy-Riemann (C-R) equations? If they satisfy the C-R equations, give the analytic function of z. \\
a) $f(x, y) = x - iy + 1$ \\
\textit{Solution:} \\
Identify $u$ and $v$ and their partial derivatives
$$ u(x,y) = x + 1 \implies u_x = 1, \: u_y = 0$$
$$v(x,y) = - y \implies v_x = 0, \: v_y = -1$$
Therefore, $v_x = 0 = - 0 = - u_y$ holds, however $u_x = 1 \neq - 1 = v_y$ does not hold.
In conclusion $f(x, y) = x - iy + 1$ does not satisfy the Cauchy-Riemann (C-R) equations.\\
\qed

\noindent
b) $f(x, y) = y^3 - 3x^2y + i(x^3 - 3xy^2 + 2)$ \\
\textit{Solution:} \\
Identify $u$ and $v$ and their partial derivatives
$$ u(x,y) = y^3 - 3x^2y \implies u_x = -6xy, \: u_y = 3y^2 -3x^2$$
$$v(x,y) = x^3 - 3xy^2 + 2 \implies v_x = 3x^2 - 3y^2, \: v_y = -6xy$$
Therefore, $v_x = 3x^2 - 3y^2 = - (3y^2 -3x^2) = - u_y$ and $u_x = -6xy = -6xy = v_y$ both hold.
In conclusion $f(x, y) = y^3 - 3x^2y + i(x^3 - 3xy^2 + 2)$ satisfies the Cauchy-Riemann (C-R) equations.
And the analytic function of $z$ is \textbf{To Be Continued after you know what the analytic form of this function is}. \\
\qed

\noindent
c) $f(x, y) = e^y(\cos\theta + i\sin\theta)$ \\
\textit{Solution:} \\
Identify $u$ and $v$ and their partial derivatives
$$ u(x,y) = ... \implies u_x = ..., \: u_y = ...$$
$$v(x,y) = ... \implies v_x = ..., \: v_y = ...$$
Therefore, $v_x = ... = ... = - u_y$ and $u_x = ... = ... = v_y$ both hold.
In conclusion $f(x, y) = e^y(\cos\theta + i\sin\theta)$ ?satisfies? the Cauchy-Riemann (C-R) equations.
And the analytic function of $z$ is \textbf{To Be Continued after you know what the analytic form of this function is}. \\
\qed
\item From A\&F: 2.1.7 \\
Consider the complex analytic function, $\Omega(z) = \phi(x, y) + i\psi(x, y)$, in a domain $D$.
Let us transform from $z$ to $w$ using $w = f(z)$, $w = u + iv$ where $f(z)$ is analytic in $D$, with the corresponding domain in the $w$ plane, $D'$.
Establish the following: \\
\begin{eqnarray*}
\frac{\partial\phi}{\partial x} &=& \frac{\partial u}{\partial x} \frac{\partial\phi}{\partial u} + \frac{\partial v}{\partial x} \frac{\partial\phi}{\partial v} \\
\\
\frac{\partial^2\phi}{\partial x^2} &=& \frac{\partial^2 u}{\partial x^2} \frac{\partial\phi}{\partial u}
							- \frac{\partial^2 u}{\partial x \partial y} \frac{\partial\phi}{\partial v}
							+ \left(\frac{\partial v}{\partial x}\right)^2 \frac{\partial^2 \phi}{\partial u^2}
							- 2 \frac{\partial u}{\partial x} \frac{\partial u}{\partial y} \frac{\partial^2 \phi}{\partial u \partial v}
							+ \left(\frac{\partial u}{\partial y}\right)^2 \frac{\partial^2 \phi}{\partial v^2} \\
\end{eqnarray*}
Also find the corresponding formulae for $\frac{\partial \phi}{\partial y}$ and $\frac{\partial^2 \phi }{\partial y^2}$.
Recall that $f'(z) = \frac{\partial u }{\partial x} + i\frac{\partial u}{\partial y}$, and $u(x, y)$ satisfies Laplace's equation the domain D.
Show that
$$
\nabla_{x, y}^2 \phi = 
\frac{\partial^2\phi}{\partial x^2} + \frac{\partial^2 \phi }{\partial y^2} = 
\left(u_x^2 + y_y^2\right) \left(\frac{\partial^2\phi}{\partial u^2} + \frac{\partial^2 \phi }{\partial v^2}\right) =
\left| f'(z)\right|^2 \nabla_{u, v}^2 \phi
$$
Consequently, we find that if $\phi$ satisfies Laplace's equation $\nabla_{x, y}^2 \phi = 0$ in the domain $D$,
then so long as $f'(z) \neq 0$ in $D$ it also satisfies Laplace's equation $\nabla_{u, v}^2 \phi = 0$ in domain $D'$. \\
\textbf{Hint:} No Hints yet... \\
\textit{Solution:} \\
\textbf{Incomplete} \\
\item Show that the derivative of $f(z)=|z|^2$ is defined at $z=0$, but nowhere else.\\
\textit{Solution:} \\
Once again we are going to use the C-R equations and the fact that satisfying them is a necessary condition for differentiability.
\begin{eqnarray*}
f(z) &=&|z|^2 \\
      &=&\sqrt{x^2 + y^2}^2 \\
      &=& x^2 + y^2 \\
      &=& x^2 + y^2 + i \cdot 0
\end{eqnarray*}
Therefore $u(x, y) = x^2 + y^2$ and $v(x, y) = 0$. Now let's calculate the necessary partials
$$u_x = 2x$$ $$v_y = 0$$ $$v_x = 0$$ $$u_y = 2y.$$
Now we need the following to hold for $f(z)$ to be differentiable
\begin{eqnarray*}
u_x &=& v_y \\
2x &=& 0
\end{eqnarray*}
\text{and}
\begin{eqnarray*}
v_x &=& - u_y \\
0 &=& - 2y.
\end{eqnarray*}
Both of these only hold if $x=0$ and $y=0$ or in other words if $z=0$.
Which means the C-R equations only hold at $z=0$ and the derivative of $f(z) = |z|^2$ is defined at $z=0$ but nowhere else. \\
\qed

\item Derive the polar-coordinates form of the Cauchy-Riemann equations
$$
u_r=\frac{1}{r} v_\theta, \quad v_r=-\frac{1}{r} u_\theta.
$$
where $x=r \cos \theta$ and $y=r \sin \theta$ \\
\textit{Solution:} \\
We know we can represent $f(z) = u(x, y) + i v(x, y)$.
Let's make a substitution to polar coordinates
\begin{eqnarray*}
f(z) &=& u(x, y) + i v(x, y) \\
      &=& u(r \cos \theta, r \sin \theta) + i v(r \cos \theta, r \sin \theta) \\
      &=& U(r, \theta) + iV(r, \theta)
\end{eqnarray*}
We need to calculate the derivatives using the chain rule
$$U_r = u_xx_r + u_yy_r$$
$$U_\theta = u_xx_\theta + u_yy_\theta$$
$$V_r = v_xx_r + v_yy_r$$
$$V_\theta = v_xx_\theta + v_yy_\theta.$$

Now we compute the necessary partials and plug them back in
\begin{eqnarray*}
x_r &=& \cos \theta \\
x_\theta &=& - r \sin \theta \\
y_r &=& \sin \theta \\
y_\theta &=& r \cos \theta
\end{eqnarray*}
$$\downarrow$$
\begin{eqnarray*}
U_r &=& u_x \cos \theta + u_y \sin \theta \\
U_\theta &=& - u_x r \sin \theta + u_y r \cos \theta \\
V_r &=& v_x \cos \theta  + v_y  \sin \theta \\
V_\theta &=& - v_x r \sin \theta + v_y r \cos \theta.
\end{eqnarray*}
Now try to make them look like each other with the C-R equations
\begin{eqnarray*}
U_r &=& u_x \cos \theta + u_y \sin \theta \\
       &=& v_y \cos \theta + u_y \sin \theta
\end{eqnarray*}
then
\begin{eqnarray*}
V_\theta &=& - v_x r \sin \theta + v_y r \cos \theta \\
	      &=&  u_y r \sin \theta + v_y r \cos \theta \\
	      &=&  r (u_y \sin \theta + v_y \cos \theta ) \\
	      &=&  r (v_y \cos \theta + u_y \sin \theta) \\
	      &=&  r U_r
\end{eqnarray*}
Therefore $U_r = \frac{1}{r}V_\theta$.
\begin{eqnarray*}
V_r &=& v_x \cos \theta  + v_y  \sin \theta \\
       &=& -u_y \cos \theta  + v_y  \sin \theta
\end{eqnarray*}
then
\begin{eqnarray*}
U_\theta &=& - u_x r \sin \theta + u_y r \cos \theta \\
	      &=& - v_y r \sin \theta + u_y r \cos \theta \\
	      &=& - r (v_y \sin \theta - u_y \cos \theta) \\
	      &=& - r (- u_y \cos \theta + v_y \sin \theta) \\
	      &=& - r V_r
\end{eqnarray*}
Therefore $V_r = - \frac{1}{r}U_\theta$.
And with this we have derived the polar-coordinates form of the Cauchy-Riemann equations. \\
\qed
\end{enumerate}

\end{document}

%%% Local Variables:
%%% mode: latex
%%% TeX-master: t
%%% End:
