\documentclass[10pt]{amsart}
\usepackage[margin=1.4in]{geometry}
\usepackage[usenames,dvipsnames,cmyk]{xcolor} %load first
\usepackage{cancel}
\usepackage{graphicx,subfig}
\usepackage{mathtools}

\graphicspath{ {./images/} }

\usepackage{amssymb,amsmath,enumitem,url}

\newcommand{\D}{\mathrm{d}}
\newcommand{\I}{\mathrm{i}}
\DeclareMathOperator{\E}{e}
\DeclareMathOperator{\OO}{O}
\DeclareMathOperator{\oo}{o}
\DeclareMathOperator{\erfc}{erfc}
\DeclareMathOperator{\real}{Re}
\DeclareMathOperator{\imag}{Im}
\DeclareMathOperator{\sech}{sech}
\DeclareMathOperator{\csch}{csch}
\usepackage{tikz}
\usepackage[framemethod=tikz]{mdframed}
\theoremstyle{nonumberplain}

\mdtheorem[innertopmargin=5pt]{lemma}{Lemma}
\mdtheorem[innertopmargin=-5pt]{sol}{Solution}
%\newmdtheoremenv[innertopmargin=-5pt]{sol}{Solution}
\definecolor{MichiganBlue}{HTML}{00274C}
\definecolor{MichiganYellow}{HTML}{FFCB05}  
\definecolor{NicePurple}{RGB}{75,56,76} %PrincePurple
\definecolor{NiceRed}{RGB}{230,37,52}
\definecolor{MidnightBlue}{rgb}{0.1, 0.1, 0.44}
\usepackage[colorlinks=true, linkcolor=MidnightBlue, citecolor=MidnightBlue, urlcolor=MidnightBlue]{hyperref}

\begin{document}
\pagestyle{empty}

\newcommand{\mline}{\vspace{.2in}\hrule\vspace{.2in}}


\title{\bf {AMATH 567 Complex Analysis Final Study Guide} }


\maketitle
\noindent
\begin{center}
The final is scheduled for Monday December 9th at 2:30pm in Johnson Hall 075.
\end{center}

\mline
\noindent
Be able to use the important theorem's and know their respective assumptions and limitations.
Be able to state examples and counter examples for each.
Be comfortable with all important computation applications of the main principals.
\\

\noindent
Below I've listed some key topics from memory (trying to force myself to recall).
Each of these should be supplemented by looking in the textbook, Bernard's notes, and homework assignments for more information related to the listed topics.
\\

\begin{enumerate}[label={\bf {\arabic*}:}]
\item Basic complex arithmetic
\begin{itemize}
\item $z = x + \I y$
\item $z = \rho \E^{\I \theta} = \rho ( \cos \theta + \I \sin \theta )$
\item $\bar z = x - \I y$
\item $|z| = \sqrt{x^2 + y^2} = \rho$
\item $z \bar z = |z|^2$ \\
\end{itemize}

\item Roots of unity, logs, and the square of complex number
\begin{itemize}
\item $z^n = a = |a|\E^{\I (\theta + 2 \pi k)} \implies
	z = |a|^{1/n}\E^{\I(\theta + 2 \pi k)/n} \text{ with } k \in \mathbb Z$
\item $\log z = \log r + \I \theta + 2 \pi \I k \text{ with } k \in \mathbb Z$ \\
\end{itemize}

\item Branch Cuts and branch points
\begin{itemize}
\item Figure out what the heck they \textit{really} are and how they apply \\
\end{itemize}

\item Analyticity, Cauchy Riemann Equations, Cauchy's Theorem, and Cauchy Integral Formula
\begin{itemize}
\item Remember the { \bf Cauchy-Riemann equations }? If $f(z) = f(x,y) = u(x, y) + \I v(x,y)$ then $f(z)$ is analytic if and only if $$\frac{\partial u}{\partial x} = \frac{\partial v}{\partial y} \quad \text{ and } \quad \frac{\partial v}{\partial x} = - \frac{\partial u}{\partial y}.$$
\item \textbf{Cauchy's Theorem:} {\it If $f(z)$ is analytic in a simply connected region $\Omega$ which contains the closed contour $C$ then $$ \oint_C f(z) \D z = 0. $$}
\item \textbf{Cauchy's Integral Formula:} {\it If $f(z)$ is analytic on and interior to a simple closed contour $C$ then for any point $z$ interior to $C$ $$f(z) = \frac 1 {2 \pi \I}\oint_C \frac {f(\zeta) \D \zeta}{\zeta - z}. $$}
\item Derivative's using \textbf{Cauchy's Integral Formula.} {\it If $f(z)$ is analytic on and interior to a simple closed contour $C$ then all derivatives $f^{(k)}(z)$, $k = 1, 2, ...$ exist in the domain $D$ interior to $C$, and $$f^{(k)}(z) = \frac {k!} {2 \pi \I}\oint_C \frac {f(\zeta) \D \zeta}{(\zeta - z)^{k + 1}}. $$} \\
\end{itemize}

\item Contour Integration \\
Just be really good at it that's it. \\
\textbf{TODO: } Maybe add a few methods of solving these problems in general cases.\\

\item Analytic Continuation, Entire Functions, Maximum Modulus Principal, and Liouville's Theorem \\

\item Taylor and Laurent Series representations of complex valued functions \\

\item Singularities and their types \\

\item Residue Theorem and applications \\

\item Miscellaneous things to have memorized (Taylor/Laurent series representations and others)
\begin{enumerate}
\item Know the series representations of $\E^z$, $\sin z$ and $\cos z$
\item Know the exponential representation of $\sin z$ and $\cos z$
\item Know helpful representations for $\tan z$, $\sec z$, $\csc z$, $\cot z$, $\sinh z$, $\cosh z$, $\tanh z$, $\coth z$, $\sech z$, and $\csch z$ in terms of exponentials, sines, and cosines.
\item Geometric series
\item
\end{enumerate}

\end{enumerate}

\end{document}

%%% Local Variables:
%%% mode: latex
%%% TeX-master: t
%%% End:
