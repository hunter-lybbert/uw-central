\documentclass[10pt]{amsart}
\usepackage[margin=1.4in]{geometry}
\usepackage{amssymb,amsmath,enumitem,url}
\usepackage{graphicx,subfig}
\graphicspath{ {./images/} }

\newcommand{\D}{\mathrm{d}}
\newcommand{\I}{\mathrm{i}}
\DeclareMathOperator{\E}{e}
\DeclareMathOperator{\OO}{O}
\DeclareMathOperator{\oo}{o}
\DeclareMathOperator{\erfc}{erfc}
\DeclareMathOperator{\real}{Re}
\DeclareMathOperator{\imag}{Im}
\usepackage{tikz}
\usepackage[framemethod=tikz]{mdframed}
\theoremstyle{nonumberplain}

\mdtheorem[innertopmargin=-5pt]{sol}{Solution}
%\newmdtheoremenv[innertopmargin=-5pt]{sol}{Solution}

\begin{document}
\pagestyle{empty}

\newcommand{\mline}{\vspace{.2in}\hrule\vspace{.2in}}

\noindent
\text{Hunter Lybbert} \\
\text{Student ID: 2426454} \\
\text{03-31-25} \\
\text{HMS 581:}
\text{Infectious Disease Modeling} \\
% header containing your name, student number, due date, course, and the homework number as a title.

\title{\bf {Quiz} }


\maketitle
\noindent
Solutions to the problems which are described in the entrance quiz to \textit{HMS 581}.
\mline
\begin{enumerate}[label={\bf {Question \arabic*}}]
\item (Linear Algebra): \\
\begin{enumerate}

\item Dominant Eigenvector and corresponding eigenvalue of
$$A =
\begin{bmatrix}
0.5 & 0.25 \\
0.5 & -0.125
\end{bmatrix}
$$

\textit{Solution:} \\
Solving using the characteristic equation gives us the following
\begin{align*}
(0.5 - \lambda)(-0.125-\lambda) - 0.5(0.25) &= 0 \\
\lambda^2 -0.5\lambda + 0.125\lambda + 0.5(-0.125) - 0.5(0.25) &= 0 \\
\lambda^2 -0.375\lambda -0.0625 - 0.125 &= 0 \\
\lambda^2 -0.375\lambda -0.1875 &= 0 \\
\implies \lambda = \frac{0.375 \pm \sqrt{(-0.375)^2 - 4(-0.1875)}}{2} &.
\end{align*}
I have verified this with the results from calculating it with NumPy in Python.
The dominant eigenvector is the eigenvector corresponding to the eigenvalue which is largest in absolute value.
In our case the larger eigenvalue in absolute value is 
$\lambda = 0.65936465$ with the eigenvector
$$
v_1 = \begin{bmatrix}
0.84324302 \\
0.53753252
\end{bmatrix}.
$$

\item If $x = [1 \;\: 1]^T$, let $y = Ax$. Further let $|\cdot|_2$ denote the $l^2$ norm or Euclidean distance. Without doing the calculation which will be larger $|x|_2$ or $|y|_2$? Why? \\

\textit{Solution:} \\
I believe $|x|_2$ will be larger.
If my linear algebra memory is serving me correctly the dominant eigenvalue being less than 1 would imply that the linear transformation which the matrix $A$ represents is compressing input vectors that it is operating on.
Therefore, the resultant $|y|_2$ would be smaller since $A$ is compressing input vectors. \\

\item Let $\hat x$ and $\hat y$ denote the normalized vectors.
We calculated $| \hat x - v_1|$ and $|\hat y - v_1|$.

\textit{Solution:} \\
I might have been able to guess that the second quantity is smaller since the matrix $A$ is compressing values and they are getting closer to the dominant eigenvalue.

\end{enumerate}
\qed \\

\item (ODE Question): \\

\item (Coding Question...with probability): \\

\item (Epidemiology question): \\

\end{enumerate}

\end{document}

%%% Local Variables:
%%% mode: latex
%%% TeX-master: t
%%% End:
